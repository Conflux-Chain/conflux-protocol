% !TEX root = ./tech-specification.tex

%%%%%%%%%%%%%%%%%%%%%%%%%%%%%%%%%%%%%%%%

\subsection{Execution Model}
\label{sec:exe model}

The execution model specifies the system state transition on input of a sequence of bytecode instructions and a small tuple of environmental data. 
The state transition function is formalized as a virtual state machine,
which 
% This virtual machine 
is Turing-complete except that its running time and storage space are intrinsically bounded by the limited amount of available gas and collateral for storage.
% 
For this moment we implement the well-known Ethereum Virtual Machine (EVM), and the execution model follows \cite{ETH_yellow}.


\subsubsection{Basics}

The \cvm is a stack-based architecture with $256$-bit word size.
The stack has a maximum size of $1024$ words.
The memory model is a simple word-addressed byte array.  
The machine also has an independent storage model which is a word-addressable word array (rather than byte array for the memory). 
The memory is volatile and storage is steady and maintained as part of the system state. 
All locations in both memory and storage are initialized as zero.
The program code is stored separately in a virtual ROM that is only interactable via specific instructions.

The execution of the virtual machine may reach exceptions for various reasons, including stack underflows/overflow, invalid instruction, invalid jump destination, out-of-gas and so on.
Like the out-of-gas exception, the machine halts immediately and throws an exception to the execution agent, either the transaction processor or recursively the spawning execution environment, which will catch and deal with it separately. 



\subsubsection{Gas Consumption}

The cost of execution, aka. \emph{gas}, is charged under following circumstances:
\begin{enumerate}[nosep]
	\item the execution of instructions, where each type of instructions is assigned an intrinsic amount of gas;

	\item the generation of subordinate message call or contract creation.

\end{enumerate}


\subsubsection{Storage Consumption}
	\label{subsec:storage consumption}

	For core space, \name requires a fixed amount of fund, i.e. \sunitprice, locked as collateral during the whole lifetime of each \sunitsize storage entry in the world-state.
	This fund is locked when the entry is created, and is unlocked and returned to the owner when that entry is cleared or overwritten by someone else eventually, as described in Section~\ref{sec:collateral}.
	The interest generated by the collateral is paid to miners as specified in Section~\ref{subsec:storagefee}. 
	Thus the cost of storing an entry  
	is proportional to the time length of storage usage.

	
	The owner of the collateral of a storage entry, 
	which is called ``the owner of that entry'' for simplicity, 
	essentially records who has written the latest content of that entry.
	Normally the initial owner of an entry should be the sender of the transaction that causes the creation of this entry. 
	However, in case a contract provides the collateral on behalf of the sender, the owner will be that contract instead (see Section~\ref{sec:sponsor} for details).
	When a storage entry is modified in the execution of a transaction,
	the ownership of this entry is changed,
	and the old owner's collateral for that entry is replaced by the new owner's collateral.


	If a storage entry is cleared from the world-state,
	then the corresponding collateral is unlocked and returned to the owner of that entry.
	We remark that there is no refund to the actor who causes the clearance, 
	which is distinct from the gas refunding policy in Ethereum \cite{ETH_yellow}.
	Furthermore, to ensure that unlocked collateral for storage is always returned properly, 
	\name does not allow destructing any smart contract with non-zero collateral for storage.



\subsubsection{Execution Environment}
\label{subsubsec:exe_env}
Besides the global system state $\st$ and the amount of remaining gas $g$, 
the execution agent must provide the following important information used in the execution environment, as contained in the tuple $I$:
\begin{itemize}[nosep]
	\item $I_a$, the address of the account which owns the code that is executing.
	
	\item $I_c$, the space the transaction belongs to. 
	
	\item $I_o$, the address of the original sender who originated this execution.
	
	\item $I_i$, the address of the storage owner.

	\item $I_p$, the gas price designated by the transaction that originated this execution.

	\item $I_{\vec{d}}$, the byte array that is the input data to this execution; in case the execution agent is a transaction $\tx$, this would be the transaction data $\tx_{\vec{d}}$.

	\item $I_s$, the address of the account that invoked the code; in case the execution agent is a transaction $\tx$, this would be the transaction sender's address $\sender{\tx}$.

	 \item $I_v$, the value, in \unit, passed to the recipient's account; in case the execution agent is a transaction $\tx$, this would be the transaction value $\tx_v$.

	 \item \linkdest{I__b}{$I_{\vec{b}}$}, the byte array of the machine code to be executed.

	 \item $I_{\head}$, the block header of the present block.

	 \item $I_e$, the depth of the current message-call or contract-creation in the stack.

	 \item $I_w$, the permission to make modifications to the state.

	 \item $I_\st$, the original world-state right before this execution.
\end{itemize}

The state transition is defined by the execution function $\execute$, which takes as input the current world-state $\st$, the amount of gas $g$, 
and the input $I$ as defined above,
and outputs the resultant state $\st'$, the remaining gas $g'$, the accrued substate $A$ and the resultant output $\vec{o}$.
Formally, we define it as follows:
\begin{align}
	\left( \st', g', A, \vec{o} \right) \eqdef \execute\left(\st,g, I\right)
\end{align}
where we recall that the accrued state $A$ consists of the selfdestructs set $A_\vec{s}$, the log series $A_\vec{l}$, the touched accounts $A_\vec{t}$, a series of addresses recording the owners of storage occupation $A_\vec{o}$ and a series of addresses recording the owners of storage release $A_\vec{e}$
(as described in Section~\ref{subsubsec:substate}):
\begin{align}
	A\eqdef\left(A_\vec{s},A_\vec{l},A_\vec{t},A_\vec{o},A_\vec{e} \right)
\end{align}


\subsubsection{Execution Overview}

The $\execute$ function is defined mostly following the Ethereum yellowpaper \cite{ETH_yellow}, except for a few instructions. 
For self-sufficiency we explain the definition of $\execute$ briefly.

In most practical implementations $\execute$ will be modeled as an iterative progression of the pair $(\st,\mst)$ comprising the world-state and the machine state. 
Formally, it can be recursively defined with a function $X$. This uses an iterator function $O$ (which defines the result of a single cycle of the state machine) together with functions \hyperlink{zhalt}{$Z$}, which determines if the present state is an \hyperlink{zhalt}{exceptional halting} state of the machine, and \hyperlink{hhalt}{$H$}, specifying the output data of the instruction if and only if the present state is a \hyperlink{hhalt}{normal halting} state of the machine.

Recall that the empty sequence, denoted by $\emptystring$, is not equal to the empty set, denoted by $\varnothing$; this is important when interpreting the output of $H$, which evaluates to $\varnothing$ when execution is to continue but a series (potentially empty) when execution should halt.
\begin{eqnarray}
\execute(\st, g, I) & \eqdef & (\st'\!, \mst'_{\mathrm{g}}, A, \mathbf{o}) \\
(\st', \mst'\!, A, ..., \mathbf{o}) & \eqdef & X\big((\st, \mst, A^0\!, I)\big) \\
\mst_{\mathrm{g}} & \eqdef & g \\
\mst_{\mathrm{pc}} & \eqdef & 0 \\
\mst_{\mathbf{m}} & \eqdef & (0, 0, ...) \\
\mst_{\mathrm{i}} & \eqdef & 0	\\
\mst_{\mathbf{s}} & \eqdef & \emptystring \\
\mst_{\mathbf{o}} & \eqdef & \emptystring	\\
\mst_{\mathbf{r}} & \eqdef & \emptystring
\end{eqnarray}
\begin{equation} \label{eq:X-def}
X\big( (\st, \mst, A, I) \big) \eqdef \begin{cases}
\big(\varnothing, \mst, A^0, I, \emptystring\big) & \text{if} \quad Z(\st, \mst, A, I) \\
\big(\varnothing, \mst', A^0, I, \mathbf{o}\big) & \text{if} \quad w =  \op{REVERT} \\
O(\st, \mst, A, I) \cdot \mathbf{o} & \text{if} \quad \mathbf{o} \neq \varnothing \\
X\big(O(\st, \mst, A, I)\big) & \text{otherwise} \\
\end{cases}
\end{equation}

where
\begin{eqnarray}
\mathbf{o} & \eqdef & H(\mst, I) \\
(a, b, c, d) \cdot e & \eqdef & (a, b, c, d, e) \\
\mst' & \eqdef & \mst\ \text{except:} \\
\mst'_{\mathrm{g}} & \eqdef & \mst_{\mathrm{g}} - C(\st, \mst, I)
\end{eqnarray}

Note that, when evaluating $\execute$ instead of $X$, 
the fourth element $I'$ is dropped and the remaining gas $\mst'_{\mathrm{g}}$ is extracted from the resultant machine state $\mst'$.

$X$ is thus cycled (recursively here, but implementations are generally expected to use a simple iterative loop) until either \hyperlink{zhalt}{$Z$} becomes true indicating that the present state is exceptional and that the machine must be halted and any changes discarded or until \hyperlink{hhalt}{$H$} becomes a series (rather than the empty set) indicating that the machine has reached a controlled halt.

\paragraph{Machine State.}
The machine state $\mst$ is defined as the tuple $(g, \mathrm{pc}, \vec{m}, i, \vec{s},\vec{r})$ which are the gas available, the program counter $pc \in \N_{256}$ , the memory contents, the active number of words in memory (counting continuously from position 0), the data stack contents and return stack contents. 
The memory contents $\mst_{\vec{m}}$ are a series of zeros of size $2^{256}$.
The return stack $\mst_{\vec{r}}$ is limited to $1023$ items.

For the ease of reading, the instruction mnemonics, e.g. $\op{ADD}$, should be interpreted as their numeric equivalents; the full table of instructions and their specifics are given in Appendix \ref{app:instruction-set}.

For the purposes of defining $Z$, $H$ and $O$, we define $w$ as the current operation to be executed:
\begin{equation}\label{eq:currentoperation}
w \eqdef 
	\begin{cases} 
		I_{\vec{b}}[\mst_{\mathrm{pc}}] & \text{if} \quad \mst_{\mathrm{pc}} < \lVert I_{\vec{b}} \rVert \\
		\hyperlink{stop}{\op{STOP}} & \text{otherwise}
	\end{cases}
\end{equation}

Furthermore, we let $\popstack$ and $\pushstack$ denote the fixed number of stack items removed from and pushed into the data stack $\mst_{\mathbf{s}}$ by executing an instruction.
Both $\popstack$ and $\pushstack$ are assumed subscriptable on the instruction. 
Similarly, we define $\poprstack$ and $\pushrstack$ for the return stack $\mst_{\mathbf{r}}$, which is only accessed when entering or returning from subroutines on $\op{JUMPSUB}$ and $\op{RETURNSUB}$ instructions. 
An instruction cost function $\cost$ evaluates to the full cost, in gas, of executing the given instruction.

\paragraph{Exceptional Halting.}\hypertarget{Exceptional_Halting_function_Z}{}\linkdest{zhalt}
%
The exceptional halting function $Z$ is defined as:
\begin{equation}
	Z(\st, \mst, A, I) \eqdef
	\begin{array}[t]{ll}
		& \mst_g < C(\st, \mst, I) \quad \\
		\vee & \popstack_w = \varnothing \quad \\
		\vee & \lVert\mst_\mathbf{s}\rVert < \popstack_w \quad \\
		\vee & \left( w =  \op{JUMP} \quad \wedge \quad \mst_\mathbf{s}[0] \notin D(I_\mathbf{b})  \right) \quad \\
		\vee & \left( w =  \op{JUMPI} \quad \wedge \quad \mst_\mathbf{s}[1] \neq 0 \quad \wedge \quad \mst_\mathbf{s}[0] \notin D(I_\mathbf{b})  \right) \quad \\
		\vee & \left( w = \op{RETURNDATACOPY} \quad \wedge \quad \mst_{\mathbf{s}}[1] + \mst_{\mathbf{s}}[2] > \lVert\mst_{\mathbf{o}}\rVert \right) \quad \\
		\vee & \lVert\mst_\mathbf{s}\rVert - \popstack_w + \pushstack_w > 1024 \quad\\ 
		\vee & \left(\neg I_{\mathrm{w}} \quad \wedge \quad W(w, \mst)\right)\quad 
	\end{array}
\end{equation}
where
\begin{equation}
W(w, \mst) \eqdef \\
\begin{array}[t]{ll}
	& w \in \{ \op{CREATE},  \op{CREATE2},  \op{SSTORE}, \op{SUICIDE}\} \quad \\
	\vee & \left(\op{LOG0} \le w \wedge w \le  \op{LOG4} \right)\\
	\vee & \left(w \in \{ \op{CALL},  \op{CALLCODE}\} \wedge \mst_{\mathbf{s}}[2] \neq 0\right)
\end{array}
\end{equation}

This states that the execution is in an exceptional halting state if there is insufficient gas, if the instruction is invalid (and therefore its $\popstack$ subscript is undefined), if there are insufficient stack items, if a $\op{JUMP}$/$\op{JUMPI}$ destination is invalid, 
if the output data size $\lVert\mst_{\mathbf{o}}\rVert$ is insufficient for the copy-output-data operation specified in a $\op{RETURNDATACOPY}$ instruction,
or if the new stack size would be larger than $1024$ or state modification is attempted during a static call. The astute reader will realize that this implies that no instruction can, through its execution, cause an exceptional halt.

\paragraph{Jump Destination Validity.}
We previously used $D$ as the function to determine the set of valid jump destinations given the code that is being run. We define this as any position in the code occupied by a  $\op{JUMPDEST}$ instruction.

All such positions must be on valid instruction boundaries, rather than sitting in the data portion of  $\op{PUSH*}$ operations and must appear within the explicitly defined portion of the code (rather than in the implicitly defined \hyperlink{stop}{$\op{STOP}$} operations that trail it).

Formally:
\begin{equation}
D(\mathbf{c}) \eqdef D_{J}(\mathbf{c}, 0)
\end{equation}

where:
\begin{equation}
D_{J}(\mathbf{c}, i) \eqdef \begin{cases}
\{\} & \text{if} \quad i \geqslant \lVert \mathbf{c} \rVert  \\
\{ i \} \cup D_{J}(\mathbf{c}, N(i, \mathbf{c}[i])) & \text{if} \quad \mathbf{c}[i] =  \op{JUMPDEST} \\
D_{J}(\mathbf{c}, N(i, \mathbf{c}[i])) & \text{otherwise} \\
\end{cases}
\end{equation}

where $N$ is the next valid instruction position in the code, skipping the data of a  $\op{PUSH*}$ instruction, if any:
\begin{equation}\label{eq:next-instruction}
N(i, w) \eqdef \begin{cases}
i + w -  \op{PUSH1} + 2 & \text{if} \quad w \in [ \op{PUSH1},  \op{PUSH32}] \\
i + 1 & \text{otherwise} \end{cases}
\end{equation}

\paragraph{Normal Halting.}\hypertarget{normal_halting_function_H}{}\linkdest{hhalt}

The normal halting function $H$ is defined:
\begin{equation}
H(\mst, I) \eqdef 
	\begin{cases}
	H_{\text{\tiny RETURN}}(\mst) & \text{if} \quad w \in \{ \op{\hyperlink{RETURN}{RETURN}},  \op{REVERT}\} \\
	\emptystring \quad\quad& \text{if} \quad w \in \{  \op{\hyperlink{stop}{STOP}},  \op{\hyperlink{selfdestruct}{SUICIDE}} \} \\
	\varnothing \quad\quad& \text{otherwise}
	\end{cases}
\end{equation}

The data-returning halt operations, \hyperlink{RETURN}{ \op{RETURN}} and  \op{REVERT}, have a special function $H_{\text{\tiny RETURN}}$. Note also the difference between the empty sequence and the empty set as discussed \hyperlink{empty_sequence_vs_empty_set}{here}.

\subsubsection{The Execution Cycle}

Stack items are added or removed from the left-most, lower-indexed portion of the series; all other items remain unchanged:
\begin{eqnarray}
O\big((\st, \mst, A, I)\big) & \eqdef & (\st', \mst', A', I) \\
\Delta & \eqdef & \pushstack_{w} - \popstack_{w} \\
\lVert\mst'_{\mathbf{s}}\rVert & \eqdef & \lVert\mst_{\mathbf{s}}\rVert + \Delta \\
\quad \forall x \in \left[ \pushstack_{w}, \lVert\mst'_{\mathbf{s}}\rVert -1 \right]: \mst'_{\mathbf{s}}[x] & \eqdef & \mst_{\mathbf{s}}\left[x-\Delta \right]
\end{eqnarray}

The gas is reduced by the instruction's gas cost.
\begin{eqnarray}
	\quad \mst'_{g} & \eqdef & \mst_{g} - C(\st, \mst, I) \label{eq:mu_pc}
\end{eqnarray}

For most instructions, the program counter $\mathrm{pc}$ increases by $1$ on each cycle, except for following instructions: $\op{PUSH*}$, $\op{JUMP}$, $\op{JUMPI}$, $\op{JUMPSUB}$, $\op{RETURNSUB}$.
The next valid instruction position for $\op{PUSH*}$ instructions is already specified in $N$ as in \cref{eq:next-instruction}. 
We assume a function $J$, subscripted by one instruction from $\big\{\op{JUMP}$, $\op{JUMPI}$, $\op{JUMPSUB}$, $\op{RETURNSUB}\big\}$, which evaluates to the according value:
\begin{eqnarray}\label{eq:u_pc}
	\quad \mst'_{\mathrm{pc}} & \eqdef & 
	\begin{cases}
		\hyperlink{JUMP}{J_{\op{JUMP}}}(\mst) & \text{if} \quad w =  \op{JUMP} \\
		\hyperlink{JUMPI}{J_{\op{JUMPI}}}(\mst) & \text{if} \quad w =  \op{JUMPI} \\
		\hyperlink{JUMPSUB}{J_{\op{JUMPSUB}}}(\mst) & \text{if} \quad w =  \op{JUMPSUB} \\
		\hyperlink{RETURNSUB}{J_{\op{RETURNSUB}}}(\mst) & \text{if} \quad w =  \op{RETURNSUB} \\
		N(\mst_{\mathrm{pc}}, w) & \text{otherwise}
	\end{cases}
\end{eqnarray}

In general, we assume the memory, self-destruct set and system state do not change:
\begin{eqnarray}
\mst'_{\mathbf{m}} & \eqdef & \mst_{\mathbf{m}} \\
\mst'_{\mathrm{i}} & \eqdef & \mst_{\mathrm{i}} \\
A' & \eqdef & A \\
\st' & \eqdef & \st
\end{eqnarray}

However, instructions do typically alter one or several components of these values. Altered components listed by instruction are noted in Appendix \ref{app:vm}, alongside values for $\pushstack$, $\popstack$, $\pushrstack,\poprstack$ and a formal description of the gas requirements.

\subsubsection{Difference from Ethereum for core space}
The execution function $\execute$ follows nearly the same definition as in Ethereum yellowpaper \cite{ETH_yellow} except for a few instructions. 
When executing $O(\st, \mu, A, I) \eqdef (\st' , \mu' , A' , I')$ 
for the iterator function $O$ which defines the result of a single cycle of the state machine,
{\name} differs from Ethereum on following instructions. 


\paragraph{Sub-call operations.} 
{\name} has two additional parameters comparing with Ethereum: 
the recipient addresses call-state $\vec{t}$ and the storage owner $i$.
In sub-call operations such as $\op{CREATE}$, $\op{CALL}$, $\op{CALLCODE}$, $\op{DELEGATECALL}$, and $\op{STATICCALL}$,  
the recipient addresses $I_\vec{t}\cdot I_a$ and storage owner $I_i$ are passed to $\creation$ and $\execution$ as part of the execution environment $I$. 

\paragraph{Re-entrance Protection (Disabled since CIP-71).}

Before CIP-71, when calling a contract, {\name} virtual machine makes sure that re-entrance attack is impossible by preventing re-entrance message call, except the message call matches some requirements which make re-entrance attack impossible. 

To be specific, the {\name} virtual machine maintains a call stack $I_\vec{t}$ \footnote{In actual implementation, the top element in $I_\vec{t}$ is lost. This bug persists until this feature is disabled.}
and enters \emph{reentrance protection mode} in a message call
when the callee is already in the call stack but different from the caller before executing the code invoked by each message call.
By requiring that the callee being different from the caller, it is still allowed to call and execute other functions in the caller's contract.
Because in such cases the developer should be able to fully anticipate the execution flow
and we do not consider it necessary to trigger the re-entrance protection.
%
The execution with \emph{reentrance protection mode} is exactly the same as execution with static flag set by $\op{STATICCALL}$. 






\paragraph{$\op{SSTORE}$ operation.} 
The $\op{SSTORE}$ operation transforms $(\st,A)$ into $(\st',A')$ as follows:

\begin{align}
	(\st', A^*)   &\eqdef \Phi(\st,I_a, \mst_{\sf s}[0],\mst_{\sf s}[1],I_i) \\ 
	A'     &\eqdef A \Cup A^*
\end{align}
%
where $\Phi$ is defined in section~\ref{sec:storage_maintain}.

In Ethereum, the cost of operation $\op{SSTORE}$ is $G_{sset}=20000$ gas when the storage value is set to non-zero from zero,
and  $G_{sreset}=5000$ gas when the storage value is set to zero. 
Ethereum will also refund $R_{sclear}=15000$ gas when the storage value is set to zero from non-zero. 

In {\name}, since the cost of using storage is reflected by collateral for storage, there is no need to charge space consumption in gas. 
Thus {\name} charged $G_{sset}=5000$ gas for all the $\op{SSTORE}$ operation, 
regardless of the storage value,
and there is no gas refund either. 
Furthermore, the {\name} ledger $\st$ tracks the owner of every storage entry with non-zero value. 
The execution substate $A$ records all changes on ownership of storage entries. 

\paragraph{$\op{SUICIDE}$ operation.} When executing the $\op{SUICIDE}$ operation, if the address receiving refund balance is invalid (i.e. \\$\mathsf{Type}_{\mst_{\mathbf{s}}[0]\mod2^{160}}\notin\set{\typereserved,\typenormal,\typecontract}$) the refund balance will be burnt. Otherwise, the $\op{SUICIDE}$ operation transforms $(\st,A)$ into $(\st',A')$ following the same steps as Ethereum. 

% Compared with Ethereum, Conflux updates the storage collateral of code owner for destructed contract and refunds the sponsor balance for gas at this time. Notice that the sponsor balance for collateral will not be refunded until all the storage collateral has been charged from or refunded to account balance. 

\paragraph{Subroutine operation.} Ethereum introduces $\op{BEGINSUB}$, $\op{JUMPSUB}$, $\op{RETURNSUB}$ instructions in EIP-2315, which is not listed in its yellowpaper. 
{\name} implements this instruction with identical behavior as Ethereum.