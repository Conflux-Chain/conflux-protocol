% !TEX root = ./tech-specification.tex

%%%%%%%%%%%%%%%%%%%%%%%%%%%%%%%%%%%%%%%%

\section{Transaction Processing}
\label{sec:tx_processing}

\name implements the same virtual machine as Ethereum \cite{ETH_yellow}. 
The execution of a transaction defines the transform function $\Upsilon(\st,\tx,\vec{L})$, which is similar with Ethereum's state transition function.

\subsection{Overview}

In what follows we focus on the \name specific designs in the execution.

\paragraph{Spaces (Since CIP-90)}

Each transaction belongs to one space. The sender and the receiver of a transaction always belongs to the same space in a transaction. 

\paragraph{Gas and Payment}\label{subsec:gas_and_pay}

As defined in Section~\ref{sec:tx} every transaction $\tx$ has two fields of {\bf gasLimit} and {\bf gasPrice} that declare the specific amount of associated gas $\tx_g$ and the price $\tx_p$ of per unit gas.
When starting the execution of a transaction $\tx$, the purchase of gas happens at the price $\tx_g \times \tx_p$ and the transaction $\tx$ is considered invalid if the actor responsible for the cost of gas consumption cannot afford such a purchase.
% , i.e. $\sender{\tx}_b < \tx_g \times \tx_p$.
Normally $\sender{\tx}$, the sender of $\tx$, is responsible for the cost of gas consumption. But the gas consumption maybe sponsored by called contract sometimes. (See the following paragraph for details.)

Like in Ethereum, gas does not exist outside the execution of transactions.

The unused gas can be refunded after the transaction $\tx$ is executed, but no more than a quarter of the total value spent on purchasing. 
Thus, the \emph{refundable amount of gas} $g^{\dagger}$ is the minimum of the \emph{legitimately remaining gas} $g'$ and a quarter of the \textbf{gasLimit} of $\tx$,
i.e. $g^{\dagger}\eqdef \min\set{g', \tx_g/4}$, 
where in principle no gas is refundable (i.e. $g^{\dagger} = g' = 0$) if the execution of $\tx$ fails due to the sender's fault. 

% The \emph{consumed amount of gas}\footnote{Here for simplicity we write $\gused(\cdot)$ as a function of $\tx$. However it is indeed a function defined on both $\tx$ and the world-state at the beginning of the execution of $\tx$. Thus, multiple occurrences of $\tx$ are considered as distinct inputs for this function and they may incur different amount of consumed gas.} $\gused(\tx)$ is defined as
% \begin{align*}
% 	\gused(\tx) \eqdef \begin{cases}
% 		\tx_g-g^{\dagger} & \mbox{if $\tx$ is executed}\\
% 		0 & \mbox{if $\tx$ is not executed (i.e. only when $R_z=2$ as in Section~\ref{sec:tx validate})}
% 	\end{cases} 
% \end{align*}
The actor who initially purchased the gas for $\tx$ will get the refund of $g^{\dagger}\times\tx_p$. And the \coinsign paid for the consumed gas is 
%
\begin{align}
	\left(\tx_g-g^{\dagger}\right)\times \tx_p.
\end{align}
%
If the $\tx$ is not executed (i.e. only when $R_z=2$ as in Section~\ref{sec:tx validate})), no gas will be charged. If the sender $\sender{\tx}$ can not afford gas fee, all its the remained balance of sender $\sender{\tx}$ will be charged as gas fee. The actual charged gas fee will be record in receipt $R_f$. 

	
The charged gas fee is added to the reward pool for miners. Thus in general a higher gas price on a transaction would cost the sender more but also increase the chance of being processed timely.

In computing the accumulated gas used of the whole block, the non-refundable gas $g^{\dagger} - g'$ is not taken into consideration. But in case the sender $\sender{\tx}$ can not afford gas fee, the gas used is considered as $\tx_g$, even if the actual charged gas fee is less than $\tx_p\times\tx_g$. The gas used is also recorded in receipt $R_g$. 
%
Thus the accumulated gas used of a block is intrinsically taking the summation of field $R_g$ over transaction receipts.

\paragraph{Storage Limit and Storage Collateral for core space transactions}

Every core space transactions also have a filed of \textbf{storageLimit} that declare the maximum storage bytes $\tx_\ell$ increasing for the present transaction. Before transaction execution, besides gas fee and transferred value, the sender $\sender{\tx}$ must have enough balance for storage collateral for specified \textbf{storageLimit}, i.e., $\tx_\ell\times\collateralperbyte\;\unit$. Unlike the gas fee, these collateral will not be charged or locked at this time. At the end of transaction execution, if the sender doesn't have enough balance paying for the increased storage collateral or the increased storage bytes of sender exceeds storage limit, the transaction execution fails. More details for collateral for storage is specified in section~\ref{sec:collateral}. 

\paragraph{Sponsorship for core space transactions}

In case a core space transaction $\tx$ is calling a smart contract $\contract$ with \textbf{sponsor for gas} and $\tx$ is qualified for the subsidy as checked in function $\mathsf{Whitelist}(\cdot)$ defined in \cref{eq:whitelist}, 
$\contract$ is responsible for the purchase of gas if it has sufficient \textbf{sponsor balance for gas}
and the transaction \textbf{gasLimit} $\tx_g$ does not exceed the {\bf sponsor limit for gas}, 
and otherwise the sender $\sender{\tx}$ is still responsible for the whole purchase of gas as if there were no sponsor at all. 
%
Formally, we define function $\mathsf{GasElig}(\st,\tx)$ to check whether transaction $\tx$ is eligible for gas consumption sponsorship and define function $\mathsf{GasSpr}(\st,\tx)$ to check where $\tx$ is actually sponsored for gas consumption. 
\begin{align}
	\mathsf{GasElig}(\st,\tx)\eqdef \quad &\tx_s=\mathsf{Core} \;\wedge\; \mathrm{Type}_{\tx_a}=\typecontract \;\wedge\; \mathsf{Whitelist}(\st,\sender{\tx},\contract) \\
	&\;\wedge\; \st[\tx_a]_p[\mathsf{gas}]_a\neq 0 \;\wedge\; \tx_g\times\tx_p \le \st[\tx_a]_p[\mathsf{limit}] \\
	\mathsf{GasSpr}(\st,\tx) \eqdef\quad  &\tx_s=\mathsf{Core} \;\wedge\; \mathsf{GasElig}(\st,\tx) \;\wedge\; \st[\tx_a]_p[{\sf gas}]_b\ge \tx_g\times\tx_p
\end{align}
%
where function $\mathsf{Whitelist}(\cdot)$ is defined in \cref{eq:whitelist}. 

If a contract has \textbf{sponsor for collateral}, the storage collateral can also be sposnored by a contract. Formally,
\begin{align}
	\mathsf{ColElig}(\st,\tx) &\eqdef \quad \tx_s=\mathsf{Core} \;\wedge\; \mathrm{Type}_{\tx_a}=\typecontract \;\wedge\; \mathsf{Whitelist}(\st,\sender{\tx},\contract) \;\wedge\; \st[\tx_a]_p[\mathsf{col}]_a\neq 0 \\
	\mathsf{ColSpr}(\st,\tx) &\eqdef\quad  \tx_s=\mathsf{Core} \;\wedge\; \mathsf{ColElig}(\st,\tx) \;\wedge\; \st[\tx_a]_p[{\sf col}]_b\ge \tx_\ell\times \collateralperbyte
\end{align}

\subsection{Transaction Execution}

\subsubsection{Pre-execution Validation}
\label{sec:tx validate}

Before being executed, a transaction $\tx$ in the processing queue must pass the following secondary test of intrinsic validity. 
\begin{enumerate}[nosep]
	\item The transaction \textbf{nonce} is valid,
   i.e. $\tx_n = \st\left[\sender{\tx}\right]_n$ where $\st$ is the current world-state.

   \item (For core space only) The current epoch is in the range specified by \textbf{epochHeight}, 
	i.e. current epoch height is in $[\tx_e - \txepochbound, \tx_e + \txepochbound]$.	

   \item (For core space only) The recipient address is valid , i.e. the type indicator (first $4$-bit) of $\tx_a$ belongs to $\set{\typereserved,\typenormal,\typecontract}$.
\end{enumerate}

Note that the local legality of the transaction, 
e.g. the $\rlp$ format
% , intrinsic gas limit, 
and the validity of signature, 
is already verified in the first intrinsic validity test before accepting the corresponding block into the \name \tg, as discussed in Section~\ref{sec:block validate},
and will not be checked again at this moment.

If $\tx$ fails at these checks, the transaction will not be executed, the nonce for account will not increase and no transaction fee is charged for such transaction. Let $R'$ be the receipt of last transaction.
Then the receipt of current transaction will be set as follows:
\begin{align}
	R_u=R'_u && R_f=0 && R_g=0 && R_{\bf l}=\emptystring && R_z=2 && R_s=0 && R_{\bf o}=\emptystring && R_{\bf i}=\emptystring
\end{align}
%
(The bloom filter $R_b$ of log $R_{\bf l}$ is computed accordingly. 
)


If $\tx$ passes all the above pre-execution checks, the execution of $\tx$ is as specified in the rest of this section.


\subsubsection{Preprocessing}
\label{subsubsec:preprocessing}

In the preprocessing phase of $\tx$, the balance of $\sender{\tx}$ (and the sponsor, if applicable) is examined so that the payment for any further operation is assured.
The world-state will be transformed from $\st$ into $\st^0\eqdef \st^{**}$ if $\tx$ passes the preprocessing, or directly into $\st'$ and the execution is aborted if $\tx$ fails at any step.

\paragraph{Nonce incremental.}
The beginning of execution 
causes an irrevocable changed to the state $\st$: 
the nonce of the sender, $\sender{\tx}_n$, is incremented by one. 
%
We define the state $\st^*$:
\begin{align}
	\st^*  &\eqdef \st \qquad \mbox{  except:}\\
	\st^*\left[\sender{\tx} \right]_n &\eqdef \st\left[\sender{\tx} \right]_n+1 
\end{align}

\paragraph{Gas consumption payment validation.}

The up-front payment of a transaction $\tx$ first figures out whether the gas consumption is sponsored. $\tx$ is sponsored on gas consumption if $\tx$ is eligible for sponsorship on gas consumption and the calling contract has sufficient \textbf{sponsor balance for gas fee}. 
\begin{itemize}
	\item If the gas consumption of $\tx$ is sponsored, it must be a core space transaction. The world-state $\st^{**}$ after gas consumption payment is as follows: 
	\begin{align}
		\st^{**}  &\eqdef \st^* \qquad \mbox{  except:}\\
		\st^{**}\left[\contract_{addr}\right]_p[{\sf gas}]_b &\eqdef \st^*\left[\sender{\tx}\right]_p[{\sf gas}]_b-\tx_g\times\tx_p
	\end{align} 
	
	\item Otherwise, the sender $\sender{\tx}$ is required to pay for the gas consumption. 
	The balance of $\sender{\tx}$ should satisfy $\st^{*} \left[(\sender{\tx},\tx_s)\right]_b \ge \tx_g\times\tx_p+\tx_v$ and otherwise a \emph{not enough balance exception} is generated. The handling of \emph{not enough balance exception} will be discussed later.
	The world-state after the gas consumption payment is defined as: 
	\begin{align}
		\st^{**}  &\eqdef \st^* \qquad \mbox{  except:}\\
		\st^{**} \left[(\sender{\tx},\tx_s)\right]_b &\eqdef \max\set{\st^*\left[(\sender{\tx},\tx_s)\right]_b-\tx_g\times\tx_p,0}
	\end{align}
\end{itemize}

\paragraph{Storage limit validation.}

After charging, Conflux decides who is responsible for storage collateral. If $\tx$ is eligible for sponsorship on storage collateral and calling contract $\contract=\tx_a$ has enough \textbf{sponsor balance for collateral}, contract $\contract$ is responsible for the storage collateral resulted in the execution of $\tx$ and will be the owner of modified entries. 
%
Otherwise, the sender $\sender{\tx}$ is the owner of modified entries
and has the obligation to pay corresponding storage collateral. 
%
\begin{align}
	\mathsf{ColOwner}(\st,\tx) &\eqdef\left\{ \begin{array}{ll}
		\tx_a & \text{if } \mathsf{ColSpr}(\st,\tx) = \true \\ 
		\sender{\tx} & \text{if } \mathsf{ColSpr}(\st,\tx) = \false \\ 
	\end{array}\right.
\end{align}

If $\sender{\tx}$ is the storage owner but his balance cannot afford the full collateral as declared in {\bf storageLimit} after transferring value $\tx_v$, 
i.e. $\st^{**}[\sender{\tx}]_b<\tx_v+\tx_\ell\times 10^{18}/1024$, 
then the execution of $\tx$ fails due to \emph{not enough balance exception}. 

\paragraph{Handling not enough balance exception.} 

Whenever the preprocessing of $\tx$ generates a \emph{not enough balance exception} during preprocessing, the execution of $\tx$ fails and there will be no further execution of $\tx$. To figure out whether this exception caused by the insufficient sponsorship balance in contract, the sender balance before transaction execution (i.e. $\st[\sender{\tx}]_b$) is compared with a \emph{minimum required balance} defined as
\begin{align}
	\tx_v+(1-\mathbb{I}(\mathsf{GasElig}(\st,\tx)))\times \tx_g \times \tx_p + (1-\mathbb{I}(\mathsf{ColElig}(\st,\tx)))\times \tx_\ell \times \collateralperbyte.
\end{align}

If $\st[\sender{\tx}]_b$ has enough balance for \emph{minimum required balance}, the sender $\sender{\tx}$ is considered not responsible for the generated \emph{not enough balance exception}. 
In this case, the resultant world-state $\st'$ is reverted to $\st$, the nonce of sender is reset so that $\tx$ is reusable. The receipt is composed as follows (where $R'$ refers the receipt of last transaction):
\begin{align}
	&R_u=R'_u && R_f=0 && R_g=0 && R_{\bf l}=\emptystring \\
	&R_z=2 && R_s=0 && R_{\bf o}=\emptystring && R_{\bf i}=\emptystring
\end{align}

In other cases, sender $\sender{\tx}$ is responsible for the exception. The resultant world-state is $\st'$ is reverted to $\st^{**}$ and the receipt is composed as follows if sender is non-existent. (i.e. $\st[\sender{\tx}]\neq\varnothing$).
\begin{align}
	&R_u=R'_u+\tx_g && R_f=\min\{\tx_g\times\tx_p,\st[(\sender{\tx},\tx_s)]_b\} && R_g=\mathsf{GasElig}(\st,\tx) && R_{\bf l}=\emptystring \\
	&R_z=1 && R_s=0 && R_{\bf o}=\emptystring && R_{\bf i}=\emptystring
\end{align}

If sender $\sender{\tx}$ is responsible for the exception and the sender is empty, the resultant world-state is $\st'$ is reverted to $\st$. The receipt is composed as follows
\begin{align}
	&R_u=R'_u && R_f=0 && R_g=0 && R_{\bf l}=\emptystring \\
	&R_z=2 && R_s=0 && R_{\bf o}=\emptystring && R_{\bf i}=\emptystring
\end{align}


\subsubsection{Execution Substate}
\label{subsubsec:substate}

The \emph{transaction substate} $A$ is a three tuple which accrues intermediate information during execution. 
\begin{align}
	A\eqdef \left( A_{\bf s}, A_{\bf l}, A_{\bf c}\right)
\end{align}
The components of $A$ are defined as follows: 
\begin{itemize}[nosep]
	\item $A_{\bf s}$ is the self-destruct set of accounts that will be discarded upon the transaction's completion.

	\item $A_{\bf l}$ is the log series consisting of indexable ``checkpoints'' in the VM code execution, allowing light clients to track the execution of a contract.
	
	\item $A_{\bf c}$ is the set of key-value pairs for the storage collateral changes for each address. Similar with the world state, we write $A_{\bf c}[k]=\varnothing$ for the case that the key $k$ does not exist and regard $A_{\bf c}[k]=\varnothing$ as $A_{\bf c}[k]=0$.

	% \item $A_{\bf t}$ is the set of touched accounts, of which the empty ones will be deleted on the transaction's completion.
\end{itemize}

The empty substate $A^0$, which is also the initial substate, has no self-destructs, no logs, no touched accounts, and zero refund. Formally, $A^0$ is defined as
\begin{align}
	A^0\eqdef \left( \varnothing, \emptystring, \varnothing\right)
\end{align}

For any two substate $A^1$ and $A^2$, the accrued substate $A\eqdef A^1\Cup A^2$ is defined by 
\begin{align}
	A_{\bf s} &\eqdef A^1_{\bf s} \cup A^2_{\bf s} \\ 
	A_{\bf l} &\eqdef A^1_{\bf l} \cdot A^2_{\bf l} \\
	\forall a\in\B_{160},\; A_{\bf c}[a] & \eqdef A^1_{\bf c}[a]+A^2_{\bf c}[a]
	% A_{\bf t} &\eqdef A^1_{\bf t} \cup A^2_{\bf t} 
\end{align}


\subsubsection{Type dependent execution}

If transaction passes the preprocessing, 
then {\name} evaluates the \emph{post-execution provisional state} $\st^P$ from \emph{pre-execution provisional state} $\st^0$ depending on the transaction type as specified in $\tx_a$: either contract creation or message call. 
%
The gas available for the proceeding computation is $g\eqdef \tx_g-g_0$, where $g_0$ is the intrinsic cost of $\tx$ as in (\ref{def:g0}). 

We define the tuple of post-execution provisional state $\st^{P}$, remaining gas $g$, accrued substate $A$ and status code $z$:
\begin{align}\label{def:transform}
	(\st^{P},g',A,z)\eqdef
	\begin{cases}
		\creation(\st^0,\tx_s,\sender{\tx},\sender{\tx},\emptystring, \mathsf{ColOwner}(\st,\tx),g,\tx_p,\tx_v,\tx_{\bf i},0,\zeta,\true) &  \tx_a=\varnothing \\
		\execution(\st^0,\tx_s,\sender{\tx},\sender{\tx},\tx_a,\emptystring,\mathsf{ColOwner}(\st,\tx),\tx_a,g,\tx_p,\tx_v,\tx_v,\tx_{\bf d},0,\true) & \tx_a\neq\varnothing
	\end{cases}
\end{align}
%
Notice that we have three more parameters compared with Ethereum. 

The specifications of function $\creation$ and $\execution$ are given in Section~\ref{sec:creation} and Section~\ref{sec:execution} respectively.

\subsubsection{Postprocessing}\label{sec:tx_post_process}

\paragraph{Storage collateral refund and charge for core space transactions.}

After the message call or contract creation is processed, Conflux checks whether the incremental storage exceeds storage limit specified in $\tx_\ell$ and if the storage owner has enough balance for storage collateral. 
Let $i\eqdef \mathsf{ColOwner}(\st,\tx)$ be the address who owns modified storage entries and $v$ be the available balance to pay for storage collateral, which is defined as 
\begin{align}
	v \eqdef \begin{cases}
		\st^{P}[\sender{\tx}]_b & \text{if }\mathsf{ColSpr}(\st,\tx)=\false \\
		\st^{P}[\tx_a]_p[{\sf col}]_b &  \text{if }\mathsf{ColSpr}(\st,\tx)=\true
	\end{cases}
\end{align}
%
Notice that $A_{\bf c}[i]$ is the incremental storage collateral during execution.
If $A_{\bf c}[i]>\min\{v,\tx_\ell\times 10^{18}/1024\}$, then the execution fails because of not enough balance for collateral or exceeding the storage limit, 
and all the modified state will be reverted to $\st^0$, 
i.e. $\st'\eqdef\st^0$. 
Let $R'$ denote the receipt of last transaction.
Then the receipt of current transaction $\tx$ will be 
\begin{align}
	&R_u=R'_u+\tx_g && R_f = \tx_g\times\tx_p && R_g = \mathsf{GasSpr}(\st,\tx) && R_{\bf l}=\emptystring \\
	&R_z=1 && R_s = \mathsf{ColSpr}(\st,\tx) && R_{\bf o}=\emptystring && R_{\bf i}=\emptystring
\end{align}

Otherwise \name charges and refunds storage collateral and transforms world-state $\st^P$ into $\st^*$. 
We skim the self-destructed contracts here because their storage collateral have been refunded during self-destruction. 
The storage collateral in account state is also updated at this time. 

\begin{align}
	&\st^1  \eqdef \st^{P} \qquad \mbox{  except:}\\
	&\forall a \in \B_{160} \text{ with }  A_{\bf c}[a]\neq 0, \\
	&\quad \begin{cases}
	\st^1[a]_p[{\sf col}]_b \eqdef \st^{P}[a]_p[{\sf col}]_b + f(a) & \mbox{if $a$ refers to a contract account, i.e. $\mathsf{Type}_{a}=\typecontract$} \\
	\st^1[a]_b \eqdef \st^{P}[a]_b + f(a) & \mbox{if $a$ refers to a normal account, i.e. $\mathsf{Type}_{a}= \typenormal$}
	\end{cases}\\
	& \st^1[a]_o \eqdef \st^P[a]_o - f(a)\\ 
	&\st^1[a_{\sf stake}]_{\bf s}[k_3] \eqdef \st^P[a_{\sf stake}]_{\bf s}[k_3] + \sum_{a\in \B_{160}}(f(a) + A_{\bf c}[a]) \\
	&\st^1[a_{\sf stake}]_{\bf s}[k_4] \eqdef \st^P[a_{\sf stake}]_{\bf s}[k_4] + \sum_{a\in \B_{160}}A_{\bf c}[a] \\
	&\mbox{where:}  \\
	&a_{\sf stake} \eqdef \stakingcontract \\ 
	&k_3 \eqdef {\sf [total\char`_issued\char`_tokens]_{\sf ch}}  \\ 
	&k_4 \eqdef {\sf [total\char`_storage\char`_tokens]_{\sf ch}}  \\ 
	& f(a) \eqdef \min\{-A_{\bf c}[a], \st^P_o[a]\}
\end{align}


\paragraph{Gas fee refund.}

The \emph{refundable amount of gas} $g^{\dagger}$ is the minimum of the \emph{legitimately remaining gas} $g'$ (as calculated in (\ref{def:transform})) and a quarter of the \textbf{gasLimit} of $\tx$,
	i.e. $g^{\dagger}\eqdef \min\set{g', \tx_g/4}$.
The refund of gas fee is applied on world-state $\st^{*}$ and results in $\st'\eqdef \Upsilon(\st,\tx)$.

\begin{align}
	& \st^2  \eqdef \st^1 \qquad \mbox{  except:}\\
	& \quad \begin{cases} 
		\st^2\left[\tx_a\right]_p[{\sf gas}]_b \eqdef \st^1\left[\tx_a\right]_p[{\sf gas}]_b+g^{\dagger}\times \tx_p 
		& \mbox{if $\mathsf{GasSpr}(\st,\tx)=\true$}\\
		\st^2 \left[(\sender{\tx},\tx_s)\right]_b \eqdef \st^1\left[(\sender{\tx},tx_s)\right]_b + g^{\dagger}\times \tx_p 
		& \mbox{if $\mathsf{GasSpr}(\st,\tx)=\false$}
	\end{cases} 
\end{align}

\paragraph{Killed contract processing} The sponsor mechanism and collateral mechanism bring additional tasks in contract destruction. First, for all the killed contract, we refund the collateral for code and storage entries to the corresponding owner. 
\begin{align}
	& \st^3  \eqdef \st^2 \qquad \mbox{  except:}\\
	& \forall a \in A_{\bf s},\; \st^2[a]_{\bf s} \eqdef \varnothing \\
	&\forall a \in \B_{160} \text{ with }  A^*_{\bf c}[a]\neq 0, \\
	&\quad \begin{cases}
	\st^3[a]_p[{\sf col}]_b \eqdef \st^2[a]_p[{\sf col}]_b + f(a) & \mbox{if $a$ refers to a contract account, i.e. $\mathsf{Type}_{a}=\typecontract$} \\
	\st^3[a]_b \eqdef \st^2[a]_b + f(a) & \mbox{if $a$ refers to a normal account, i.e. $\mathsf{Type}_{a}= \typenormal$}
	\end{cases}\\
	& \st^3[a]_o \eqdef \st^2[a]_o - f(a)\\ 
	&\st^3[a_{\sf stake}]_{\bf s}[k_3] \eqdef \st^2[a_{\sf stake}]_{\bf s}[k_3] + \sum_{a\in \B_{160}}(f(a) + A^*_{\bf c}[a]) \\
	&\st^3[a_{\sf stake}]_{\bf s}[k_4] \eqdef \st^2[a_{\sf stake}]_{\bf s}[k_4] + \sum_{a\in \B_{160}}A^*_{\bf c}[a] \\
	& A' \eqdef A \Cup A^* \\ 
	&\mbox{where:}  \\
	&a_{\sf stake} \eqdef \stakingcontract \\ 
	&k_3 \eqdef {\sf [total\char`_issued\char`_tokens]_{\sf ch}}  \\ 
	&k_4 \eqdef {\sf [total\char`_storage\char`_tokens]_{\sf ch}}  \\ 
	& f(a) \eqdef \min\{-A^*_{\bf c}[a], \st^2_o[a]\} \\
	& A^* \eqdef A^0 \qquad \mbox{  except:}\\
	& \forall a \in \B^{160}, A^*[a]_{\bf c} = - \collateralperbyte \times \sum_{a'\in A_{\bf s}}\left(\mathbb{I}(\st^2[a']_{w}=a) \times |\st^2[a']_{\bf p}| + \sum_{k\in \B_*} \mathbb{I}(\st^2[a']_{\bf s}[k]_o=a)\times 64\right) 
\end{align}

Then we refund the sponsor balance for all the killed contract. 
%
\begin{align}
	&\st^4  \eqdef \st^3 \qquad \mbox{  except:}\\
	&\forall a \in \B_{160},\;  \st^4[a]_b\eqdef \st^3[a]_b + \sum_{a'\in A_{\bf s}} \left(\mathbb{I}(\st^3[a']_p[{\sf col}]_a=a)\times \st^3[a']_p[{\sf col}]_b + \mathbb{I}(\st^3[a']_p[{\sf gas}]_a=a)\times \st^3[a']_p[{\sf gas}]_b \right)\\
	&\forall a' \in A_{\bf s},\; \st^4[a']_p[{\sf col}]_b\eqdef 0 \\
	&\forall a' \in A_{\bf s},\; \st^4[a']_p[{\sf gas}]_b\eqdef 0
\end{align}

Then we delete all the contracts and burn all the rest balance 
%
\begin{align}
	&\st' \eqdef \st^4 \qquad \mbox{  except:} \\ 
	&\forall (a,p) \in A_{\bf s}, \st'[(a,p)] = \varnothing \\ 
	&\st'[a_{\sf stake}]_{\bf s}[k_3] \eqdef \st^4[a_{\sf stake}]_{\bf s}[k_3] - \sum_{(a,p)\in A_{\bf s}\;\wedge \; p=\mathsf{Core}} (\st[a]_b + \st[a]_t) \\
	&\mbox{where:}  \\
	&a_{\sf stake} \eqdef \stakingcontract \\ 
	&k_3 \eqdef {\sf [total\char`_issued\char`_tokens]_{\sf ch}}  
\end{align}

\paragraph{Transaction Receipt.} 

Now the transaction execution is accomplished.
The returning status code $z$ denotes whether the execution succeeds or not. 
Supposing that $R'$ is the receipt of last transaction, 
the receipt of current transaction will be as follows:
\footnote{Note: before CIP-78, the actual Conflux system has a wrong implementation, see CIP-78 for details.}
\begin{align}
	\begin{array}{llll}
		R_u=R'_u+g' & R_f = (\tx_g-g^{\dagger})\times \tx_p & R_g = \mathsf{GasSpr}(\st,\tx) & R_{\bf l}=A'_{\bf l} \\ 
		R_z=z & \multicolumn{3}{l}{R_s = \left\{\begin{array}{ll}
			\mathsf{ColSpr}(\st,\tx) & \text{if }z=0\\
			0 & \text{if }z=1
		\end{array}\right.} \\
		\multicolumn{4}{l}{R_{\bf o}=\mathsf{ToList}\left(\left\{ (a,A'_{\bf c}[a]) | a\in \B_{160}\;\wedge\; A_{\bf c}[a]>0 \right\}\right)}\\
		\multicolumn{4}{l}{R_{\bf i}=\mathsf{ToList}\left(\left\{(a,-A'_{\bf c}[a]) | a\in \B_{160}\;\wedge\; A_{\bf c}[a]<0  \right\}\right)}
	\end{array}
\end{align}
