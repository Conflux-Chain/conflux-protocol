% !TEX root = ./tech-specification.tex

%%%%%%%%%%%%%%%%%%%%%%%%%%%%%%%%%%%%%%%%

\if 0
\section{Checklist for porting EVM contract to Conflux}

The Ethereum contract is also a valid Conflux contract. So Ethereum contracts can be ported to Conflux easily and have almost same execution results. But notice that Conflux may have different behavior in the following points:
\begin{itemize}
	\item {\bf Gas used and refund:} Conflux requires less gas in SSTORE operation but no longer refunds resetting storage and contract destruction. 
	\item {\bf Gas fee refund:} Conflux will refund at most 1/4 of gas limit. So try to provide an accurate estimation for gas limit before signing transactions. 
	\item {\bf Contract address:} Conflux uses a different way to compute address for normal account from public key and compute contract address in contract creation. (See equation~(\ref{eq:account-address}) and (\ref{eq:new-address}) for details.) The contract developers usually don't need to handle this difference. 
	\item {\bf Contract address conflict:} If the contract address has existed before contract creation, Conflux will abort the contract creation. This is different with the behavior in Ethereum. 
	\item {\bf Collateral for storage:} Conflux requires collateral for storage. Please make sure there is enough balance for storage collateral. 

\end{itemize}
\fi 

\section{List of Hardforks}

\begin{center}
	\begin{tabu}{l c c X}
		\toprule
		\textbf{Hardforks} & \shortstack{\textbf{Activation} \\ \textbf{block number}} & \shortstack{\textbf{Activation} \\ \textbf{epoch number}} & \textbf{Activated CIPs} \\
		\midrule
		Tanzanite & N/A & 3,615,000 & CIP-40 \\
		Hydra & 92,060,600 & 36,935,000 & CIP-43, 62, 64, 71, 76, 78, 86, 90, 92\\
		\bottomrule
	\end{tabu}
\end{center}

Note: the features related to the execution layer are activated by given block numbers. The features related to the consensus layer are activated by given epoch numbers. 
\newpage

\section{Difference between Ethereum and {\name}}
\begin{center}
	\begin{tabu}{X X X X l}%[htbp]
		% \caption{Difference between Ethereum and {\name}}
		% \smallskip
		% \begin{tabularx}{\textwidth}{lXXl}
			\toprule
			 &  
			\textbf{Ethereum} & \textbf{{\name}} (Core Space) &  \textbf{{\name}} (eSpace) & \\
			\midrule
			% \multicolumn{4}{l}{(Virtual Machine and transaction execution)}\\
			% \hline
			Address	type & indistinguishable for all accounts  
			& distinct prefixes for normal (non-contract) account, \emph{Solidity} contracts, and reserved contracts (a.k.a. ``precomipled contracts'')  & Same as Ethereum & \cref{subsec:accounts} \\
			\hline
			Transaction field	& --	& added {\bf chainID}, {\bf storageLimit} and {\bf epochHeight} & Same as Ethereum 	& \cref{sec:tx} 
			\\
			% \hline
			% Block field & {\bf nonce} has $64$ bits & {\bf nonce} has $256$ bits & \cref{sec:block}\\
			% & {\bf number} records number of all ancestor blocks  & {\bf height} records number of ancestor blocks on the pivot chain &\\
			% & {\bf stateRoot} etc. commit to the state after executing the latest block & {\bf deferredStateRoot} etc. commit to the state of deferred execution &\\
			% & --  & added {\bf blame}, {\bf adaptiveWeight} &
			% \\
			\hline
			Gas charge and refund & all unused gas is refundable & at most a quarter of {\bf gasLimit} & Same as Conflux core space & \cref{subsec:gas_and_pay} \\
			\cline{2-5}			
			& full gas fee charged if execution fails on any exception 
			& no gas fee when exception is not caused by the sender  & Same as Ethereum &  \cref{subsubsec:preprocessing} \\
			\hline
			Cost of storage & one-off gas fee &  gas fee + collateral & Same as Ethereum& \cref{subsec:storage consumption}  \\
			\cline{2-5}			
			% Fee schedule ($\hyperlink{SSTORE}{\op{SSTORE}}$)
			& $\hyperlink{SSTORE}{\op{SSTORE}}$ costs $5000$ or $20000$ gas depending on the effect of executing this instruction, may cause a refund of $15000$ gas for clearing a storage value 
			& $\hyperlink{SSTORE}{\op{SSTORE}}$ costs \hyperlink{G__sset}{$G_{sset}$}$=5000$ gas and every \sunitsize storage costs \sunitprice for collateral (locked until the storage is overwritten or released)   
			& Same as Ethereum, but $G_{sset}$, $G_{newaccount}$ and $G_{codedeposit}$ are the twice in Ethereum.
			& \cref{sec:collateral}\\
			\hline
			Transaction validation & 
			{any invalid transaction leads to the whole block being invalid} & 
			{invalid transactions are skipped, while other transactions in the same block can still be valid} & Same as Conflux core space & \cref{sec:txorder}\smallskip\\
			\cline{2-5}
			& {a transaction is invalid if sender's balance cannot afford the up-front payment for transferred value and gas fee (indeed the whole block will be invalid)} & {the transaction is valid if it satisfies all other assertions, but the execution fails immediately because of insufficient balance for the up-front payment (sender's nonce is increased and gas fee is charged)}	& Same as Ethereum & \cref{sec:tx validate} \\\cline{2-5}
			% \hline
			& sender must pay transaction fee from his own balance (sender's balance cannot be zero)	& a sponsor may pay for the cost of calling a smart contract (sender's balance can be zero)	& Same as Ethereum & \cref{sec:sponsor} \\
			\cline{2-5}
			& validity of a transaction cannot depend on current time or height & a transaction is only valid in a specified window of epochs & Same as Ethereum & \cref{sec:tx} \\
			\cline{2-5}
			& no check on recipient's address & recipient address must have a valid type (i.e. normal account, Solidity contract, or reserved contract)  & Same as Ethereum & \cref{subsec:accounts} \\
		\bottomrule
	\end{tabu}
\end{center}
\begin{center}
	\begin{tabu}{X X X X l}%[htbp]
		% \caption{Difference between Ethereum and {\name}}
		% \smallskip
		% \begin{tabularx}{\textwidth}{lXXl}
			\toprule
				&  
			\textbf{Ethereum} & \textbf{{\name}} (Core Space) &  \textbf{{\name}} (eSpace) & \\
			\midrule
			Contract creation & the address of contract created by $\hyperlink{CREATE}{\op{CREATE}}$ does not depend on the initialization code  &  the address of contract created by both $\hyperlink{CREATE}{\op{CREATE}}$ and $\hyperlink{CREATE2}{\op{CREATE2}}$ depends on the initialization code & Same as Ethereum &  \cref{eq:new-address} \\
			\cline{2-5} & 
			$\hyperlink{CREATE}{\op{CREATE}}$ costs $G_{create}=32000$, regardless initilization code length	
			& $\hyperlink{CREATE}{\op{CREATE}}$ costs the same as $\hyperlink{CREATE2}{\op{CREATE2}}$  & Same as Conflux core space & \cref{eq:gas_cost} \\
			\cline{2-5}
			& the maximum size of the byte-code is 24756 bytes & The maximum size of the byte-code is 49152 bytes  & Same as Conflux core space & section~\ref{sec:creation}\\
			\cline{2-5}
			& on address conflict, reset contract but inherit the balance &  
			on address conflict, abort the contract creation   
			& Same as Ethereum
			& \cref{sec:creation}  \smallskip\\
			\hline
			Contract destruction & only by $\hyperlink{SUICIDE}{\op{SUICIDE}}$ & 
			destruction may effect on request of the contract's administrator
			(via the AdminControl contract) & Same as Ethereum & \cref{sec:admin}\\
			% \hline
			% 	&	&	& \\
			% \hline
			% 	&	&	& \\
			\midrule
			Internal Contract & -- & introduces internal contracts & Same as Ethereum 
			& \cref{sec:internal} \smallskip\\
			\hline
			$\hyperlink{BLOCKHASH}{\op{BLOCKHASH}}$ 	&  get the hash of one of the 256 most recent complete blocks	& get the hash of the last block, return zero if querying other blocks	& same as Conflux core & \cref{app:instruction-set}\\
			\hline
			$\hyperlink{CHAINID}{\op{CHAINID}}$ 	& EIP-1344	&  get the {\name} core space chain ID ($1029$)	& get the {\name} eSpace chain ID ($1030$) & \cref{app:instruction-set}\\
			% \hline
			% $\hyperlink{BEGINSUB}{\op{BEGINSUB}}$ 	& OpCode = 0x5c (EIP-2315)	& OpCode = 0xb2 	& \cref{app:instruction-set}\\
			% \hline
			% $\hyperlink{JUMPSUB}{\op{JUMPSUB}}$ 		& OpCode = 0x5d (EIP-2315)	& OpCode = 0xb3	& \cref{app:instruction-set}\\
			% \hline
			% $\hyperlink{RETURNSUB}{\op{RETURNSUB}}$ 	& OpCode = 0x5e (EIP-2315)	& OpCode = 0xb7	& \cref{app:instruction-set}\\
			% \midrule
			% \multicolumn{4}{l}{(Implemented in Ethereum code but not included in Ethereum yellowpaper)}\\
			% \hline
			% $\hyperlink{SELFBALANCE}{\op{SELFBALANCE}}$ 	&  EIP-1884	& push the balance of the currently executing account to the stack.	& \cref{app:instruction-set}\\	
			\bottomrule
		% \end{tabularx}
		% \begin{tablenotes}
		% 	\item[$\ast$] Slightly better security than confirmation with $6$ successive blocks in Bitcoin.

		% 	% \item[$\ast\ast$] It is possible that the fast confirmation rule is never satisfied for a predetermined risk tolerance,{}
		% 	% in which case the block can only be confirmed with the Slow Confirmation Rule.
		% \end{tablenotes}
	\end{tabu}
\end{center}
