%%%%%%%%%%%%%%%%%%%%%%%%%%%%%%%%%%%%%%%%%
% Stylish Article
% LaTeX Template
% Version 2.1 (1/10/15)
%
% This template has been downloaded from:
% http://www.LaTeXTemplates.com
%
% Original author:
% Mathias Legrand (legrand.mathias@gmail.com) 
% With extensive modifications by:
% Vel (vel@latextemplates.com)
%
% License:
% CC BY-NC-SA 3.0 (http://creativecommons.org/licenses/by-nc-sa/3.0/)
%
%%%%%%%%%%%%%%%%%%%%%%%%%%%%%%%%%%%%%%%%%

%----------------------------------------------------------------------------------------
%	PACKAGES AND OTHER DOCUMENT CONFIGURATIONS
%----------------------------------------------------------------------------------------

\documentclass[fleqn,10pt]{SelfArx} % Document font size and equations flushed left


\usepackage[english]{babel} % Specify a different language here - english by default

% \usepackage{lipsum} % Required to insert dummy text. To be removed otherwise

\usepackage{xspace}
\usepackage[flushleft]{threeparttable}
\usepackage[nomessages]{fp}% http://ctan.org/pkg/fp
\usepackage{tabu}
\usepackage{appendix}
\usepackage{titlesec}
\usepackage{apptools}
\usepackage{tikz}
%----------------------------------------------------------------------------------------
%	MACROS
%----------------------------------------------------------------------------------------

% !TEX root = ./tech-specification.tex

%  Notations
\DeclareMathOperator*{\argmax}{arg\,max}
\DeclareMathOperator*{\argmin}{arg\,min}

\renewcommand{\mod}{\mbox{ mod }}
\newcommand{\B}{\ensuremath{\mathbb{B}}}
\newcommand{\Byte}{\ensuremath{\mathbb{BY}}}
\newcommand{\N}{\ensuremath{\mathbb{N}}}
\newcommand{\emptystring}{\ensuremath{\mathbb{\varepsilon}}}


\newcommand{\st}{\ensuremath{\mathbf{\sigma}}}
\newcommand{\mst}{\ensuremath{\mathbf{\mu}}}
\newcommand{\popstack}{\ensuremath{\mathbf{\delta}}}
\newcommand{\pushstack}{\ensuremath{\mathbf{\rho}}}
\newcommand{\poprstack}{\ensuremath{\mathbf{\delta}^*}}
\newcommand{\pushrstack}{\ensuremath{\mathbf{\rho}^*}}

\newcommand{\transition}{\ensuremath{\mathcal{C}}}
\newcommand{\cost}{\ensuremath{\mathsf{C}}}

\newcommand{\creation}{\ensuremath{\mathrm{\Lambda}}}
\newcommand{\execute}{\ensuremath{\mathrm{\Xi}}}
\newcommand{\execution}{\ensuremath{\mathrm{\Theta}}}
\newcommand{\sstore}{\ensuremath{\mathrm{\Phi}}}

\newcommand{\account}{\ensuremath{\mathbf{\alpha}}}
\newcommand{\contract}{\ensuremath{\mathbf{C}}}


\newcommand{\rec}{\ensuremath{\mathbf{R}}}

\newcommand{\name}{Conflux\xspace}
\newcommand{\tg}{\text{Tree-Graph}\xspace}
% \newcommand{\phvname}[1]{{\fontfamily{phv}\selectfont #1}}
\newcommand{\phvname}[1]{\textsf{#1}}
\newcommand{\op}[1]{\ensuremath{\mathsf{#1}}}

\newcommand{\tx}{\textsf{T}}
\newcommand{\txs}{\textsf{Ts}}
\newcommand{\block}{\textsf{B}}
\newcommand{\gblock}{\textsf{G}}
\newcommand{\ablock}{\textsf{A}}
\newcommand{\genesisblock}{\phvname{Genesis}}
\newcommand{\ommers}{\textsf{U}}
\newcommand{\referees}{\textsf{U}}
\newcommand{\head}{\textsf{H}}
\newcommand{\epoch}{\ensuremath{\textsc{Epoch}}}

\newcommand{\graph}{\textbf{G}}


\newcommand{\reward}{\ensuremath{\textsf{R}}}
\newcommand{\award}{\ensuremath{\textsf{R}_{block}}}
\newcommand{\baseaward}{\ensuremath{\textsf{R}_{base}}}

\newcommand{\feeaward}{\ensuremath{\textsf{R}_{fee}}}
\newcommand{\storageaward}{\ensuremath{\textsf{R}_{storage}}}

\newcommand{\anticone}{\ensuremath{\mathcal{A}}}
\newcommand{\af}{\ensuremath{\mathsf{AF}}}
\newcommand{\basef}{\ensuremath{\mathsf{BF}}}
\newcommand{\wf}{\ensuremath{\mathsf{WF}}}


%% System constants
%	account types
\newcommand{\typenormal}{\ensuremath{[0001]_2}}
\newcommand{\typecontract}{\ensuremath{[1000]_2}}
\newcommand{\typereserved}{\ensuremath{[0000]_2}}

% A new name of {\name} VM
\newcommand{\cvm}{{\textsf EVM}\xspace}


% Unit of {\name} tokens
\newcommand{\unit}{{\sf Drip}\xspace}
\newcommand{\gunit}{{\sf GDrip}\xspace}
\newcommand{\coin}{{\sf Conflux}\xspace}
\newcommand{\coinsign}{{\sf CFX}\xspace}
\newcommand{\ucoinsign}{{\sf uCFX}\xspace}
\newcommand{\cfx}{\coinsign}



% Mathematical notations

\newcommand{\set}[1]{{\left \{{#1} \right \}}}
\newcommand{\ceil}[1]{ {\left\lceil{#1} \right\rceil}}
\newcommand{\floor}[1]{ {\left \lfloor{#1} \right \rfloor}}
\newcommand{\zo}{\ensuremath{ \{ 0, 1 \} }}
\newcommand{\eps}{\ensuremath{\varepsilon}}
\newcommand{\union}{\ensuremath{\cup}}
\renewcommand{\vec}[1]{\ensuremath{\mathbf{#1}}}

\newcommand{\eqdef}{\ensuremath{\equiv}}
% \newcommand{\eqdef}{\ensuremath{:=}}

\newcommand{\true}{\mathsf{True}}
\newcommand{\false}{\mathsf{False}}

% Important functions
\newcommand{\kec}{\ensuremath{\mathsf{KEC}}}
\newcommand{\trie}{\ensuremath{\mathsf{TRIE}}}
\newcommand{\hash}{\ensuremath{\mathsf{Hash}}}
\newcommand{\rlp}{\ensuremath{\mathsf{RLP}\xspace}}
\newcommand{\past}{\ensuremath{\mathsf{PAST}}}
\newcommand{\future}{\ensuremath{\mathsf{FUTURE}}}
\newcommand{\chain}{\ensuremath{\mathsf{CHAIN}}}
\newcommand{\sible}{\ensuremath{\mathsf{SIBLING}}}
\newcommand{\weight}{\ensuremath{\mathsf{Weight}}}
\newcommand{\tolist}{\ensuremath{\mathsf{ToList}}}


\newcommand{\timerchain}{\ensuremath{\mathsf{TimerChain}}}
\newcommand{\timerdis}{\ensuremath{\mathsf{TimerDis}}}

\newcommand{\blockno}{\ensuremath{\mathsf{BlockNo}}}


\newcommand{\risk}{\ensuremath{\mathsf{Risk}}}


\newcommand{\parentf}{\ensuremath{\mathsf{P}}}
\newcommand{\pivotf}{\ensuremath{\mathsf{PIVOT}}}
\newcommand{\parent}[1]{\ensuremath{\mathsf{P}{\left({#1} \right)}}}
\newcommand{\pivot}[1]{\ensuremath{\mathsf{PIVOT}{\left({#1} \right)}}}

\newcommand{\sender}[1]{\ensuremath{\mathsf{S}{\left({#1} \right)}}}
\newcommand{\receiver}[1]{\ensuremath{\mathsf{S}{\left({#1} \right)}}}

\newcommand{\senderf}{\ensuremath{\mathsf{S} }}
\newcommand{\epf}{\ensuremath{\mathsf{EPOCH} }}


\newcommand{\cfs}{\ensuremath{\mathsf{CFS}}}
\newcommand{\gused}{\ensuremath{\mathsf{Gas_{used}}}}
\newcommand{\sol}[1]{\texttt{\textbf{#1}}}
\newcommand{\solkw}[1]{{\color{blue}{\texttt{#1}}}}



\newcommand{\pow}{\ensuremath{\mathsf{PoW}}}
\newcommand{\mpethash}{\ensuremath{\mathsf{MpEthash}}}

\newcommand{\quality}{\ensuremath{\mathsf{QUALITY}}}
\newcommand{\offset}{\ensuremath{\mathsf{OFFSET}}}

\newcommand{\hcancel}[1]{%
	% -{#1}
    \tikz[baseline=(tocancel.base)]{
        \node[inner sep=1pt,outer sep=0pt] (tocancel) {{$#1$}};
        \draw[black] (tocancel.south west) -- (tocancel.north east);
    }%
}%
\newcommand{\compressedmix}{\ensuremath{\mathbf{m}_{\mathrm{c}}}}
\newcommand{\mpmix}{\ensuremath{{m}_{\mathrm{p}}}}
\newcommand{\mppnt}{\ensuremath{\vec{p}}}
\newcommand{\fnv}{\ensuremath{E_\text{\tiny FNV}}}

\newcommand{\seedhash}{\ensuremath{\vec{s}_{\mathrm{h}}}}
\newcommand{\dataset}{\ensuremath{\vec{d}}}
\newcommand{\headernon}{\ensuremath{\head_{\hcancel{n}}}}


%%%%%%%%%%%%%%%%%%%%%%%%
% Setting of parameters
%%%%%%%%%%%%%%%%%%%%%%%%

\newcommand{\chainid}{2}
% \newcommand{\chainid}{{\color{red} Conflux Chain ID}}

\newcommand{\txepochbound}{{\color{red}100000}}

\newcommand{\startblockgastlimit}{3\times 10^7}
\newcommand{\collateralperbyte}{\frac{10^{18}}{1024}}
\newcommand{\collateralperbyteline}{10^{18}/{1024}}
\newcommand{\startblockgastlimitline}{30000000}
\newcommand{\maxblocksize}{{$200$} KB}
\newcommand{\maxblocksizeinbytes}{\ensuremath{204800}}
\newcommand{\snapshotperiod}{100000}
\newcommand{\minblockgaslimit}{10^7}


%%%%%%%%%%%%%%%%%%%%
% Internal Contracts
%%%%%%%%%%%%%%%%%%%%%%%%
\newcommand{\admincontract}{{\sf 0x0888000000000000000000000000000000000000}}
\newcommand{\sponsorcontract}{{\sf 0x0888000000000000000000000000000000000001}}
\newcommand{\stakingcontract}{{\sf 0x0888000000000000000000000000000000000002}}

%%%%%%%%%%%%%%%%%%%%
% Mining Schedule
%%%%%%%%%%%%%%%%%%%%%%%%
\newcommand{\blocktime}{0.5}
\FPeval{\blocktimeunix}{clip(\blocktime*1000000)}
\FPeval{\blockinyear}{clip(31536000/\blocktime)}

\newcommand{\quarter}{\mathsf{QRT}}
\newcommand{\initialblockreward}{7}
\newcommand{\eventualblockreward}{1.75}


\newcommand{\decayperiodinday}{91.25}
\FPeval{\decayperiodinblock}{clip(\decayperiodinday*86400/\blocktime)}
% \newcommand{\decayperiod}{15768000} % a quarter = 60*60*24*365/4 seconds, times 2 blocks/s

\newcommand{\decaystartinquarter}{16}
\newcommand{\decayendinquarter}{48}
\FPeval{\decayquarters}{clip(\decayendinquarter-\decaystartinquarter)}
\FPeval{\decaystartinyear}{clip(\decaystartinquarter/4)}
\FPeval{\decayyears}{clip(\decayquarters/4)}

\newcommand{\halfendinyear}{10}
% \FPeval{\decayendinblock}{round(\decayperiodinblock*\halfendinyear*365/\decayperiodinday,0)}

\newcommand{\halfinperiod}{16}
\FPeval{\decayfactor}{round((0.5)^(1/\halfinperiod),3)}
\FPeval{\decayby}{round(1-\decayfactor,3)}
\FPeval{\decaybypercent}{round(\decayby*100,1)}
% \newcommand{\decaypercent}{4.2}
% \FPeval{\decayfactor}{round(1-\decaypercent/100,3)}

\newcommand{\genesistotal}{5000000000}
\newcommand{\mininginflationfirstperiod}{0.0356}
% \newcommand{\mininginflationfirstperiod}{0.05}

% \FPeval{\initialblockreward}{round(\genesistotal*\mininginflationfirstperiod/\decayperiodinblock,1)}
% \newcommand{\blockreward}{900}
% \FPeval{\eventualblockreward}{round(\initialblockreward*(\decayfactor^40),2)}

\newcommand{\targetinflationpercent}{2}



\newcommand{\dfb}{5}
\newcommand{\minerfreeze}{12}
\newcommand{\deferblk}{{\dfb}\xspace}
\FPeval{\dfbresult}{clip(\dfb+1)}
\FPeval{\minerfreezeexec}{clip(\dfb+\minerfreeze)}


% !TEX root = ./contract.tex

%% GHAST parameters

\newcommand{\timerweight}{{\color{red}180}}

\newcommand{\heavyw}{\mathrm{h}}
\newcommand{\heavywvalue}{{\color{red}250}}
% \newcommand{\palpha}{\alpha}
% \newcommand{\pbeta}{\beta}
\newcommand{\palpha}{{\color{red}1000}}
\newcommand{\valuepbeta}{240}
\newcommand{\pbeta}{{\color{red}\valuepbeta}}
\FPeval{\pbetam}{clip(\valuepbeta-1)}


\newcommand{\numberofommers}{100}

% Incentive Parameter

\newcommand{\anticonecountepoch}{{10}}
\newcommand{\anticoneconstant}{{100}}

\newcommand{\difficultyadjustperiod}{5000}

\newcommand{\interest}{{4\%}\xspace}
\newcommand{\annualinterest}{{4.08\%}\xspace}


\newcommand{\commissionrate}{{0.0005}}
\FPeval{\commissionpercent}{clip(round(\commissionrate*100,3))}

\newcommand{\commissiondecayinday}{91.25}
\FPeval{\commissiondecayinblock}{clip(\commissiondecayinday*86400/\blocktime)}

%%  Gas limit and difficulty adjustment

\newcommand{\mingas}{{21000}\xspace}
\newcommand{\mingaslimit}{{5000}\xspace}

% \newcommand{\startdifficulty}{5\times 10^9}
% \newcommand{\startdifficultyline}{5000000000}
\newcommand{\startdifficulty}{2\times 10^{10}}
\newcommand{\startdifficultyline}{20 \mathrm{G}}

\newcommand{\mindifficulty}{\startdifficultyline\xspace}
\newcommand{\diffup}{{1.5}\xspace}
\newcommand{\diffdown}{{0.5}\xspace}


\newcommand{\eragap}{50000}
\FPeval{\eraexample}{clip(2*\eragap+100)}

%% Storage price
\newcommand{\sunitsize}{{64B}\xspace}
\newcommand{\fsunitprice}{\frac{1}{16}}
\newcommand{\sunitprice}{$1/16$ \coinsign}
\newcommand{\storagepertoken}{{$1$KB}\xspace}
\newcommand{\storagebytepertoken}{1024}



% Version controls
% \newcommand{\newversion}[1]{{\color{blue} #1}}
\newcommand{\newversion}[1]{{#1}}
% \newcommand{\oldversion}[1]{{\color{gray} {#1}}}
\newcommand{\oldversion}[1]{{\color{gray} {}}}


% ********* Upgrade *************

% Tanzanite
\newcommand{\tanzaniteepoch}{3615000}
\newcommand{\tanzanitebasereward}{2}


%  Remark and comments

% slience all the notes in the publish version
% \newcommand{\underdiscussion}[1]{}
% \newcommand{\guangsays}[1]{}
% \newcommand{\note}[1]{}
% \newcommand{\todo}[2]{}

\newcommand{\guangsays}[1]{{\color{red}\textbf{}#1}}
\newcommand{\underdiscussion}[1]{{\color{blue}\textbf{}#1}}
\newcommand{\note}[1]{{\color{red} $\langle$NOTE: #1$\rangle$}}
\newcommand{\todo}[2]{{\color{red} $\langle$TODO [#1]: #2$\rangle$}}

%----------------------------------------------------------------------------------------
%	ALGORITHMS
%----------------------------------------------------------------------------------------

\usepackage[figure,vlined,linesnumbered]{algorithm2e}
%----------------------------------------------------------------------------------------
%	STRIKE OUT
%----------------------------------------------------------------------------------------
\usepackage[normalem]{ulem}


%----------------------------------------------------------------------------------------
%	COLUMNS
%----------------------------------------------------------------------------------------

\setlength{\columnsep}{0.55cm} % Distance between the two columns of text
\setlength{\fboxrule}{0.75pt} % Width of the border around the abstract
\usepackage{array,tabularx,multirow}

%----------------------------------------------------------------------------------------
%	COLORS
%----------------------------------------------------------------------------------------

\definecolor{color1}{RGB}{0,0,90} % Color of the article title and sections
\definecolor{color2}{RGB}{150,150,80} % Color of the boxes behind the abstract and headings
\definecolor{color3}{RGB}{120,120,10} % Color of the boxes behind the abstract and headings


\definecolor{backgroundcolor}{RGB}{250,250,233} % Color of the page background

\pagecolor{backgroundcolor}
%----------------------------------------------------------------------------------------
%	HYPERLINKS
%----------------------------------------------------------------------------------------

\usepackage{hyperref} % Required for hyperlinks
\hypersetup{hidelinks,colorlinks,breaklinks=true,urlcolor=color3,citecolor=color1,linkcolor=color1,bookmarksopen=false,pdftitle={Title},pdfauthor={Author}}

\usepackage{cleveref}

\usepackage{tabu} %requires array.

%This should be the last package before \input{Version.tex}
\PassOptionsToPackage{hyphens}{url}\usepackage{hyperref}
% "hyperref loads the url package internally. Use \PassOptionsToPackage{hyphens}{url}\usepackage{hyperref} to pass the option to the url package when it is loaded by hyperref. This avoids any package option clashes." Source: <https://tex.stackexchange.com/questions/3033/forcing-linebreaks-in-url/3034#comment44478_3034>.
% Note also this: "If the \PassOptionsToPackage{hyphens}{url} approach does not work, maybe it's "because you're trying to load the url package with a specific option, but it's being loaded by one of your packages before that with a different set of options. Try loading the url package earlier than the package that requires it. If it's loaded by the document class, try using \RequirePackage[hyphens]{url} before the document class." Source: <https://tex.stackexchange.com/questions/3033/forcing-linebreaks-in-url/3034#comment555944_3034>.
% For more information on using the hyperref package, refer to e.g. https://en.wikibooks.org/w/index.php?title=LaTeX/Hyperlinks&stable=0#Hyperlink_and_Hypertarget.

\makeatletter
 \newcommand{\linkdest}[1]{\Hy@raisedlink{\hypertarget{#1}{}}}
\makeatother
\usepackage{seqsplit}



%----------------------------------------------------------------------------------------
%	ARTICLE INFORMATION
%----------------------------------------------------------------------------------------

\JournalInfo{[Oceanus v0.6.0]} % Journal information
\Archive{\today} % Additional notes (e.g. copyright, DOI, review/research article)

\PaperTitle{{\name} Protocol Specification} % Article title

\Authors{Chenxing Li\textsuperscript{$\dagger$}, Guang Yang\textsuperscript{$\dagger$} } % Authors
\affiliation{\textsuperscript{$\dagger$}\textit{{\name} Dev Team} 
\begin{figure}[ht]\centering
\includegraphics[width=0.5\linewidth]{figs/logo}
\end{figure}
} % Author affiliation
%\affiliation{\textsuperscript{2}\textit{Department of Chemistry, University of Examples, London, United Kingdom}} % Author affiliation
%\affiliation{*\textbf{Corresponding author}: john@smith.com} % Corresponding author

\Keywords{} % Keywords - if you don't want any simply remove all the text between the curly brackets
\newcommand{\keywordname}{Keywords} % Defines the keywords heading name

%----------------------------------------------------------------------------------------
%	ABSTRACT
%----------------------------------------------------------------------------------------

\Abstract{
The success of Bitcoin and its follow-ups have demonstrated the value of decentralized consensus system among anonymous participants not trusting each other.
On top of the consensus network there can be a public ledger or even a general state transition machine, 
such that all participants agree on the state of the ledger or the state machine.
Conceptually the state machine can be Turing-complete and hence essentially a ``world computer'' shared by all participants, whose results cannot be tampered by any single person or entity.
However, the processing power of the shared state machine is currently bottlenecked on the throughput of underlying consensus system.

{\name} implements a Turing-complete state machine on top of a high-throughput consensus network.
% , which is capable of handling thousands of transactions per second.
To achieve a throughput of thousands of transactions per second, {\name} guarantees consensus on the total order of blocks organized in a \tg.
In this way, all forked blocks contribute to the security and throughput of {\name} as well.
In this work we discuss {\name} protocol design and implementation specifications.
}

%----------------------------------------------------------------------------------------

\begin{document}

\flushbottom % Makes all text pages the same height

\maketitle % Print the title and abstract box

\setcounter{tocdepth}{3}

\tableofcontents % Print the contents section

\thispagestyle{empty} % Removes page numbering from the first page

%----------------------------------------------------------------------------------------
%	ARTICLE CONTENTS
%----------------------------------------------------------------------------------------
\newpage
\section{Introduction} % The \section*{} command stops section numbering

%\addcontentsline{toc}{section}{Introduction} % Adds this section to the table of contents

Since the born of Bitcoin, various blockchain projects have demonstrated extraordinary success with the power of consensus among permissionless and trustless parties.
The most successful blockchain project after Bitcoin is widely considered to be Ethereum, which generalizes the blockchain paradigm from a specialized value-transfer system to a more generalized Turing-complete state machine that allows conceptually all kinds of computation.
This generalized state machine, known as \emph{Ethereum Virtual Machine} (EVM), makes the Ethereum network essentially a decentralized computing platform 
where the state advances on input of transactions.
Sometimes Ethereum is referred to as the ``world computer'' that nobody can shut down, 
except that its processing power is rather poor and severely bounded by the throughput of underlying consensus.

The consensus throughput of Bitcoin is (in expectation) one block per $10$ minutes, with block size $1$MB (or $2$MB with Segregated Witness (segwit)).
Bitcoin is set to small block size and low generation rate mainly for security concerns.
Intuitively, when there is no adversary, the natural probability of forks is proportional to the ratio of  block broadcasting time to block (generation) time,
since under the longest chain rule honest mining power may keep working on a fork during the propagation of a newly mined block.
Ethereum applies a tailored version of GHOST rule \cite{GHOST} and smaller block size to achieve a much shorter block time, i.e. roughly $<100$KB per $15$ seconds.
Inclusive Block Chain Protocol \cite{Inclusive} is a ``block-DAG'' proposal which defines a total order of blocks in a directed acyclic graph (DAG) rather than a chain, with the major advantage over GHOST that all forked blocks contribute to the consensus throughput as well. 
Another line of scaling techniques trades security and decentralization for scalability by using sharding, sidechains, or other second layer extensions.
In extreme cases, centralized and somehow permissioned consensus systems are implemented in practice.


{\name} is a project which aims at building a high throughput first layer consensus system without any compromises in security and decentralization; a generalized computation platform that securely processes at least thousands of transactions per second which makes the  throughput of consensus is no more a bottleneck.
The positioning of {\name} is a strong backbone consensus network on which a numerous number of unprecedented applications and extensions can germinate and thrive.
Technically, we follow a similar idea as \cite{Inclusive} but organize blocks in a \tg, 
which enables a fast implementation of the {\name} protocol.


%------------------------------------------------
%%%%%%%%%%%%%%%%%%%%%%%%%%%%%%%%%%%%%%%%
% \section{Conventions}
\input{Convention}


%------------------------------------------------

%%%%%%%%%%%%%%%%%%%%%%%%%%%%%%%%%%%%%%%%
% \section{Basic Components}
\input{Components}


%------------------------------------------------

%%%%%%%%%%%%%%%%%%%%%%%%%%%%%%
%%%%%% \section{Consensus}

\input{Consensus}

\input{Checkpoint}


%------------------------------------------------

%%%%%%%%%%%%%%%%%%%%%%%%%%%%%%
%%%%%%  \section{Transaction Execution}


% !TEX root = ./tech-specification.tex

%%%%%%%%%%%%%%%%%%%%%%%%%%%%%%%%%%%%%%%%

\section{Blockchain Execution}

After determining the total order of blocks, the transactions are executed as if they are packed into sequential blocks on an Ethereum-like chain. 

Blockchain execution is based on a series of ordered blocks $\mathbf{L}$ and a subsequence of pivot blocks $\mathbf{P}$ output by figure~\ref{fig:order}. 
%
The pivot blocks divided into $\mathbf{L}$ into several epochs.  For $k\ge 1$, the epoch $k$ (denoted by $\mathbf{E}_k$) refers the slice in $\mathbf{L}$ started with the next block of $\mathbf{P}[k-1]$ and ended at block $\mathbf{P}[k]$. The epoch 0 refers the genesis block. 


\subsection{Initial state}

The initialization world state $\st^0$ is set as follows. A list $\vec{a}$ with elements $(a,b)$ gives the addresses $a$ and their balance $b$ when the \name blockchain launched.  
\begin{align}
	\forall (a,b)\in \vec{a}, \st^0[a]= \account^0 \quad \mbox{except:} \account_b=b 
\end{align}

In Oceanus, this list contains the following two addresses for faucet. Each address has initialization balance $5\times 10^{33}$ \unit, i.e., 5000 trillion \coinsign.
%
\begin{align*}
	\mathsf{0x1be45681ac6c53d5a40475f7526bac1fe7590fb8} \\
	\mathsf{0x1e768d12395c8abfdedf7b1aeb0dd1d27d5e2a7f}
\end{align*}

The \name internal contracts are also initialized.
%
\begin{align}
	\forall a\in \{a_{\sf stake},a_{\sf sponsor},a_{\sf admin}\}, \st^0[a] &\eqdef \account^0 \quad \mbox{except:} \account_n=1 \\
	\mbox{where:}&\\
	a_{\sf admin} &\eqdef \admincontract \\ 
	a_{\sf sponsor} &\eqdef \sponsorcontract \\
	a_{\sf stake} &\eqdef \stakingcontract
\end{align}

Besides, the global statistic information will be set as follows:

\begin{align}
	\st^0[a_{\sf stake}][k_1]_v & \eqdef \blockinyear\times 2^{80} \\ 
	\st^0[a_{\sf stake}][k_2]_v & \eqdef \blockinyear\times 40000 \\
	\st^0[a_{\sf stake}][k_3]_v & \eqdef 0 \\
	\st^0[a_{\sf stake}][k_4]_v & \eqdef 0 \\
	\st^0[a_{\sf stake}][k_5]_v & \eqdef \sum\nolimits_{(a,b)\in \vec{a}} b \\
	\mbox{where:}&\\
	a_{\sf stake} &\eqdef \stakingcontract \\
	k_1 &\eqdef \sf [accumulate\char`_interest\char`_rate]_{\sf ch} \\ 
	k_2 &\eqdef \sf [interest\char`_rate]_{\sf ch} \\
    k_3 &\eqdef \sf [total\char`_staking\char`_tokens]_{\sf ch} \\
    k_4 &\eqdef \sf [total\char`_storage\char`_tokens]_{\sf ch} \\
    k_5 &\eqdef \sf [total\char`_issued\char`_tokens]_{\sf ch} 
\end{align}


\subsection{Epoch execution}

The blockchain is executed epoch by epoch started with epoch 1. Let $\st_{k-1}$ denote the world state after the execution of epoch $k-1$. The Conflux protocol updates world state from $\st_{k-1}$ to $\st_k$ as follows. Besides updating the world state, the protocol also generates a receipt list $\mathbf{R}_k$ for epoch execution. 

\paragraph{Blocks execution. } First, all the blocks in epoch are executed in sequence by block execution function $\transition_{\sf block}(\st,\block,\mathbf{L}[0..(\tau-1)])=(\st',\mathbf{R}')$, where $\block$ is the block to be executed, $\tau$ is the the index of block $\block$ in $\mathbf{L}$ and $\mathbf{R}'$ is the sequence of transaction receipts. After executing all the blocks, the resultant world-state $\st^*$ becomes the input of the next step and the concatenation of block receipts becomes epoch receipts $\mathbf{R}$. Function $\transition_{\sf block}(\st,\block,\vec{L})$ is defined in section~\ref{sec:block_exec}.

\paragraph{Distribute mining reward. } Since Conflux incentive mechanism puts off the mining reward distribution for $\minerfreeze$ epochs, after execution of epoch $k$, Conflux distribute the mining reward for blocks in $\mathbf{E}_{k-\minerfreeze}$. The computing of mining reward for blocks in epoch $k-\minerfreeze$ requires the following context information.
\begin{itemize}[nosep]
	\item The epoch block set $\mathbf{E}_{k-\minerfreeze}$
	\item The world-state before the execution of all the block $\block$ in $\mathbf{E}_{k-\minerfreeze}$, denoted by $\st(\block)$.
	\item The transaction receipts of all the block $\block$ in $\mathbf{E}_{k-\minerfreeze}$, denoted by $\mathbf{R}'(\block)$.
	\item The tree-graph structure for blocks in $\past(\mathbf{P}[k-\minerfreeze+\anticonecountepoch])$. 
\end{itemize}

Section~\ref{sec:incentive} describes how to compute the block reward $\reward(\block)$ with the context information. The mining reward will be distributed to the block author if the author is . The global parameter \emph{total issued tokens} is updated accordingly. Suppose $\st^*$ is the world state after blocks execution, it will be updated to $\st^{**}$ by
%
\begin{align}
	\st^{**}&\eqdef \st^* \qquad \mbox{except:} \\ 
	\forall a\in \B_{160} \mbox{ with }& \mathsf{Type}_a \in \{[0000]_2,[0001]_2,[1000]_2\} \\ 
	\st^{**}[a]_b&\eqdef \st^*[a]_b +\sum\nolimits_{\block \in \mathbf{E}_{k-\minerfreeze}} \mathbb{I}(\block_{\head_a}=a) \times \reward(\block) \\ 
	\st^{**}[a_{\sf stake}]_{\bf s}[k_3]&\eqdef \st^*[a_{\sf stake}]_{\bf s}[k_3] +\sum\nolimits_{\block \in \mathbf{E}_{k-\minerfreeze}} \left(\mathbb{I}(\mathsf{Type}_{\block_{\head_a}} \in \{[0000]_2,[0001]_2,[1000]_2\} ) \times \reward(\block)-\sum\nolimits_{R\in \mathbf{R}'(\block)}R_f\right) \\ 
	\mbox{where:}&  \\
	a_{\sf stake} &\eqdef \stakingcontract \\ 
	k_3  &\eqdef [{\sf total\char`_issued\char`_tokens}]_{\sf ch}
\end{align}

\paragraph{Recycle storage for killed account. }

Let $\st^{**}$ be the world state after distributing mining reward. \name finally releases the storage for killed account and refunds corresponding storage collaterals. The transaction receipt $R$ contains a field $R_{\bf l}$ record all the killed address during execution. Let $K$ collects all the killed contracts in the whole epoch, the world state $\st'$ after storage recycle is defined by 
%
\begin{align}
	\st'&\eqdef \st^{**} \qquad \mbox{except:}\\
	\forall a'\in K, \st'[a']&\eqdef\varnothing \\
	\st'[a_{\sf stake}]_{\bf s}[k_4] & \eqdef \st^{**}[a_{\sf stake}]_{\bf s}[k_4] - \sum\nolimits_{a\in \B_{160}}f(a) \\
	\forall a \in \B_{160}
	 \mbox{ with }& {\sf Type}_a = [0001]_2, \\
	 \st'[a]_b& \eqdef\st^{**}[a]_b + f(a) \\ 
	 \st'[a]_o& \eqdef \st^{**}[a]_o - f(a)\\ 
	\forall a \in \B_{160}
	 \mbox{ with }&{\sf Type}_a = [1000]_2, \\ 
	 \st'[a]_p[{\sf col}]_b& \eqdef\st^{**}[a]_p[{\sf col}]_b + f(a) \\
	 \st'[a]_o& \eqdef \st^{**}[a]_o - f(a)\\
	\mbox{where:}  \\
	a_{\sf stake} & \eqdef \stakingcontract \\ 
	k_4 & \eqdef {\sf [total\char`_storage\char`_tokens]_{\sf ch}} \\ 
	f(a) &\eqdef \left\{\begin{array}{ll}
		\sum_{a'\in K}\sum_{k\in \B^*} \mathbb{I}[\st^{**}[a']_{\bf s}[k]_o=a]\times 64 \times \collateralperbyte & {\sf Type}_a \in \{[0001]_2, [1000]_2\} \\ 
		0 & \mbox{otherwise}
	\end{array}\right.
\end{align}

\subsection{Block execution}\label{sec:block_exec}
The block execution function $\transition_{\sf block}(\st,\block,\vec{L})$ consists of two steps. 

\paragraph{Update accumulate interest}
\begin{align}
	\st^{*}&\eqdef \st \qquad \mbox{except:} \\ 
	\st^{*}[a_{\sf stake}]_{\bf s}[k_1] & \eqdef \left\lfloor\st[a_{\sf stake}]_{\bf s}[k_1] \times \left(1+\frac{4\%}{\blockinyear}\right)\right\rfloor\\
	\mbox{where:}& \\ 
	a_{\sf stake} & \eqdef \stakingcontract \\ 
	k_1 & \eqdef [{\sf accumulate\char`_interest\char`_rate}]_{\sf ch}
\end{align}

\paragraph{Execute transactions in block}

Each block $\block$ contains a series of transactions $\block_{\sf Ts}$. Start with world state $\st^{*}$, \name executes these transactions in sequence by a transform function $\Upsilon(\st,\tx,\vec{L})=(\st',R)$, which updates world state from $\st$ to $\st'$ by processing transaction $\tx$ and outputs a receipt $R$. The input $\vec{L}$ represents the blocks in front of the present block.After executing all the transaction, the resultant state $\st'$ and the concatenation of all the transaction receipts $\mathbf{R}'$ consists of the output of function $\transition_{\sf block}$.

\section{Transaction Processing}
\label{sec:tx_processing}

\name implements the same virtual machine as Ethereum \cite{ETH_yellow}. 
The execution of a transaction defines the transform function $\Upsilon(\st,\tx,\tau)$, which is similar with Ethereum's state transition function.
In what follows we focus on the \name specific designs in the execution.

\subsection{Gas, Payment and Collateral}
\label{subsec:gas_and_pay}

As defined in Section~\ref{sec:tx} every transaction $\tx$ has two fields of {\bf gasLimit} and {\bf gasPrice} that declare the specific amount of associated gas $\tx_g$ and the price $\tx_p$ of per unit gas.
When starting the execution of a transaction $\tx$, the purchase of gas happens at the price $\tx_g \times \tx_p$ and the transaction $\tx$ is considered invalid if the actor responsible for the cost of gas consumption cannot afford such a purchase.
% , i.e. $\sender{\tx}_b < \tx_g \times \tx_p$.
\newversion{
	Normally $\sender{\tx}$, the sender of $\tx$, is responsible for the cost of gas consumption. 
	In case the transaction $\tx$ is calling a smart contract $\contract$ with sponsorship for gas consumption and $\tx$ is qualified for the subsidy as specified in Section~\ref{sec:sponsor}, 
	$\contract$ is responsible for the purchase of gas if it has sufficient \textbf{sponsor balance for gas},
	and otherwise the sender $\sender{\tx}$ is still responsible for the whole purchase of gas as if there were no sponsor at all.
}
Like in Ethereum, gas does not exist outside the execution of transactions.


	The unused gas can be refunded after the transaction $\tx$ is executed, but no more than a quarter of the total value spent on purchasing. 
	Thus, the \emph{refundable amount of gas} $g^{\dagger}$ is the minimum of the \emph{legitimately remaining gas} $g'$ and a quarter of the \textbf{gasLimit} of $\tx$,
	i.e. $g^{\dagger}\eqdef \min\set{g', \tx_g/4}$, 
	where in principle no gas is refundable (i.e. $g^{\dagger} = g' = 0$) if the execution of $\tx$ fails due to the sender's fault. 
	% The \emph{consumed amount of gas}\footnote{Here for simplicity we write $\gused(\cdot)$ as a function of $\tx$. However it is indeed a function defined on both $\tx$ and the world-state at the beginning of the execution of $\tx$. Thus, multiple occurrences of $\tx$ are considered as distinct inputs for this function and they may incur different amount of consumed gas.} $\gused(\tx)$ is defined as
	% \begin{align*}
	% 	\gused(\tx) \eqdef \begin{cases}
	% 		\tx_g-g^{\dagger} & \mbox{if $\tx$ is executed}\\
	% 		0 & \mbox{if $\tx$ is not executed (i.e. only when $R_z=2$ as in Section~\ref{sec:tx validate})}
	% 	\end{cases} 
	% \end{align*}
	The actor who initially purchased the gas for $\tx$ will get the refund of $g^{\dagger}\times\tx_p$. And the \coinsign paid for the consumed gas is 
	%
	\begin{align}
		\left(\tx_g-g^{\dagger}\right)\times \tx_p.
	\end{align}
	%
	If the $\tx$ is not executed (i.e. only when $R_z=2$ as in Section~\ref{sec:tx validate})), no gas will be charged. If the sender $\sender{\tx}$ can not afford gas fee, all its the remained balance of sender $\sender{\tx}$ will be charged as gas fee. The actual charged gas fee will be record in receipt $R_f$. 

	
The charged gas fee is added to the reward pool for miners. Thus in general a higher gas price on a transaction would cost the sender more but also increase the chance of being processed timely.

In computing the accumulated gas used of the whole block, the non-refundable gas $g^{\dagger} - g'$ is not taken into consideration. But in case the sender $\sender{\tx}$ can not afford gas fee, the gas used is considered as $\tx_g$, even if the actual charged gas fee is less than $\tx_p\times\tx_g$. The gas used is also recorded in receipt $R_g$. 
%
Thus the accumulated gas used of a block is intrinsically taking the summation of field $R_g$ over transaction receipts.

Every transactions also have a filed of \textbf{storage limit} that declare the maximum storage bytes $\tx_\ell$ increasing for the present transaction. Before transaction execution, besides gas fee and transferred value, the sender $\sender{\tx}$ must have enough balance for storage collateral for specified storage limit, i.e., $\tx_\ell\times\collateralperbyte\;\unit$. Unlike the gas fee, these collateral will not be charged or locked at this time. At the end of transaction execution, if the sender doesn't have enough balance paying for the increased storage collateral or the increased storage bytes of sender exceeds storage limit, the transaction execution fails. More details for collateral for storage is specified in section~\ref{sec:collateral}. 

\subsection{Pre-execution Validation}
\label{sec:tx validate}

Before being executed, a transaction $\tx$ in the processing queue must pass the following secondary test of intrinsic validity. 
\begin{enumerate}[nosep]
	\item The current epoch is in the range specified by \textbf{epochHeight}, 
	i.e. current epoch height is in $[\tx_e - \txepochbound, \tx_e + \txepochbound]$.
	
	\item The transaction \textbf{nonce} is valid,
   i.e. $\tx_n = \st\left[\sender{\tx}\right]_n$ where $\st$ is the current world-state.

   \item The recipient address is valid , i.e. the type indicator (first $4$-bit) of $\tx_a$ belongs to $\set{\typereserved,\typenormal,\typecontract}$.
\end{enumerate}

Note that the local legality of the transaction, 
e.g. the $\rlp$ format
% , intrinsic gas limit, 
and the validity of signature, 
is already verified in the first intrinsic validity test before accepting the corresponding block into the \name \tg, as discussed in Section~\ref{sec:block validate},
and will not be checked again at this moment.

If $\tx$ fails at these checks, the transaction will not be executed, the nonce for account will not increase and no transaction fee is charged for such transaction. Let $R'$ be the receipt of last transaction.
Then the receipt of current transaction will be set as follows:
\begin{align}
	R_u=R'_u && R_f=0 && R_g=0 && R_{\bf l}=\emptystring && R_z=2 && R_s=0 && R_{\bf o}=\emptystring && R_{\bf i}=\emptystring
\end{align}
%
(The bloom filter $R_b$ of log $R_{\bf l}$ is computed accordingly. 
)


If $\tx$ passes all the above pre-execution checks, the execution of $\tx$ is as specified in the rest of this section.

\subsection{Transaction Execution}

\subsubsection{Eligibility for sponsorship}

When $\tx$ is calling a contract $\contract$, the gas fee and storage collateral during the transaction execution could be sponsored by the contract $\contract$. If the sender $\sender{\tx}$ is in the {\bf sponsor whitelist} of contract $\contract$, the gas fee of the current transaction does not exceed the {\bf sponsor limit for gas} of contract and contract $\contract$ has a {\bf sponsor for gas}, transaction $\tx$ is eligible for sponsorship on gas consumption. Formally, we define 
\begin{align}
	\mathsf{GasElig}(\st,\tx)\eqdef \quad \mathrm{Type}_{\tx_a}=\typecontract \;\wedge\; \mathsf{Whitelist}(\st,\sender{\tx},\contract) \;\wedge\; \st[\tx_a]_p[\mathsf{gas}]_a\neq 0 \;\wedge\; \tx_g\times\tx_p \le \st[\tx_a]_p[\mathsf{limit}] 
\end{align}
where function $\mathsf{Whitelist}(\cdot)$ is defined in \cref{eq:whitelist}. 

If the sender $\sender{\tx}$ is in the {\bf sponsor whitelist} of contract $\contract$ and contract $\contract$ has a {\bf sponsor for collateral}, transaction $\tx$ is also eligible for sponsorship on storage collateral.
\begin{align}
	\mathsf{ColElig}(\st,\tx)\eqdef \quad \mathrm{Type}_{\tx_a}=\typecontract \;\wedge\; \mathsf{Whitelist}(\st,\sender{\tx},\contract) \;\wedge\; \st[\tx_a]_p[\mathsf{col}]_a\neq 0
\end{align}

\subsubsection{Preprocessing}
\label{subsubsec:preprocessing}

In the preprocessing phase of $\tx$, the balance of $\sender{\tx}$ (and the sponsor, if applicable) is examined so that the payment for any further operation is assured.
The world-state will be transformed from $\st$ into $\st^0\eqdef \st^{**}$ if $\tx$ passes the preprocessing, or directly into $\st'$ and the execution is aborted if $\tx$ fails at any step.

\paragraph{Nonce incremental.}
The beginning of execution 
causes an irrevocable changed to the state $\st$: 
the nonce of the sender, $\sender{\tx}_n$, is incremented by one. 
%
We define the state $\st^*$:
\begin{align}
	\st^*  &\eqdef \st \qquad \mbox{  except:}\\
	\st^*\left[\sender{\tx} \right]_n &\eqdef \st\left[\sender{\tx} \right]_n+1 
\end{align}

\paragraph{Gas consumption payment validation.}

The up-front payment of a transaction $\tx$ first figures out whether the gas consumption is sponsored. $\tx$ is sponsored on gas consumption if $\tx$ is eligible for sponsorship on gas consumption and the calling contract has sufficient \textbf{sponsor balance for gas fee}. 
\begin{align}
	\mathsf{GasSpr}(\st,\tx) \eqdef\quad  \mathsf{GasElig}(\st,\tx) \;\wedge\; \st[\tx_a]_p[{\sf gas}]_b\ge \tx_g\times\tx_p
\end{align}
\begin{itemize}
	\item If the gas consumption of $\tx$ is sponsored, the world-state $\st^{**}$ after gas consumption payment is as follows: 
	\begin{align}
		\st^{**}  &\eqdef \st^* \qquad \mbox{  except:}\\
		\st^{**}\left[\contract_{addr}\right]_p[{\sf gas}]_b &\eqdef \st^*\left[I_a\right]_p[{\sf gas}]_b-\tx_g\times\tx_p
	\end{align} 
	
	\item Otherwise, the sender $\sender{\tx}$ is required to pay for the gas consumption. 
	The balance of $\sender{\tx}$ should satisfy $\st^{*} \left[\sender{\tx}\right]_b \ge \tx_g\times\tx_p+\tx_v$ and otherwise a \emph{not enough balance exception} is generated. The handling of \emph{not enough balance exception} will be discussed later.
	The world-state after the gas consumption payment is defined as: 
	\begin{align}
		\st^{**}  &\eqdef \st^* \qquad \mbox{  except:}\\
		\st^{**} \left[\sender{\tx}\right]_b &\eqdef \max\set{\st^*\left[\sender{\tx}\right]_b-\tx_g\times\tx_p,0}
	\end{align}
\end{itemize}

\paragraph{Storage limit validation.}

After charging, Conflux decides who is responsible for storage collateral. If $\tx$ is eligible for sponsorship on storage collateral and calling contract $\contract=\tx_a$ has enough \textbf{sponsor balance for collateral}, contract $\contract$ is responsible for the storage collateral resulted in the execution of $\tx$ and will be the owner of modified entries. 
%
Otherwise, the sender $\sender{\tx}$ is the owner of modified entries
and has the obligation to pay corresponding storage collateral. 
%
\begin{align}
	\mathsf{ColSpr}(\st,\tx) &\eqdef\quad  \mathsf{ColElig}(\st,\tx) \;\wedge\; \st[\tx_a]_p[{\sf col}]_b\ge \tx_\ell\times 10^{18}/1024 \\
	\mathsf{ColOwner}(\st,\tx) &\eqdef\left\{ \begin{array}{ll}
		\tx_a & \text{if } \mathsf{ColSpr}(\st,\tx) = \true \\ 
		\sender{\tx} & \text{if } \mathsf{ColSpr}(\st,\tx) = \false \\ 
	\end{array}\right.
\end{align}

If $\sender{\tx}$ is the storage owner but his balance cannot afford the full collateral as declared in {\bf storageLimit} after transferring value $\tx_v$, 
i.e. $\st^{**}[\sender{\tx}]_b<\tx_v+\tx_\ell\times 10^{18}/1024$, 
then the execution of $\tx$ fails due to \emph{not enough balance exception}. 

\paragraph{Handling not enough balance exception.} 

Whenever the preprocessing of $\tx$ generates a \emph{not enough balance exception} during preprocessing, the execution of $\tx$ fails and there will be no further execution of $\tx$. To figure out whether this exception caused by the insufficient sponsorship balance in contract, the sender balance before transaction execution (i.e. $\st[\sender{\tx}]_b$) is compared with a \emph{minimum required balance} defined as
\begin{align}
	\tx_v+(1-\mathbb{I}(\mathsf{GasElig}(\st,\tx)))\times \tx_g \times \tx_p + (1-\mathbb{I}(\mathsf{ColElig}(\st,\tx)))\times \tx_\ell \times \collateralperbyte.
\end{align}

If $\st[\sender{\tx}]_b$ has enough balance for \emph{minimum required balance}, the sender $\sender{\tx}$ is considered not responsible for the generated \emph{not enough balance exception}. 
In this case, the resultant world-state $\st'$ is reverted to $\st$, the nonce of sender is reset so that $\tx$ is reusable. The receipt is composed as follows (where $R'$ refers the receipt of last transaction):
\begin{align}
	&R_u=R'_u && R_f=0 && R_g=0 && R_{\bf l}=\emptystring \\
	&R_z=2 && R_s=0 && R_{\bf o}=\emptystring && R_{\bf i}=\emptystring
\end{align}

In other cases, sender $\sender{\tx}$ is responsible for the exception. The resultant world-state is $\st'\eqdef \st^{**}$ and the receipt is composed as follows (where $R'$ refers the receipt of last transaction):
\begin{align}
	&R_u=R'_u+\tx_g && R_f=\min\{\tx_g\times\tx_p,\st[\sender{\tx}]_b\} && R_g=\mathsf{GasElig}(\st,\tx) && R_{\bf l}=\emptystring \\
	&R_z=1 && R_s=0 && R_{\bf o}=\emptystring && R_{\bf i}=\emptystring
\end{align}


\subsubsection{Execution Substate}
\label{subsubsec:substate}

The \emph{transaction substate} $A$ is a five tuple which accrues intermediate information during execution. 
\begin{align}
	A\eqdef \left( A_{\bf s}, A_{\bf l}, A_{\bf t}\right)
\end{align}
The components of $A$ are defined as follows: 
\begin{itemize}[nosep]
	\item $A_{\bf s}$ is the self-destruct set of accounts that will be discarded upon the transaction's completion.

	\item $A_{\bf l}$ is the log series consisting of indexable ``checkpoints'' in the VM code execution, allowing light clients to track the execution of a contract.

	\item $A_{\bf t}$ is the set of touched accounts, of which the empty ones will be deleted on the transaction's completion.
\end{itemize}

The empty substate $A^0$, which is also the initial substate, has no self-destructs, no logs, no touched accounts, and zero refund. Formally, $A^0$ is defined as
\begin{align}
	A^0\eqdef \left( \varnothing, \emptystring, \varnothing\right)
\end{align}

For any two substate $A^1$ and $A^2$, the accrued substate $A\eqdef A^1\Cup A^2$ is defined by 
\begin{align}
	A_{\bf s} &\eqdef A^1_{\bf s} \cup A^2_{\bf s} \\ 
	A_{\bf l} &\eqdef A^1_{\bf l} \cdot A^2_{\bf l} \\
	A_{\bf t} &\eqdef A^1_{\bf t} \cup A^2_{\bf t} 
\end{align}


\subsubsection{Type dependent execution}

If transaction passes the preprocessing, 
then {\name} evaluates the \emph{post-execution provisional state} $\st^P$ from \emph{pre-execution provisional state} $\st^0$ depending on the transaction type as specified in $\tx_a$: either contract creation or message call. 
%
The gas available for the proceeding computation is $g\eqdef \tx_g-g_0$, where $g_0$ is the intrinsic cost of $\tx$ as in (\ref{def:g0}). 

We define the tuple of post-execution provisional state $\st^{P}$, remaining gas $g'$, accrued substate $A$ and status code $z$:
\begin{align}\label{def:transform}
	(\st^{P},g',A,z)\eqdef
	\begin{cases}
		\creation(\st^0,\sender{\tx},\sender{\tx},\emptystring, \mathsf{ColOwner}(\st,\tx),g,\tx_p,\tx_v,\tx_{\bf i},0,\top) &  \tx_a=\varnothing \\
		\execution(\st^0,\sender{\tx},\sender{\tx},\tx_a,\emptystring,\mathsf{ColOwner}(\st,\tx),\tx_a,g,\tx_p,\tx_v,\tx_v,\tx_{\bf d},0,\top) & \tx_a\neq\varnothing
	\end{cases}
\end{align}
%
Notice that we have three more parameters compared with Ethereum. 

The specifications of function $\creation$ and $\execution$ are given in Section~\ref{sec:creation} and Section~\ref{sec:execution} respectively.

\subsubsection{Postprocessing}\label{sec:tx_post_process}

\paragraph{Storage collateral refund and charge.}

After the message call or contract creation is processed, Conflux checks whether the incremental storage exceeds storage limit specified in $\tx_\ell$ and if the storage owner has enough balance for storage collateral. 
Let $i\eqdef \mathsf{ColOwner}(\st,\tx)$ be the address who owns modified storage entries and $v$ be the available balance to pay for storage collateral, which is defined as 
\begin{align}
	v \eqdef \begin{cases}
		\st^{P}[\sender{\tx}]_b & \text{if }\mathsf{ColSpr}(\st,\tx)=\false \\
		\st^{P}[\tx_a]_p[{\sf col}]_b &  \text{if }\mathsf{ColSpr}(\st,\tx)=\true
	\end{cases}
\end{align}
%
Notice that $\st^P[i]_o-\st^0[i]_o$ is the incremental storage collateral during execution.
If $\st^P[i]_o-\st^0[i]_o>\min\{v,\tx_\ell\times 10^{18}/1024\}$, then the execution fails because of exceeding the storage limit, 
and all the modified state will be reverted to $\st^0$, 
i.e. $\st'\eqdef\st^0$. 
Let $R'$ denote the receipt of last transaction.
Then the receipt of current transaction $\tx$ will be 
\begin{align}
	&R_u=R'_u+\tx_g && R_f = \tx_g\times\tx_p && R_g = \mathsf{GasSpr}(\st,\tx) && R_{\bf l}=\emptystring \\
	&R_z=1 && R_s = \mathsf{ColSpr}(\st,\tx) && R_{\bf o}=\emptystring && R_{\bf i}=\emptystring
\end{align}

Otherwise \name charges and refunds storage collateral and transforms world-state $\st^P$ into $\st^*$. 
We skim the self-destructed contracts here because their storage collateral have been refunded during self-destruction. 
The storage collateral in account state is also updated at this time. 

\begin{align}
	&\st^1  \eqdef \st^{P} \qquad \mbox{  except:}\\
	&\forall a \in \B_{160} \text{ with } \st^{P}[a]_o\neq\st^0[a]_o, \\
	&\quad \begin{cases}
	\st^1[a]_p[{\sf col}]_b \eqdef \st^{P}[a]_p[{\sf col}]_b - (\st^{P}[a]_o-\st^0[a]_o) & \mbox{if $a$ refers to a contract account, i.e. $\mathsf{Type}_{a}=\typecontract$} \\
	\st^1[a]_b \eqdef \st^{P}[a]_b - (\st^{P}[a]_o-\st^0[a]_o) & \mbox{if $a$ refers to a normal account, i.e. $\mathsf{Type}_{a}= \typenormal$}
	\end{cases}\\
	&\st^1[a_{\sf stake}]_{\bf s}[k_4] \eqdef \st^P[a_{\sf stake}]_{\bf s}[k_4] + \sum_{a\in \B_{160}}(\st^{P}[a]_o-\st^0[a]_o) \\
	&\mbox{where:}  \\
	&a_{\sf stake} \eqdef \stakingcontract \\ 
	&k_4 \eqdef {\sf [total\char`_storage\char`_tokens]_{\sf ch}} 
\end{align}

\paragraph{Sponsor balance for collateral refund.}

After refunding collateral for storage, the sponsor balance for collateral of all the destructed contract are refunded to their sponsor.

\begin{align}
	&\st^2  \eqdef \st^1 \qquad \mbox{  except:}\\
	&\forall a \in \B_{160},\;  \st^2[a]_b\eqdef \st^1[a]_b + \sum_{a'\in A_{\bf s}} \mathbb{I}(\st^1[a']_p[{\sf col}]_a=a)\times \st^1[a']_p[{\sf col}]_b \\
	&\forall a' \in A_{\bf s},\; \st^2[a']_p[{\sf col}]_b\eqdef 0
\end{align}


\paragraph{Gas fee refund.}

The \emph{refundable amount of gas} $g^{\dagger}$ is the minimum of the \emph{legitimately remaining gas} $g'$ (as calculated in (\ref{def:transform})) and a quarter of the \textbf{gasLimit} of $\tx$,
	i.e. $g^{\dagger}\eqdef \min\set{g', \tx_g/4}$.
The refund of gas fee is applied on world-state $\st^{*}$ and results in $\st'\eqdef \Upsilon(\st,\tx)$.

\begin{align}
	& \st'  \eqdef \st^2 \qquad \mbox{  except:}\\
	& \quad \begin{cases} 
		\st'\left[\tx_a\right]_p[{\sf gas}]_b \eqdef \st^2\left[\tx_a\right]_p[{\sf gas}]_b+g^{\dagger}\times \tx_p 
		& \mbox{if $\mathsf{GasSpr}(\st,\tx)=\true$}\\
		\st' \left[\sender{\tx}\right]_b \eqdef \st^2\left[\sender{\tx}\right]_b + g^{\dagger}\times \tx_p 
		& \mbox{if $\mathsf{GasSpr}(\st,\tx)=\false$}
	\end{cases} 
\end{align}

\paragraph{Transaction Receipt.} 

Now the transaction execution is accomplished.
The returning status code $z$ denotes whether the execution succeeds or not. 
Supposing that $R'$ is the receipt of last transaction, 
the receipt of current transaction will be as follows:
\begin{align}
	\begin{array}{llll}
		R_u=R'_u+g' & R_f = (\tx_g-g^{\dagger})\times \tx_p & R_g = \mathsf{GasSpr}(\st,\tx) & R_{\bf l}=A_{\bf l} \\ 
		R_z=z & \multicolumn{3}{l}{R_s = \left\{\begin{array}{ll}
			\mathsf{ColSpr}(\st,\tx) & \text{if }z=0\\
			0 & \text{if }z=1
		\end{array}\right.} \\
		\multicolumn{4}{l}{R_{\bf o}=\mathsf{ToList}\left(\left\{ (a,\st^P[a]_o - \st^0[a]_o) | a\in \B_{160}\;\wedge\; \st^P[a]_o - \st^0[a]_o>0 \right\}\right)}\\
		\multicolumn{4}{l}{R_{\bf i}=\mathsf{ToList}\left(\left\{(a,\st^0[a]_o - \st^P[a]_o) | a\in \B_{160}\;\wedge\; \st^P[a]_o - \st^0[a]_o<0  \right\}\right)}
	\end{array}
\end{align}


\subsection{Contract Creation}
\label{sec:creation}

A number of intrinsic parameters are used when creating a smart contract account:
\begin{itemize}[nosep]
	\item world-state ${\st}$;
	
	\item sender $s$;

	\item original sender $o$;
	
	\item other recipients in call stack $\vec{t}$;
	
	\item storage owner $i$;
		
	\item available gas $g$;

	% \item storage limit $\ell$;

	\item gas price $p$;

	\item endowment $v$;

	\item initialization code $\vec{i}$ as an arbitrary length byte array;

	\item the present depth of message-call/contraction-creation stack $e$;

	\item the salt for new account's address $\zeta$,\\
	where $\zeta = \varnothing$ if the creation was caused by {\hyperlink{create}{$\op{CREATE}$}}, 
	and $\zeta\in \B_{256}$ if the creation was caused by {\hyperlink{create2}{$\op{CREATE2}$}};

	\item and finally the permission to change the state $w$.
\end{itemize}


We define the contract creation function by $\creation$,
which evaluates from the above parameters and modifies the state $\st$ to a new state $\st'$, together with the leftover gas $g'$, the accrued substate $A$, the result of creation, and the output $\vec{o}$. 
\begin{align}
	\left(\st',g', A, z, \vec{o} \right)\eqdef \creation\left(\st, s, o, \vec{t},i, g, p, v, \vec{i}, e, w \right)
\end{align}


The address $a$ of the account $\account$ newly created by {\hyperlink{create}{$\op{CREATE}$}} is defined as the $4$-bit contract type indicator concatenating the rightmost $156$ bits (i.e. the $100$-th to $255$-th bit) of the Keccak hash of a zero byte, the sender address $s$, the little-endian 32-byte array of its account nonce and the Keccak hash of \cvm code. 
% 
For {\hyperlink{create2}{$\op{CREATE2}$}} the rule is slightly different by substituting account nonce with the salt $\zeta$ and changing the leading byte before taking Keccak (following EIP-1014).
Combining these two cases, 
the resultant address for the new contract account $\account$ is defined as follows:
\begin{align}\label{eq:new-address}
	a= A(s, \st[s]_{n} - 1, \zeta, \vec{i}) \eqdef 
	\left\{\begin{array}{l l l l l}
	 	\typecontract \circ \kec\big([\mathrm{00}]_{16} &~\circ~ s &\circ~ \mathrm{LE}_{32}(\st[s]_n-1) &\circ~ \kec(\vec{i}) \big)[100 \dots 255]
	 	& \text{if}\ \zeta = \varnothing \\
	 	\typecontract \circ \kec\big([\mathrm{ff}]_{16} &~\circ~ s &\circ~  \zeta   &\circ~ \kec(\vec{i}) \big)[100 \dots 255] 
		& \text{otherwise}
	\end{array} \right.
\end{align}
where $\mathrm{LE}_{32}(\cdot)$ denotes the function that expands an integer value in $[0,2^{256}-1]$ to a little-endian 32-byte array. 
%
Note that we use $\st[s]_n-1$ since it is indeed the sender's nonce at the generation of the respective transaction or VM operation. 

If $\st[a]_c\neq \kec(\emptystring)$, a \emph{Contract Address Conflict} exception is triggered. Function $\creation$ returns $(\varnothing,g,A^0,1)$ immediately. 

Otherwise, the account's nonce is initialized to one, the balance as the value passed by the contract creation transaction,
the storage and code as for the empty string.
The sender's balance is reduced by the transferred value (there must be enough balance or the transaction will not be executed).
Thus the mutated state becomes $\st^*$:
\begin{align}
	\st^* & \eqdef \st \qquad{ \text{except:}}\\
	\st^*[a] &\eqdef \account^0 \quad\text{except:}\; \st^*[a]_n=1 \wedge \st^*[a]_b=v+\st[a]_b \wedge \st^*[a]_a=s\\
	% \left(a, \account_{state}\right)\\
	\st^*[s] &\eqdef \begin{cases}
		\varnothing & \mbox{if $\st[s]=\varnothing$ $\land$ $v=0$}\\
		\st[s]\quad\mbox{except}:	\st^*[s]_b=\st[s]_b-v	& \mbox{otherwise}
	\end{cases}
\end{align}
where $\account^0$ is the default account specified in~\cref{eq:default_account}. 

The unmentioned components of an account are initialized by default.

Finally the account $\account$ is initialized by \cvm code $\vec{i}$ according to the execution model.
Code execution may effect several events that are not internal to the execution state:
the account's storage can be altered, further accounts can be created and further messages calls can be made.
As such, the code execution function $\execute$ evaluates to a tuple of resultant state $\st^{**}$, available gas remaining $g^{**}$, the accrued substate $A$ and the body code $\vec{o}$.
\begin{align}
	\left(\st^{**}, g^{**},  A, \vec{o} \right) \eqdef \execute\left(\st^*, g, I\right)
\end{align}
where $I$ consists of the parameters of the execution environment as follows:
\begin{align}
	I_a &\eqdef a\\
	I_o &\eqdef o\\
	I_i &\eqdef i\\
	I_p &\eqdef p\\
	I_\vec{d} &\eqdef \emptystring\\
	I_\vec{t} & \eqdef \vec{t}\\
	I_s &\eqdef s\\
	I_v &\eqdef v\\
	I_\vec{b} &\eqdef \vec{i}\\
	I_{\head} & \eqdef \head \\
	I_\vec{L} & \eqdef \vec{L} \\ 
	I_e &\eqdef e\\
	I_w &\eqdef w
\end{align}

$I_{\vec{d}}$ evaluates to the empty tuple as there is no input data to this call. 
$I_{\head}$ is the block header of the present block.
$I_\vec{L}$ is the list of block headers ordered in front of the current block.

Code execution depletes gas, and gas may not go below zero, thus the actual execution may exit before the code has come to a natural halting state.
In this (and several other) exceptional cases (i.e. $\st^{**}=\varnothing \land \vec{o}=\varnothing$), we say an out-of-gas (OOG) exception has occurred:
the evaluated state is set to the empty set $\varnothing$, 
and the entire contract creation should have no effect on the state, effectively leaving it as it was immediately prior to the attempt of the failed creation.
%
Function $\creation$ returns $(\varnothing,g^{**},A^0,1)$ immediately. 


If the initialization code completes successfully,
a final storage cost is charged for depositing the code.
The storage cost $s$ is proportional to the code size of the created contract and it consists of two parts:
\begin{itemize}
	\item the code-deposit cost $d$ charged as gas consumption:
	\begin{align}
		d \eqdef   |\vec{o}| \times G_{codedeposit}
	\end{align}

	\item a substate $A^{*}$ will be generated to record the storage occupied by code size. the code size collateral will be charged in transaction post processing and will be locked during the lifetime of the created contract. (Conflux will record the owner of code in world-state and refund the collateral when the contract is destroyed):
	\begin{align}
		A^{*} \eqdef   A^0 \quad\mbox{except: }A^*_{\bf c}[i]=|\vec{o}|
	\end{align}
\end{itemize}


If the remaining gas cannot afford the code-deposit cost (i.e. $g^{**}<d$) or the code size exceeds 49152 bytes (i.e. $|\vec{o}|<49152$), then we also declare that an exception occurs and handle it as a failed contract creation attempt. 
%
Function $\creation$ returns $(\varnothing,g^{**},A^0,1)$ immediately. 

If the contract creation fails for any reason, the value of the transaction is not transferred to the aborted contract, and collateral for storing the code is not locked either.
If the contract creation succeeds, we formally specify the resultant state, gas, storage limit, substate, and status code by $\left(\st', g', A', z\right)$ as follows:
\begin{align}
	g' &\eqdef g^{**}-d \\
	\st' &\eqdef \st^{**} \quad\mbox{except:} \\
		\st'[a]_c &\eqdef \kec(\vec{o}) \\ 
		\st'[a]_{code} &\eqdef (\vec{o},i) \\
		\st'[i]_{o}    &\eqdef \st^{**}[i]_{o}  + |\vec{o}|\times \collateralperbyte \\
	\notag \\
	A' &\eqdef A \Cup A^{*} \\ 
	z &\eqdef 0
\end{align}

In the determination of $\st'$, the final body code for the newly created account is specified by the byte sequence $\vec{o}$ derived from the execution of the initialization code $\vec{i}$.
The status code $z$ is an indicator of whether the contract creation succeeds.

Therefore the result of contract creation is either a successfully created new contract with its endowment and collateral for storage, or no new contract and no transfer of value or collateral at all.

\paragraph{Subtleties.} 
Note that while the initialization code is executing, the newly created address exists but with no intrinsic body code. 
Thus any message call received by it during this time causes no code to be executed. 
If the initialization execution ends with a $\op{SUICIDE}$ instruction, the matter is moot since the account will be deleted before the transaction is completed. 
For a normal $\op{STOP}$ code, or if the code returned is otherwise empty, then the world-state may left with a zombie account. Only the administrator of such contract can destroy it by calling the internal contract described in \cref{sec:admin}.



\subsection{Message Call}\label{sec:execution}
The following intrinsic parameters are used when executing a message call:
\begin{itemize}[nosep]
	\item world-state ${\st}$;
	
	\item sender $s$;

	\item original sender $o$;

	\item recipient $r$;
	
	\item other recipients in call stack ${\bf t}$
	
	\item storage owner $i$

	\item the account $c$ whose code is to be executed, usually the same as recipient; 

	\item available gas $g$;

	\item gas price $p$;

	\item value $v$;

	\item input data $\vec{d}$ of the call, as an arbitrary length byte array;

	\item the present depth of message-call/contraction-creation stack $e$;

	\item and finally the permission to change the state $w$.
\end{itemize}

During the execution of message calls, 
the state and transaction substate may change,
and finally an output data array $\vec{o}$ will be generated.
In case of executing transactions (generated by external controllers) the output data $\vec{o}$ is ignored, however message calls (generated by internal execution process) can result further consequences due to the execution of VM-codes, especially when the message call is generated inside the execution of another message call (or transaction).
\begin{align}
  	\left(\st', g', A, z, \vec{o} \right) \eqdef \execution\left(\st,s,o,r,\vec{t},i,c,g, p,v,\tilde{v},\vec{d},e,w \right)
\end{align}  
Note that we differentiate between the value to be transferred, $v$, from the value apparent in the execution context, $\tilde{v}$, for the $\op{DELEGATECALL}$ instruction.

We let $\st^*$ denote the first transitional world-state, which is the same as the original state except for the value transferred from sender $s$ to recipient $r$ (if $s\ne r$):
\begin{align}
	\st^*[r]_b\eqdef \st[r]_b+v \qquad \land  \qquad\st^*[s]_b\eqdef \st[s]_b-v
	\label{state:first transitional}
\end{align}

In particular, if $\st[r]$ was undefined in $\st$, \name will treat it as an empty account with address $r$ which has no code or state and zero balance and nonce.
If furthermore the transferred value $v$ is positive, the account will be created and stored in $\st^*[r]$. 
Thus the previous equation should be taken to mean:
\begin{align}
	\st^* &\eqdef \st \qquad \mbox{except:}\\
	\st^*\left[ s \right] &\eqdef \begin{cases}
		\varnothing & \mbox{if $\st[s]=\varnothing$ $\land$ $v=0$}\\
		\st[s]\quad\mbox{except:}\st^*[s]_b=\st[s]_b-v & \mbox{otherwise}
	\end{cases}\\
% \end{align}
% \begin{align}
	% \mbox{and}\qquad \st'_1 &\eqdef \st \qquad \mbox{except:}\\
	\st^*[r] &\eqdef \begin{cases}
		\account^0 \quad\text{except:}\; \st^*[r]_b=v  & \mbox{if $\st[r]=\varnothing \;\land\; v\ne 0$}\\
		\varnothing & \mbox{if $\st[r]=\varnothing \;\land\; v = 0$}\\
		\st[r]\quad\mbox{except:}\st^*[r]_b=\st[r]_b+v\; & \mbox{otherwise}
	\end{cases}
\end{align}

The recipient's associated code $\vec{b}$, whose Keccak hash is $\st[r]_c$, is executed according to the execution model if the re-entrance protection is not triggered as in Section~\ref{subsubsec:reentrance}.
Note that the pair $\left( \kec(\vec{b}), \vec{b} \right)$ must be stored at some previous point, i.e. at the last update of the code hash $\st[r]_c$ of the recipient's account. 
Thus $\vec{b}$ can be efficiently determined from $\st[r]_c$,
and it is unique following the collision resistance of $\kec$.

Similar as with contract creation, if the execution halts due to an exception, then the state is reverted to the point immediately prior to balance transfer (i.e. $\st$) of the message call but no gas is refunded.
The new state $\st'$ after executing this message call is as follows:
\begin{align}
	\st' &\eqdef 
	\begin{cases}
		\st 	 	& \mbox{if $\st^{**}=\varnothing$}\\
		\st^{**} 	& \mbox{otherwise}
	\end{cases}\\
	g' & \eqdef 
	\begin{cases}
		0 & \mbox{if $\st^{**} =\varnothing$ $\land$ $\vec{o}=\varnothing$}\\
		g^{**} & \mbox{otherwise}	
	\end{cases}\\
	z &\eqdef 
	\begin{cases}
		1	 	& \mbox{if $\st^{**}=\varnothing$}\\
		0	 	& \mbox{otherwise}
	\end{cases}
\end{align}
where the resultant state $\st^{**}$ and available gas remaining $g^{**}$, together with the accrued substate $A$ and the output data $\vec{o}$, 
are determined by the code execution function $\execute$ evaluated on state $\st^*$.
\begin{align}
	\left(\st^{**}, g^{**},  A, \vec{o} \right) \eqdef \execute \left(\st^*, g, I  \right)
\end{align}
where $I$ contains the parameters of the execution environment as follows:
\begin{align}
	I_a &\eqdef r\\
	I_o &\eqdef o\\
	I_i &\eqdef i\\
	I_p &\eqdef p\\
	I_\vec{d} &\eqdef \vec{d}\\
	I_\vec{t} &\eqdef \vec{t}\\
	I_s &\eqdef s\\
	I_v &\eqdef \tilde{v}\\
	I_\vec{b} &\eqdef \vec{b}\\
	I_{\head} & \eqdef \head \\
	I_{\mathbf{L}} & \eqdef \mathbf{L}\\ 
	I_e &\eqdef e\\
	I_w &\eqdef w	% I_\vec{b} \;\; &\text{such that }  \kec\left(I_\vec{b}\right)  = \st[r]_c
\end{align}




For the frequently used functionalities such as the elliptic curve public key recovery, the SHA2-$256$ hash scheme, and so on, we set up eight ``precompiled computation contracts'' with reserved code's address $c\in\set{1,2,\dots,8}$ (with type indicator $\typereserved$). The precompiled computation contracts have no side-effect during execution. They will not generate logs, modify accounts' storage or trigger another message call. 
%
In the present implementation of \name these exceptional contracts are specified as in the latest version of Ethereum \cite{ETH_yellow}.

\name also introduces internal contracts for specific usage. A high-level description for the internal contracts is given in Section~\ref{sec:internal}. When the recipient's address $r$ is one of the internal contracts, Conflux processes $\execute_{\sf internal}(\st^*,g,I)$ and returns $\left(\st^{**}, g^{**},  A, \vec{o} \right)$. A formal definition is given in section~\ref{sec:internal_contract}. 


\subsection{Execution Model}
\label{sec:exe model}

The execution model specifies the system state transition on input of a sequence of bytecode instructions and a small tuple of environmental data. 
The state transition function is formalized as a virtual state machine,
which 
% This virtual machine 
is Turing-complete except that its running time and storage space is intrinsically bounded by the limited amount of available gas and collateral for storage.
% 
For this moment we implement the well-known Ethereum Virtual Machine (EVM), and the execution model follows \cite{ETH_yellow}.


\subsubsection{Basics}

The \cvm is a stack-based architecture with $256$-bit word size.
The stack has a maximum size of $1024$ words.
The memory model is a simple word-addressed byte array.  
The machine also has an independent storage model which is a word-addressable word array (rather than byte array for the memory). 
The memory is volatile and storage is steady and maintained as part of the system state. 
All locations in both memory and storage are initialized as zero.
The program code is stored separately in a virtual ROM that is only interactable via specific instructions.

The execution of the virtual machine may reach exceptions for various reasons, including stack underflows/overflow, invalid instruction, invalid jump destination, out-of-gas and so on.
Like the out-of-gas exception, the machine halts immediately and throws an exception to the execution agent, either the transaction processor or recursively the spawning execution environment, which will catch and deal with it separately. 



\subsubsection{Gas Consumption}

The cost of execution, aka. \emph{gas}, is charged under following circumstances:
\begin{enumerate}[nosep]
	\item the execution of instructions, where each type of instructions is assigned an intrinsic amount of gas;

	\item the generation of subordinate message call or contract creation.

\end{enumerate}


\subsubsection{Storage Consumption}
	\label{subsec:storage consumption}

	\name requires a fixed amount of fund, i.e. \sunitprice, locked as collateral during the whole lifetime of each \sunitsize storage entry in the world-state.
	This fund is locked when the entry is created, and is unlocked and returned to the owner when that entry is cleared or overwritten by someone else eventually, as described in Section~\ref{sec:collateral}.
	The interest generated by the collateral is paid to miners as specified in Section~\ref{subsec:storagefee}. 
	Thus the cost of storing an entry  
	is proportional to the time length of storage usage.

	
	The owner of the collateral of a storage entry, 
	which is called ``the owner of that entry'' for simplicity, 
	essentially records who has written the latest content of that entry.
	Normally the initial owner of an entry should be the sender of the transaction that causes the creation of this entry. 
	However, in case a contract provides the collateral on behalf of the sender, the owner will be that contract instead (see Section~\ref{sec:sponsor} for details).
	When a storage entry is modified in the execution of a transaction,
	the ownership of this entry is changed,
	and the old owner's collateral for that entry is replaced by the new owner's collateral.


	If a storage entry is cleared from the world-state,
	then the corresponding collateral is unlocked and returned to the owner of that entry.
	We remark that there is no refund to the actor who causes the clearance, 
	which is distinct from the gas refunding policy in Ethereum \cite{ETH_yellow}.
	Furthermore, to ensure that unlocked collateral for storage is always returned properly, 
	\name does not allow destructing any smart contract with non-zero collateral for storage.



\subsubsection{Execution Environment}
\label{subsubsec:exe_env}
Besides the global system state $\st$ and the amount of remaining gas $g$, 
the execution agent must provide the following important information used in the execution environment, as contained in the tuple $I$:
\begin{itemize}[nosep]
	\item $I_a$, the address of the account which owns the code that is executing.
	
	\item $I_o$, the address of the original sender who originated this execution.
	
	\item $I_i$, the address of the storage owner.

	\item $I_p$, the gas price designated by the transaction that originated this execution.

	\item $I_{\vec{d}}$, the byte array that is the input data to this execution; in case the execution agent is a transaction $\tx$, this would be the transaction data $\tx_{\vec{d}}$.

	\item $I_s$, the address of the account that invoked the code; in case the execution agent is a transaction $\tx$, this would be the transaction sender's address $\sender{\tx}$.

	 \item $I_v$, the value, in \unit, passed to the recipient's account; in case the execution agent is a transaction $\tx$, this would be the transaction value $\tx_v$.

	 \item \linkdest{I__b}{$I_{\vec{b}}$}, the byte array of the machine code to be executed.

	 \item $I_{\head}$, the block header of the present block.

	 \item $I_e$, the depth of the current message-call or contract-creation in the stack.

	 \item $I_w$, the permission to make modifications to the state.

	 \item $I_\st$, the original world-state right before this execution.
\end{itemize}

The state transition is defined by the execution function $\execute$, which takes as input the current world-state $\st$, the amount of gas $g$, 
and the input $I$ as defined above,
and outputs the resultant state $\st'$, the remaining gas $g'$, the accrued substate $A$ and the resultant output $\vec{o}$.
Formally, we define it as follows:
\begin{align}
	\left( \st', g', A, \vec{o} \right) \eqdef \execute\left(\st,g, I\right)
\end{align}
where we recall that the accrued state $A$ consists of the selfdestructs set $A_\vec{s}$, the log series $A_\vec{l}$, the touched accounts $A_\vec{t}$, a series of addresses recording the owners of storage occupation $A_\vec{o}$ and a series of addresses recording the owners of storage release $A_\vec{e}$
(as described in Section~\ref{subsubsec:substate}):
\begin{align}
	A\eqdef\left(A_\vec{s},A_\vec{l},A_\vec{t},A_\vec{o},A_\vec{e} \right)
\end{align}


\subsubsection{Execution Overview}

The $\execute$ function is defined mostly following the Ethereum yellowpaper \cite{ETH_yellow}, except for a few instructions. 
For self-sufficiency we explain the definition of $\execute$ briefly.

In most practical implementations $\execute$ will be modeled as an iterative progression of the pair $(\st,\mst)$ comprising the world-state and the machine state. 
Formally, it can be recursively defined with a function $X$. This uses an iterator function $O$ (which defines the result of a single cycle of the state machine) together with functions \hyperlink{zhalt}{$Z$}, which determines if the present state is an \hyperlink{zhalt}{exceptional halting} state of the machine, and \hyperlink{hhalt}{$H$}, specifying the output data of the instruction if and only if the present state is a \hyperlink{hhalt}{normal halting} state of the machine.

Recall that the empty sequence, denoted by $\emptystring$, is not equal to the empty set, denoted by $\varnothing$; this is important when interpreting the output of $H$, which evaluates to $\varnothing$ when execution is to continue but a series (potentially empty) when execution should halt.
\begin{eqnarray}
\execute(\st, g, I) & \eqdef & (\st'\!, \mst'_{\mathrm{g}}, A, \mathbf{o}) \\
(\st', \mst'\!, A, ..., \mathbf{o}) & \eqdef & X\big((\st, \mst, A^0\!, I)\big) \\
\mst_{\mathrm{g}} & \eqdef & g \\
\mst_{\mathrm{pc}} & \eqdef & 0 \\
\mst_{\mathbf{m}} & \eqdef & (0, 0, ...) \\
\mst_{\mathrm{i}} & \eqdef & 0	\\
\mst_{\mathbf{s}} & \eqdef & \emptystring \\
\mst_{\mathbf{o}} & \eqdef & \emptystring	\\
\mst_{\mathbf{r}} & \eqdef & \emptystring
\end{eqnarray}
\begin{equation} \label{eq:X-def}
X\big( (\st, \mst, A, I) \big) \eqdef \begin{cases}
\big(\varnothing, \mst, A^0, I, \emptystring\big) & \text{if} \quad Z(\st, \mst, A, I) \\
\big(\varnothing, \mst', A^0, I, \mathbf{o}\big) & \text{if} \quad w =  \op{REVERT} \\
O(\st, \mst, A, I) \cdot \mathbf{o} & \text{if} \quad \mathbf{o} \neq \varnothing \\
X\big(O(\st, \mst, A, I)\big) & \text{otherwise} \\
\end{cases}
\end{equation}

where
\begin{eqnarray}
\mathbf{o} & \eqdef & H(\mst, I) \\
(a, b, c, d) \cdot e & \eqdef & (a, b, c, d, e) \\
\mst' & \eqdef & \mst\ \text{except:} \\
\mst'_{\mathrm{g}} & \eqdef & \mst_{\mathrm{g}} - C(\st, \mst, I)
\end{eqnarray}

Note that, when evaluating $\execute$ instead of $X$, 
the fourth element $I'$ is dropped and the remaining gas $\mst'_{\mathrm{g}}$ is extracted from the resultant machine state $\mst'$.

$X$ is thus cycled (recursively here, but implementations are generally expected to use a simple iterative loop) until either \hyperlink{zhalt}{$Z$} becomes true indicating that the present state is exceptional and that the machine must be halted and any changes discarded or until \hyperlink{hhalt}{$H$} becomes a series (rather than the empty set) indicating that the machine has reached a controlled halt.

\paragraph{Machine State.}
The machine state $\mst$ is defined as the tuple $(g, \mathrm{pc}, \vec{m}, i, \vec{s},\vec{r})$ which are the gas available, the program counter $pc \in \N_{256}$ , the memory contents, the active number of words in memory (counting continuously from position 0), the data stack contents and return stack contents. 
The memory contents $\mst_{\vec{m}}$ are a series of zeros of size $2^{256}$.
The return stack $\mst_{\vec{r}}$ is limited to $1023$ items.

For the ease of reading, the instruction mnemonics, e.g. $\op{ADD}$, should be interpreted as their numeric eqdefalents; the full table of instructions and their specifics is given in Appendix \ref{app:instruction-set}.

For the purposes of defining $Z$, $H$ and $O$, we define $w$ as the current operation to be executed:
\begin{equation}\label{eq:currentoperation}
w \eqdef 
	\begin{cases} 
		I_{\vec{b}}[\mst_{\mathrm{pc}}] & \text{if} \quad \mst_{\mathrm{pc}} < \lVert I_{\vec{b}} \rVert \\
		\hyperlink{stop}{\op{STOP}} & \text{otherwise}
	\end{cases}
\end{equation}

Furthermore, we let $\popstack$ and $\pushstack$ denote the fixed number of stack items removed from and pushed into the data stack $\mst_{\mathbf{s}}$ by executing an instruction.
Both $\popstack$ and $\pushstack$ are assumed subscriptable on the instruction. 
Similarly we define $\poprstack$ and $\pushrstack$ for the return stack $\mst_{\mathbf{r}}$, which is only accessed when entering or returning from subroutines on $\op{JUMPSUB}$ and $\op{RETURNSUB}$ instructions. 
An instruction cost function $\cost$ evaluates to the full cost, in gas, of executing the given instruction.

\paragraph{Exceptional Halting.}\hypertarget{Exceptional_Halting_function_Z}{}\linkdest{zhalt}
%
The exceptional halting function $Z$ is defined as:
\begin{equation}
	Z(\st, \mst, A, I) \eqdef
	\begin{array}[t]{ll}
		& \mst_g < C(\st, \mst, I) \quad \\
		\vee & \popstack_w = \varnothing \quad \\
		\vee & \lVert\mst_\mathbf{s}\rVert < \popstack_w \quad \\
		\vee & \left( w =  \op{JUMP} \quad \wedge \quad \mst_\mathbf{s}[0] \notin D(I_\mathbf{b})  \right) \quad \\
		\vee & \left( w =  \op{JUMPI} \quad \wedge \quad \mst_\mathbf{s}[1] \neq 0 \quad \wedge \quad \mst_\mathbf{s}[0] \notin D(I_\mathbf{b})  \right) \quad \\
		\vee & \left( w = \op{RETURNDATACOPY} \quad \wedge \quad \mst_{\mathbf{s}}[1] + \mst_{\mathbf{s}}[2] > \lVert\mst_{\mathbf{o}}\rVert \right) \quad \\
		\vee & \lVert\mst_\mathbf{s}\rVert - \popstack_w + \pushstack_w > 1024 \quad\\ 
		\vee & \left(\neg I_{\mathrm{w}} \quad \wedge \quad W(w, \mst)\right)\quad \\
		\vee & \left( w = \op{SELFDESTRUCT} \quad \wedge \quad \mathsf{Type}_{\mst_{\mathbf{s}}[0]\mbox{ mod }2^{160}}\notin \set{\typereserved,\typenormal,\typecontract}\right) \\
		\vee & \left( w = \op{SELFDESTRUCT} \quad \wedge \quad \st[I_a]_o> 0\right)
	\end{array}
\end{equation}
where
\begin{equation}
W(w, \mst) \eqdef \\
\begin{array}[t]{ll}
	& w \in \{ \op{CREATE},  \op{CREATE2},  \op{SSTORE}, \op{SUICIDE}\} \quad \\
	\vee & \left(\op{LOG0} \le w \wedge w \le  \op{LOG4} \right)\\
	\vee & \left(w \in \{ \op{CALL},  \op{CALLCODE}\} \wedge \mst_{\mathbf{s}}[2] \neq 0\right)
\end{array}
\end{equation}

This states that the execution is in an exceptional halting state if there is insufficient gas, if the instruction is invalid (and therefore its $\popstack$ subscript is undefined), if there are insufficient stack items, if a $\op{JUMP}$/$\op{JUMPI}$ destination is invalid, 
if the output data size $\lVert\mst_{\mathbf{o}}\rVert$ is insufficient for the copy-output-data operation specified in a $\op{RETURNDATACOPY}$ instruction,
or if the new stack size would be larger than $1024$ or state modification is attempted during a static call. The astute reader will realize that this implies that no instruction can, through its execution, cause an exceptional halt.

\paragraph{Jump Destination Validity.}
We previously used $D$ as the function to determine the set of valid jump destinations given the code that is being run. We define this as any position in the code occupied by a  $\op{JUMPDEST}$ instruction.

All such positions must be on valid instruction boundaries, rather than sitting in the data portion of  $\op{PUSH*}$ operations and must appear within the explicitly defined portion of the code (rather than in the implicitly defined \hyperlink{stop}{$\op{STOP}$} operations that trail it).

Formally:
\begin{equation}
D(\mathbf{c}) \eqdef D_{J}(\mathbf{c}, 0)
\end{equation}

where:
\begin{equation}
D_{J}(\mathbf{c}, i) \eqdef \begin{cases}
\{\} & \text{if} \quad i \geqslant \lVert \mathbf{c} \rVert  \\
\{ i \} \cup D_{J}(\mathbf{c}, N(i, \mathbf{c}[i])) & \text{if} \quad \mathbf{c}[i] =  \op{JUMPDEST} \\
D_{J}(\mathbf{c}, N(i, \mathbf{c}[i])) & \text{otherwise} \\
\end{cases}
\end{equation}

where $N$ is the next valid instruction position in the code, skipping the data of a  $\op{PUSH*}$ instruction, if any:
\begin{equation}\label{eq:next-instruction}
N(i, w) \eqdef \begin{cases}
i + w -  \op{PUSH1} + 2 & \text{if} \quad w \in [ \op{PUSH1},  \op{PUSH32}] \\
i + 1 & \text{otherwise} \end{cases}
\end{equation}

\paragraph{Normal Halting.}\hypertarget{normal_halting_function_H}{}\linkdest{hhalt}

The normal halting function $H$ is defined:
\begin{equation}
H(\mst, I) \eqdef 
	\begin{cases}
	H_{\text{\tiny RETURN}}(\mst) & \text{if} \quad w \in \{ \op{\hyperlink{RETURN}{RETURN}},  \op{REVERT}\} \\
	\emptystring \quad\quad& \text{if} \quad w \in \{  \op{\hyperlink{stop}{STOP}},  \op{\hyperlink{selfdestruct}{SUICIDE}} \} \\
	\varnothing \quad\quad& \text{otherwise}
	\end{cases}
\end{equation}

The data-returning halt operations, \hyperlink{RETURN}{ \op{RETURN}} and  \op{REVERT}, have a special function $H_{\text{\tiny RETURN}}$. Note also the difference between the empty sequence and the empty set as discussed \hyperlink{empty_sequence_vs_empty_set}{here}.

\subsubsection{The Execution Cycle}

Stack items are added or removed from the left-most, lower-indexed portion of the series; all other items remain unchanged:
\begin{eqnarray}
O\big((\st, \mst, A, I)\big) & \eqdef & (\st', \mst', A', I) \\
\Delta & \eqdef & \pushstack_{w} - \popstack_{w} \\
\lVert\mst'_{\mathbf{s}}\rVert & \eqdef & \lVert\mst_{\mathbf{s}}\rVert + \Delta \\
\quad \forall x \in \left[ \pushstack_{w}, \lVert\mst'_{\mathbf{s}}\rVert -1 \right]: \mst'_{\mathbf{s}}[x] & \eqdef & \mst_{\mathbf{s}}\left[x-\Delta \right]
\end{eqnarray}

The gas is reduced by the instruction's gas cost.
\begin{eqnarray}
	\quad \mst'_{g} & \eqdef & \mst_{g} - C(\st, \mst, I) \label{eq:mu_pc}
\end{eqnarray}

For most instructions, the program counter $\mathrm{pc}$ increases by $1$ on each cycle, except for following instructions: $\op{PUSH*}$, $\op{JUMP}$, $\op{JUMPI}$, $\op{JUMPSUB}$, $\op{RETURNSUB}$.
The next valid instruction position for $\op{PUSH*}$ instructions is already specified in $N$ as in \cref{eq:next-instruction}. 
We assume a function $J$, subscripted by one instruction from $\big\{\op{JUMP}$, $\op{JUMPI}$, $\op{JUMPSUB}$, $\op{RETURNSUB}\big\}$, which evaluates to the according value:
\begin{eqnarray}\label{eq:u_pc}
	\quad \mst'_{\mathrm{pc}} & \eqdef & 
	\begin{cases}
		\hyperlink{JUMP}{J_{\op{JUMP}}}(\mst) & \text{if} \quad w =  \op{JUMP} \\
		\hyperlink{JUMPI}{J_{\op{JUMPI}}}(\mst) & \text{if} \quad w =  \op{JUMPI} \\
		\hyperlink{JUMPSUB}{J_{\op{JUMPSUB}}}(\mst) & \text{if} \quad w =  \op{JUMPSUB} \\
		\hyperlink{RETURNSUB}{J_{\op{RETURNSUB}}}(\mst) & \text{if} \quad w =  \op{RETURNSUB} \\
		N(\mst_{\mathrm{pc}}, w) & \text{otherwise}
	\end{cases}
\end{eqnarray}

In general, we assume the memory, self-destruct set and system state do not change:
\begin{eqnarray}
\mst'_{\mathbf{m}} & \eqdef & \mst_{\mathbf{m}} \\
\mst'_{\mathrm{i}} & \eqdef & \mst_{\mathrm{i}} \\
A' & \eqdef & A \\
\st' & \eqdef & \st
\end{eqnarray}

However, instructions do typically alter one or several components of these values. Altered components listed by instruction are noted in Appendix \ref{app:vm}, alongside values for $\pushstack$, $\popstack$, $\pushrstack,\poprstack$ and a formal description of the gas requirements.

\subsubsection{Difference from Ethereum}
The execution function $\execute$ follows nearly the same definition as in Ethereum yellowpaper \cite{ETH_yellow} except for a few instructions. 
When executing $O(\st, \mu, A, I) \eqdef (\st' , \mu' , A' , I')$ 
for the iterator function $O$ which defines the result of a single cycle of the state machine,
{\name} differs from Ethereum on following instructions. 


\paragraph{Sub-call operations.} 
{\name} has two additional parameters comparing with Ethereum: 
the recipient addresses call-state $\vec{t}$ and the storage owner $i$.
In sub-call operations such as $\op{CREATE}$, $\op{CALL}$, $\op{CALLCODE}$, $\op{DELEGATECALL}$, and $\op{STATICCALL}$,  
the recipient addresses $I_\vec{t}\cdot I_a$ and storage owner $I_i$ are passed to $\creation$ and $\execution$ as part of the execution environment $I$. 

\paragraph{Re-entrance Protection.}

When calling a contract, {\name} virtual machine makes sure that re-entrance attack is impossible by preventing re-entrance message call, except the message call matches some requirements which make re-entrance attack impossible. 

To be specific, the {\name} virtual machine maintains a call stack $I_\vec{t}$ and 
prevents the message call
when the callee is already in the call stack but different from the caller before executing the code invoked by each message call.
By requiring that the callee being different from the caller, it is still allowed to call and execute other functions in the caller's contract.
Because in such cases the developer should be able to fully anticipate the execution flow
and we do not consider it necessary to trigger the re-entrance protection.

If the message call is indeed re-entering some other contract in the call stack,
the re-entrance protection is triggered 
unless two requirements are satisfied: a) the available gas in sub-execution of message call is no more than $G_{stipend}$, 
b) the message call has no call data. 
It means that the message call generated from solidity built-in function {\tt send()} and {\tt transfer()} will never trigger
re-entrance protection. This design allows some widely used contract logic currently, 
e.g. withdrawing \cfx from a contract when the transfer has to happen via a re-entrance call. 
%
When the re-entrance protection is triggered, {\name} virtual machine deals with it in the same way as ``call stack overflow'' or ``not enough balance for value transfer''. It stops calling the next contract and refunds all the available gas.



\paragraph{$\op{SSTORE}$ operation.} 
The $\op{SSTORE}$ operation transforms $(\st,A)$ into $(\st',A')$ as follows:

\begin{align}
	\st'   &\eqdef \Phi(\st,I_a, \mst_{\sf s}[0],\mst_{\sf s}[1],I_i) \\ 
	A'     &\eqdef A
\end{align}

where $\Phi$ is defined in section~\ref{sec:storage_maintain}.

In Ethereum, the cost of operation $\op{SSTORE}$ is $G_{sset}=20000$ gas when the storage value is set to non-zero from zero,
and  $G_{sreset}=5000$ gas when the storage value is set to zero. 
Ethereum will also refund $R_{sclear}=15000$ gas when the storage value is set to zero from non-zero. 

In {\name}, since cost of using storage is reflected by collateral for storage, there is no need to charge space consumption in gas. 
Thus {\name} charged $G_{sset}=5000$ gas for all the $\op{SSTORE}$ operation, 
regardless of the storage value,
and there is no gas refund either. 
Furthermore, the {\name} ledger $\st$ tracks the owner of every storage entry with non-zero value. 
The execution substate $A$ records all changes on ownership of storage entries. 

\paragraph{$\op{SUICIDE}$ operation.} When executing the $\op{SUICIDE}$ operation on $(\st,\mu,A,I)$, the destruction process is interrupted and an exception will be generated if the contract has non-zero collateral for storage (i.e. $\st[I_a]_o>0$) or the address receiving refund balance is invalid (i.e. $\mathsf{Type}_{\mst_{\mathbf{s}}[0]\mod2^{160}}\notin\set{\typereserved,\typenormal,\typecontract}$). Otherwise, the $\op{SELFDESTRUCT}$ operation transforms $(\st,A)$ into $(\st',A')$ by function $(\st',A')\eqdef \Psi(\st,A)$, which is defined in section~\ref{sec:contract_destruct}. 

Compared with Ethereum, Conflux updates the storage collateral of code owner for destructed contract and refunds the sponsor balance for gas at this time. Notice that the sponsor balance for collateral will not be refunded until all the storage collateral has been charged from or refunded to account balance. 

\paragraph{Subroutine operation.} Ethereum introduces $\op{BEGINSUB}$, $\op{JUMPSUB}$, $\op{RETURNSUB}$ instructions in EIP-2315, which is not listed in its yellowpaper. 
{\name} implements this instruction with identical behavior as Ethereum.



%------------------------------------------------

%%%%%%%%%%%%%%%%%%%%%%%%%%%%%%
%%%%%%  \section{Internal Contracts}

\newversion{
	% !TEX root=./tech-specification.tex

\section{Collateral for Storage}
\label{sec:collateral}

\emph{Collateral for storage} (CFS for short) mechanism is introduced in \name as a pricing method for the usage of storage,
which is more fair and reasonable 
than the one-off storage fee in Ethereum. 
% 
In principle, this mechanism requires a fund being locked as collateral for any occupation of storage space.
The collateral is locked until the corresponding storage is freed or overwritten by someone else,
and the corresponding interest generated by the locked collateral is assigned directly to miners for the maintenance of storage.
Thus, the cost of storage in \name also depends on the duration of space occupation. 

In \name, every entry of storage is \sunitsize, which is exactly the size of a single key/value pair in the world-state.
The required collateral for storage is proportional to the smallest multiple of \sunitsize that are capable to cover all stored items.
For every storage entry, the account that last writes to the entry is 
called \emph{the owner of that storage entry}.
If a storage entry is written in the execution of a contract $\contract$ with sponsorship for collateral, 
then $\contract$ is regarded as the account writing to that entry and hence becomes the owner accordingly (see Section~\ref{sec:sponsor} for more details).
In the whole lifetime of a storage entry in the world-state, 
the owner of that entry must lock a fixed amount of \cfx as collateral for the occupation of storage space. 
In particular, for each storage entry of size \sunitsize the owner should lock \sunitprice. 
This price is essentially $1$ \cfx for \storagepertoken space,
i.e. every byte of storage requires $10^{18}/\storagebytepertoken$ \unit.

At the time that an account $\account$ becomes the owner of a storage entry (at either creation or modification), $\account$ should lock \sunitprice for that entry at the end of transaction execution.
If $\account$ is a normal address, the locked \sunitprice is deducted from its balance. If $\account$ is a contract address, the locked \sunitprice is deducted from its \textbf{sponsor balance for collateral}.
If $\account$ has enough balance then the required collateral is locked automatically,
otherwise if $\account$ does not have enough balance, the whole transaction execution will fail.

When a storage entry is deleted from the world-state, the corresponding \sunitprice collateral is unlocked and returned to the balance of that entry's owner.
In case the ownership of a storage entry is changed, 
the old owner's \sunitprice collateral is unlocked,
while the new owner must lock \sunitprice as collateral at the same time.

For convenience, we introduce the function $\cfs$ which takes an account address $a$ and a world-state $\st$ as input and returns the total amount of \unit's of locked collateral for storage of account $a$ in world-state $\st$.
In case the world-state $\st$ is clear from context, we write $\cfs(a)$ instead of $\cfs(a;\st)$ for succinctness.

\begin{align}
	\cfs(a) \eqdef \cfs(a;\st) \eqdef \st[a]_o
\end{align}

The world state also maintains the total number of locked tokens for collateral, which is stored in storage entry of staking internal contract. We introduce function ${\sf ACFS}$ to read this value from world state $\st$

\begin{align}
	\mathsf{ACFS}(\st) &\eqdef \st[a_{\sf stake}]_{\bf s}[k_4]_v\\
	\mbox{where:} & \\
	a_{\sf stake} &\eqdef \stakingcontract \\ 
	k_4 &\eqdef {\sf [total\char`_storage\char`_tokens]_{\sf ch}} 
\end{align}

\subsection{Storage writing}\label{sec:storage_maintain}

In order to refund the storage collateral to payer when the storage entry is released, Conflux must track the owner for each storage entry. Here we formally describe the storage writing function $\Phi(\st,a,k,v,o)$, which sets storage entry $k$ of account $a$ to value $v$ and address $o$ is the storage owner. It returns updated world-state $\st'$ and a substate $A$.

\begin{align}
	\st'   &\eqdef \st \qquad \mbox{  except:}\\ 
	\st'[a]_{\bf s}[k] &\eqdef\left\{
		\begin{array}{ll}
			(v,o) & v\neq 0\\
			\varnothing & v= 0\\
		\end{array}
	\right.\\
	A^1 &\eqdef A^0  \qquad \mbox{  except:}\\ 
	A^1[s_o]_{\bf c} &\eqdef - 64 \times \collateralperbyte \quad \mbox{ if } v\neq \st[a]_{\bf s}[k]_v \;\wedge\; s_o\neq s'_o \;\wedge\; s_o\neq\varnothing \\ 
	A^2 &\eqdef A^0  \qquad \mbox{  except:}\\ 
	A^2[s'_o]_{\bf c} &\eqdef 64 \times \collateralperbyte \quad \mbox{ if }v\neq \st[a]_{\bf s}[k]_v \;\wedge\; s_o\neq s'_o \;\wedge\; s'_o\neq\varnothing \\ 
	A &\eqdef A^1\Cup A^2 \\
	\text{where:} & \\
	s &\eqdef \st[a]_{\bf s}[k] \\
	s' &\eqdef \st'[a]_{\bf s}[k] 
\end{align}

There are five special storage entries in staking vote contract $\stakingcontract$, which record the statistic information about Conflux blockchain. Their owners are always the staking vote contract and they are exempted from storage collateral. Their keys are list as follows 
\begin{align}
	& \sf [accumulate\char`_interest\char`_rate]_{\sf ch} \\ 
	& \sf [interest\char`_rate]_{\sf ch} \\
    & \sf [total\char`_staking\char`_tokens]_{\sf ch} \\
    & \sf [total\char`_storage\char`_tokens]_{\sf ch} \\
    & \sf [total\char`_issued\char`_tokens]_{\sf ch} 
\end{align}

These five storage entries can only be accessed by the internal contract. In this document, we don't use function $\Phi$ when dealing with these entries and thus function $\Phi$ does not need to consider this special case. 

	
	% !TEX root=./tech-specification.tex
\section{Internal Contracts}
\label{sec:internal}
\name introduces several built-in internal contracts for better system maintenance and on-chain governance.
%
They provide solidity-like interface for developers. The interface list and their gas consumptions are list in section~\ref{sec:internal_gas}. Section~\ref{sec:internal_contract} formally describes the behavior of internal contracts. In this section, we introduce the high-level design of internal contracts. 


\subsection{Sponsorship for Usage of Contracts}
\label{sec:sponsor}

\name implements a sponsorship mechanism to subsidize the usage of smart contracts. 
Thus, a new account with zero balance is able to call smart contracts as long as the execution is sponsored (usually by the operator of Dapps).
The built-in \emph{SponsorControl Contract} is introduced to record the sponsorship information of smart contracts.

The SponsorControl contract keeps the {\bf SponsorInfo} information for each user-established contract $\contract$. The {\bf SponsorInfo} contains the following fields. 
\begin{itemize}[nosep]
	\item {\bf sponsor for gas}: this is the account that provides the subsidy for gas consumption;

	\item {\bf sponsor for collateral}: this is the account that provides the subsidy for collateral for storage; 

	\item {\bf sponsor balance for gas}: this is the balance of subsidy available for gas consumption;

	\item {\bf sponsor balance for collateral}: this is the balance of subsidy available for collateral for storage;

	\item {\bf sponsor limit for gas fee}: this is the upper bound for the gas fee subsidy paid for every sponsored transaction;
\end{itemize}

The SponsorControl contract also keeps a {\bf whitelist} for each user-established contract $\contract$, which records normal accounts that are eligible for the subsidy. A special all-zero address refers to all normal accounts. If the storage entry of SponsorControl contract with key $\contract_{addr}\cdot a$ is set to one, the address $a$ is in the whitelist of contract $\contract$.

So we can check if an address $a$ is in whitelist of contract $\contract$ by
\begin{align}
	\mathsf{Whitelist}(\st,a,\contract) &\eqdef \st[a_{\sf sponsor}]_{\bf s}[\contract_{addr}\cdot a]_v \neq 0 \;\vee\; \st[a_{\sf sponsor}]_{\bf s}[\contract_{addr}\cdot y]_v \neq 0 \notag \\
	\mbox{where:}&\notag \\
	y&\eqdef [0000000000000000000000000000000000000000]_2 \notag \\ 
	a_{\sf sponsor} & \eqdef \sponsorcontract \label{eq:whitelist}
\end{align}



% \medskip

\paragraph{Sponsor for gas consumption.}
For a contract $\contract$ with non-empty {\bf sponsor for gas} and a transaction $\tx$ calling $\contract$, 
$\tx$ is eligible for the subsidy for gas consumption 
if $\senderf(\tx)$ is in the \textbf{whitelist} of $\contract$
and the gas fee specified by $\tx$ is within the limit, i.e. $\tx_p\times\tx_g \le \text{\textbf{sponsor limit for gas fee} of $\contract$}$.
The gas consumption of $\tx$ is paid from the \textbf{sponsor balance for gas} of $\contract$ (if it is sufficient) rather than from the sender's balance,
and the execution of $\tx$ would fail if the \textbf{sponsor balance for gas} cannot afford the gas consumption.
In case the transaction specifies a gas fee (i.e. $\tx_p\times\tx_g$) greater than $\min\set{\textbf{sponsor limit for gas fee}, \textbf{sponsor balance for gas}}$, there is no subsidy and the sender $\sender{\tx}$ should pay for the gas consumption as usual.

\paragraph{Sponsor for storage collateral.}
For a contract $\contract$ with non-empty {\bf sponsor for collateral} and a transaction $\tx$ calling $\contract$,
$\tx$ is eligible for the subsidy for storage usage if: 
a) its {\bf sender} $\sender{\tx}$ is in the \textbf{whitelist} of $\contract$, and 
b) the {\bf sponsor balance for collateral} of $\contract$ can afford its {\bf storageLimit} $\tx_\ell$ (i.e. $\tx_\ell\times\collateralperbyteline$ \unit).
If $\tx$ is eligible, then the collateral for storage incurred in the execution of $\tx$ 
is deducted from \textbf{sponsor balance for collateral} of $\contract$, 
and the owner of those modified storage entries is set to $\contract$ accordingly.
In case $\sender{\tx}$ is not in the \textbf{whitelist} or the \textbf{sponsor balance for collateral} cannot cover the {\bf storageLimit}, the sender $\sender{\tx}$ has to pay for storage usage from its own balance as usual.


\subsubsection{Sponsorship Update}

Both sponsorship for gas and for collateral can be updated by calling the SponsorControl contract.
The current sponsors can call this contract to transfer funds to increase the sponsor balances directly,
and the current sponsor for gas is also allowed to increase the \textbf{sponsor limit for gas} without transferring new funds.
Other normal accounts can replace the current sponsors by calling this contract and providing more funds for sponsorship.



To replace the \textbf{sponsor for gas} of a contract $\contract$, the new sponsor should transfer to $\contract$ a fund more than the current \textbf{sponsor balance for gas} of $\contract$ and set a new value for \textbf{sponsor limit for gas fee}.
The new value of \textbf{sponsor limit for gas fee} should be no less than the old sponsor's limit  
unless the old \textbf{sponsor balance for gas} cannot afford the old limit.
Moreover, the transferred fund should be $\ge 1000$ times of the new limit, so that it is sufficient to subsidize at least $1000$  transactions calling $\contract$. 
If the above conditions are satisfied, the remaining \textbf{sponsor balance for gas} will be refunded to the old \textbf{sponsor for gas},
and then \textbf{sponsor balance for gas}, \textbf{sponsor for gas} and \textbf{sponsor limit for gas fee} will be updated according to the new sponsor's specification.


The replacement of \textbf{sponsor for collateral} is similar except that there is no analog of the limit for gas fee.
The new sponsor should transfer to $\contract$ a fund more than the fund provided by the current \textbf{sponsor for collateral} of $\contract$.
Then the current \textbf{sponsor for collateral} will be fully refunded, i.e. the sum of \textbf{sponsor balance for collateral} and $\cfs(\contract)$,
and both collateral sponsorship fields are changed as the new sponsor's request accordingly.
Note that the contract $\contract$ is the owner of subsidized storage entries, so that the replacement of sponsorship for collateral will not affect ownership of existing storage entries.

\paragraph{Note:} A contract account is also allowed to be a sponsor.
Therefore it is possible that the sponsoring contract may be destructed before its sponsorship is replaced,
in which case the receiver of sponsorship refund will be an already-destructed contract.
Since the balance of such a contract is not operable (unless there is collision of $\kec$),
it is meaningless to record that number in state and hence the refund will be burnt immediately.


\subsection{Admin Management}
\label{sec:admin}

The \emph{AdminControl Contract} is introduced for better maintenance of other smart contracts, espeically which are generated tentatively without a proper destruction routine:
it records the administrator of every user-established smart contract and handles the destruction on request of corresponding administrators.

The default administrator of a smart contract $\contract$ is the  creator of $\contract$, i.e. the sender $\account$ of the transaction that causes the creation of $\contract$.
The current administrator of a smart contract can transfer its authority to another \emph{normal account} by sending a request to the AdminControl contract.
Contract accounts are not allowed to be the administrator of other contracts,
since this mechanism is mainly for tentative maintenance.
Any long term administration with customized authorization rules should be implemented inside the contract,
i.e. as a specific function that handles destruction requests.

At any time, the administrator $\account$ of an existing contract $\contract$ has the right to request destruction of $\contract$ by calling AdminControl.
However, the request would be rejected if the collateral for storage of contract $\contract$ is not zero, i.e. $\cfs(\contract)>0$, or $\account$ is not the current administrator of $\contract$.
If $\account$ is the current administrator of $\contract$ and $\cfs(\contract)=0$, then the destruction request is accepted and processed as follows:
\begin{enumerate}[nosep]
 	\item the balance of $\contract$ will be refunded to $\account$; 

	\item the \textbf{sponsor balance for gas} of $\contract$ will be refunded to \textbf{sponsor for gas};

	\item the \textbf{sponsor balance for collateral} of $\contract$ will be refunded to \textbf{sponsor for collateral};

	\item the internal state in $\contract$ will be released and the corresponding collateral for storage refunded to owners;

	\item the contract $\contract$ is deleted from world-state.
\end{enumerate} 

The administrator of contract $a$ is stored in account component $a_a$. 

\subsection{Staking Mechanism}
\label{sec:staking}

\name introduces the staking mechanism for two reasons:
first, staking mechanism provides a better way to charge the occupation of storage space (comparing to ``pay once, occupy forever'');
and second, this mechanism also helps in defining the voting power in decentralized governance.

At a high level, \name implements a built-in \emph{Staking Contract} 
% (at address $\stakingcontract$) 
to record the staking information of all accounts.
By sending a transaction to this contract, 
users (both external users and smart contracts) can deposit/withdraw funds, which is also called \emph{stakes} in the contract.
The interest of staked funds is issued at withdrawal, 
and depends on both the amount and staking period of the fund being withdrawn. 



In \name, the staking contract keeps track of staked funds and freezing rules. For every account $\account$ the staking contract records the following:
\begin{itemize}
	\item {\bf staking funds}: each staking fund entry consists of the balance $v\in\N_{256}$ and creation time $t\in\N_{64}$ of a staked fund from the sender $\account$, and the entry is cleared when the fund is completely withdrawn;
	

	\item {\bf freezing rules}: each freezing rule entry is a combination of $(v,t)\in \N_{256}\times\N_{64}$ which promises that the total stake balance of account $\account$ must be at least $v$ (measured in \unit) as long as the block number (as defined in \hyperlink{blockno}{$\blockno$}) does not exceed $t$. 
	Expired freezing rule entries are cleared at the next update of freezing rules of the same account $\account$.	
\end{itemize}

Both kinds of entries in the staking contract requires collateral for storage, and the collateral is returned at the clearance of corresponding entries.



\subsubsection{Interest Rate}

The staking interest rate is currently set to \interest per year.
% Compound interest is not implemented yet, however users may achieve it manually. 
Compound interest is implemented in the granularity of blocks.
% On withdrawing funds staked for less than \guangsays{three months}, a commission fee applies depending on the length of actual staking time.
So the annualized interest rate is about \annualinterest.

When executing a transaction sent by account $\alpha$ at block $\block$ to withdraw a fund of value $v$ deposited at block $\block'$,
the interest is calculated as follows:
\begin{align}
	\text{Interest issued to $\account$} 
	&\eqdef \left\lfloor v \times \frac{f^{(\blockno(\block))}(n)}{f^{(\blockno(\block'))}(n)} \right\rfloor- v \\ 
	\mbox{where:}& \\
	f(x) &\eqdef \left\lfloor x \times \left(1+\frac{\interest}{\blockinyear}\right)\right\rfloor \\
	n &\eqdef \blockinyear \times 2^{80}
\end{align}
The interest is approximately equals to 
\begin{align}
	\left(\left(1+\frac{\interest}{\blockinyear}\right)^T-1\right)\times v, 
\end{align}
where $T\eqdef\blockno(\block) - \blockno(\block')$ is the staking period measured by number of blocks, 
and $\blockinyear$ is the expected number of blocks generated in $365$ days with the target block time $\blocktime$ seconds.
% 
% The commission fee starts from $\commissionpercent\%$ and linearly decreases to $0$ after $\commissiondecayinblock$ blocks (i.e. $\commissiondecayinday$ days in expectation).
Therefore after the withdrawal $\account$'s total amount of staking funds is decreased by $v$,
and its balance is increased by:
\begin{align}
	\Delta(\account_b) \eqdef v + \text{Interest issued to $\account$} 
\end{align}

The account $\account$ only specifies the value $v$ in its withdrawal request. 
The withdrawal always starts from the earliest staked fund and recursively continues to the next one until the accumulative amount is sufficient.

The same interest rate applies to collateral for storage as well, but the CFS interest is issued directly to the miners as the payment for storage usage at the generation of every new block.
More details about CFS interest issuance are specified in Section~\ref{subsec:storagefee}.


\subsubsection{Staking for Voting Power}

For decentralized governance, the staking mechanism provides a way to measure the involvement and devotion of users with the new dimension of staking age.

When deciding voting power, it is fair and reasonable to take the staking time into consideration,
rather than merely the amount of tokens.
By relating voting power to committed staking period, the risk of being attacked is also mitigated,
since the attacker must hold the tokens for a sufficiently long period to obtain enough voting power, which increases the cost of launching an attack.

For every account $\account$, its committed staking time is recorded in the build-in staking contract in the form of 
{\bf freezing rules},
where each entry $(v,t)\in \N_{256}\times\N_{64}$ is a promise that the  staking balance of $\account$ must be at least $v$ \unit until the index of block in total order (e.g. $\blockno$) reaches $t$.
A withdrawal request from $\account$ is invalid if any freezing rule is violated after fulfilling that request.
The list of freezing rules for every account is append-only so that committed staking period cannot be canceled or shorten.
However every single rule will eventually expire as the block number grows, e.g. the rule $(v,t)$ expires at block $\block$ with $\blockno(\block)\ge t$. 
Expired freezing rule entries are cleared from state storage of the built-in staking contract at the next update of freezing rules of the same account.	

Note that freezing rules are decoupled from specific staking funds, i.e. old funds can be withdrawn as long as the remaining staking balance is sufficient. 
Therefore, the staking contract is allowed to maintain staked funds in a first-in-first-out manner (i.e. the earliest staked fund is also first withdrawn).
% to minimize commission fee.


The voting power of each staked token is defined in the following table:

\par
\begin{center}
\begin{tabular}{ll}
\toprule
Remaining Committed Staking Time & Voting Power \\
\midrule
One year or more (i.e. $\ge \blockinyear$ blocks) & $1$  \\
Six months to one year ($\ge 31536000$ but $<63072000$ blocks) & $0.5$ \\
Three to six months ($\ge 15768000$ but $<31536000$ blocks) & $0.25$\\
Less than three month (i.e. $< 15768000$ blocks) & $0$ \\
\bottomrule
\end{tabular}
\end{center}
\par
Therefore the total voting power of each account can be easily calculated from its freezing rules as recorded in the staking contract.

}
%------------------------------------------------

% !TEX root=./tech-specification.tex

\section{Proof of Work}
\label{sec:pow}



% \subsection{Proof of Work Function}

\oldversion{In the Pontus version, {\name} tentatively applies a twisted tripple Keccack as the proof of work function $\pow$.
It will be updated in the next stage of {\name} mainnet launch.

More specifically, the function $\pow$ is defined on block headers as follows:
\begin{align}
	\pow\left( \head \right) \eqdef 
	\kec\big(\kec\left(\kec\left(\rlp( \head_{-n} )\right)\circ \head_n\right) 
	\oplus \kec\left(\rlp( \head_{-n})\right)
	% \rlp(\head_{-n})[0\dots 255]
	\big)
\end{align}
}

{\name} applies the \emph{Multi-point Ethash} function $\mpethash$
as the Proof-of-Work function $\pow$.
$\mpethash$ is a twisted version of \textsf{Ethash} function with additional evaulation of a polynomial on multiple points.
The detailed specification of this function is in Appendix~\ref{app:mp_eval_hash}.

The $\mpethash$ function is defined as:
\begin{equation}\label{eq:mpethash}
	\mpethash(\head)
	\eqdef \mpethash\left(\kec\left(\rlp( \head_{-n} )\right), \head_{n},\dataset\right) 
	\eqdef 
	 \kec\left(\seedhash \circ \compressedmix \right)
\end{equation}
where $\dataset$ denotes the dataset derived from $\head$ as in Appendix~\ref{app:dataset} and $\head_{-n}$ denotes the header excluding the {\bf nonce} field,
i.e. $\head \equiv \head_{-n} \circ \head_n$ since {\bf nonce} is indeed the last field in the structure of block header,
and the fields of $\head_{-n}$ are \rlp-serialized according to their order in Section~\ref{sec:block}. 

The output of $\mpethash$ is the Keccak-256 hash of the concatenation of the seed hash $\seedhash\in \B_{512}$ and the compressed mix $\compressedmix\in \B_{256}$.
See Appendix~\ref{appsec:pow} for more details.

The $\mpethash$ function is essentially the Proof-of-Work function $\pow$:
\begin{align}\label{eq:pow}
	% \mpethash(\head)
	% &\eqdef \mpethash\left(\kec\left(\rlp( \head_{-n} )\right), \head_{n},\dataset\right) 
	% \eqdef 
	% \set{\compressedmix, \kec\left(\seedhash \circ \compressedmix \circ \mpmix \right)}\\
	\pow(\head) 
	&\eqdef \mpethash\left(\head\right)
	% \head_x 
	% &\eqdef \mpethash\left(\kec\left(\rlp( \head_{-n} )\right), \head_{n},\dataset\right)[0] 
	% = \compressedmix \label{eq:mixhash}
\end{align}




% \subsection{The PoW Puzzle [new name?]}

% \lipsum[10-12] % Dummy text

\subsection{Proof-of-Work Quality}
\label{subsec:quality}

The \emph{proof-of-work quality} (a.k.a. \emph{PoW quality} or simply \emph{quality}) of a block refers to the expected amount of work spent in finding such a block.
Given a block $\block$ with header $\head(\block)$ and 
the $256$-bit scalar $\offset(\head) \eqdef \left[\head_n(\block)[1\dots127]\right]_2\times 2^{128} \in \N_{256}$ which denotes the offset of proof-of-work validation,
the quality of $\block$ essentially represents the expected number of random trials to find a block $\block'$ with header $\head'$ satisfying that $\pow(\head')$ is in between of $\offset(\head)$ and $\pow(\head)$.

More specifically, the block $\block$ with header $\head(\block)$ has quality
\begin{align}
	\quality(\block)\eqdef 
	\quality(\head) \eqdef
	\begin{cases}
		\left\lfloor2^{256}/\left(\pow(\head) - \offset(\head) +1 \right) \right\rfloor 
		& \mbox{if $\pow(\head) > \offset(\head)$}\\
		\left\lfloor2^{256}/\left(2^{256}+\pow(\head) - \offset(\head) +1 \right) \right\rfloor 
		& \mbox{if $\pow(\head) < \offset(\head)$}\\
		2^{256}-1 &  \mbox{if $\pow(\head) = \offset(\head)$}
	\end{cases}
\end{align}


\subsection{Difficulty Adjustment}
\label{sec:difficulty}

The difficulty is adjusted according to the block generation rate in the past every $\tau$ epochs, where $\tau=\difficultyadjustperiod$ before CIP-86 and $\tau=250$ after CIP-86.
More specifically,
we estimate the current computing power of all miners from the number of blocks in the last $\tau$ epochs and the average timestamps of blocks in the beginning and ending epochs,
and then set the target difficulty for the next $\tau$ epochs such that the expected block generation rate should be roughly one block per $\blocktime$ seconds.

Formally,
for $0\le j\le \tau$,
the target difficulty of a block at height $j$ is set to $\mathbf{d}_0\eqdef {\startdifficulty=\startdifficultyline}$;
for any positive integer $i\ge 1$,
the target difficulty of blocks at height $j\in [\tau i+1,\tau (i+1)]$ is set to $\mathbf{d}_i \in\N_{256}$ such that

\newversion{
	\begin{align}\label{eq:difficulty_adjust}
		\mathbf{d}_i \eqdef \begin{cases}
			\min(\max(\mathbf{d}', 20G, \lceil \mathbf{d}_{i-1} \times \diffdown \rceil),\lfloor  \mathbf{d}_{i-1} \times \diffup \rfloor) & \text{Before CIP-86} \\
			\min(\max(0.2\mathbf{d}'+0.8\mathbf{d}_{i-1}, 20G),\lfloor  \mathbf{d}_{i-1} \times \diffup \rfloor) & \text{After CIP-86}
		\end{cases}
	\end{align}
	where $\mathbf{d}'_i\in\N_{256}$ is the estimation of ideal target difficulty defined as follows
	\begin{align}
		% \mathbf{d}'_i\eqdef \mathbf{d}_{i-1} \times \blocktime \times {\sum_{j=1}^{\tau}|\epoch_{\tau i+j}|}/
		% {\left(\frac{\sum_{\block'\in\epoch_{\tau i}} \block'_{\head_s}}{|\epoch_{\tau i}|} - \frac{\sum_{\block'\in\epoch_{\tau(i-1)}} \block'_{\head_s}}{|\epoch_{\tau(i-1)}|} \right)}
		\mathbf{d}'_i\eqdef \mathbf{d}_{i-1} \times \blocktimeunix \times 
		\left({\sum_{j=\tau (i-1)+1}^{\tau i}|\epoch_{j}|} -1\right)/
		{\left( \head\left(\block^{(\tau i)}\right)_{s} - \min_{\tau (i-1)\le j< \tau i \;\land\; \head\left(\block^{\left(j\right)}\right)_{s}\ne 0}\set{ \head\left(\block^{\left(j\right)}\right)_{s} } \right)}
	\end{align}
	where for every $k\in\N$, $\epoch_k$ denotes the set of \emph{fully valid} blocks in the $k$-th epoch and $\block^{(k)}$ denotes the pivot block in $\epoch_k$.
	In the above formula the total number of blocks in last $\tau$ epochs is decreased by $1$, which leads to an unbiased estimation of block generation rate.
}

Note that a single block $\block$ may not have a global view.
Indeed, the best it could do is to compute the target difficulty $\mathbf{d}_i$ from its local view of blocks in $\past(\block)$.
In particular, a block $\block$ at height $h\eqdef \head\left(\block\right)_h$ should have target difficulty
\begin{align}
	\head\left(\block\right)_d\eqdef \begin{cases}
		\mathbf{d}_0 & h= 0\\
		\mathbf{d}_{\left\lfloor \frac{ h-1}{\tau} \right\rfloor} & \mbox{$h>0$, $\mathbf{d}_{\left\lfloor \frac{ h-1}{\tau} \right\rfloor}$ is  calculated with respect to $\past(\block)$}
	\end{cases} 
\end{align}

\paragraph{Epoch difficulty.}
As soon as all nodes agree on the pivot block $\block^{(k)}$ at the $k$-th epoch $\epoch_k$,
we can uniquely define the target difficulty of $\epoch_k$ as the target difficulty of $\block^{(k)}$.
Formally, 
\begin{align}
	\mathbf{d}_{\epoch_k}\eqdef \head\left( \block^{(k)} \right)_d
\end{align}
where $ \head\left( \block^{(k)} \right)_d$ is the {\bf difficulty} field in $\block^{(k)}$'s header and it equals to $\mathbf{d}_{\left\lfloor \frac{k-1}{\tau} \right\rfloor}$ derived from the past view of $\block^{(k)}$.


% \paragraph{Threshold condition.}
% 	We say that a block $\block$ fulfills the threshold condition of difficulty adjustment if $\mathbf{d}\eqdef\block_{\head_d}$ and $\mathbf{d}'\eqdef\parent{\block}_{\head_d}$ satisfy the following:
% 	\begin{align}
% 		\begin{cases}
% 			\mathbf{d}'\times\diffdown \le \mathbf{d} \le \mathbf{d}'\times \diffup & \text{if $h\eqdef \head\left(\block\right)_h$ is a height for difficulty adjustment, i.e.~$h> \difficultyadjustperiod$ and $\difficultyadjustperiod \;|\; (h-1)$}\\
% 			\mathbf{d} = \mathbf{d}' & \text{otherwise}
% 		\end{cases}
% 	\end{align}


% \subsection{Definition of Work}

% \lipsum[16-17] % Dummy text









%------------------------------------------------

% !TEX root=./tech-specification.tex

\section{Incentive Mechanism}
\label{sec:incentive}

\name miners get paid by \name coins from two sources: the newly minted \name coins as block award,
and the fees paid by transaction senders.
In this section we specify the mechanism design for incentivizing \name miners. 
\newversion{The adaptive weight introduced by the GHAST rule only affects the distribution of the first part of block award.}




\subsection{Base Block Award}
\label{subsec:baseaward}
The amount of coins issued to miners in every block is set in accordance to a global parameter which follows the mining schedule.
 % as stated in Section~\ref{sec:mining schedule}.
We refer to the global parameter as the \emph{base block award} or simply \emph{base award}, and denote it by $\baseaward$. 



	The base block award starts at $\baseaward(\gblock) = \initialblockreward$ \coinsign per block,
	and reduces to $\tanzanitebasereward$ \coinsign stated at the \tanzaniteepoch epoch. 
	% The based block award is adjusted in granularity of quarter. (i.e., $\decayperiodinblock$ blocks, $\decayperiodinday$ days in expectation).
	% In the first $\decaystartinquarter$ quarters (i.e. roughly $\decaystartinyear$ years), the base block award is the initial value $\initialblockreward$ \coinsign per block.
	% In the next $\decayquarters$ quarters (i.e. roughly $\decayyears$ years), it decreases by $\sqrt[\decayquarters]{1/4}$ (about $\decaybypercent\%$) each quarter. 
	% %
	% Eventually the block reward is fixed at $\eventualblockreward$ \coinsign per block and annual inflation rate of mining issuance is below $\targetinflationpercent$ percent.
	% 升级公告
	For every pivot block $\block$, the base award is defined as follows:
	\begin{align}
		\baseaward(\block)&\eqdef
		\begin{cases}
			\initialblockreward \times 10^{18} & \mbox{if $\block_{\head_h}<\tanzaniteepoch$} \\
			\tanzanitebasereward \times 10^{18} & \mbox{if $\block_{\head_h}\ge \tanzaniteepoch$}
		\end{cases}
	\end{align}

	For every non-pivot block $\block$,
	the base award $\baseaward(\block)$ of $\block$ 
	equals to the base award of the pivot block of the epoch that $\block$ belongs to,
	i.e. 
	\begin{align*}
		\baseaward(\block)\eqdef \baseaward(\pivot{\block})
	\end{align*}

	Based on $\baseaward(\block)$, 
	\name defines the the actual block award issued to the author of block $\block$ with  adjustments as described in the rest of Section~\ref{subsec:baseaward}.



\subsubsection{Anti-cone Penalty}
\label{subsec:anticone}

For every block $\block$, we recall that a block $\block'$ is in the anti-cone of $\block$ if there is no directed path between $\block'$ and $\block$, which means the chronological order of these two blocks is not reflected by the underlying \tg.
For every given \tg $\graph$, let $\anticone(\block;\graph)$ denote the set of all anti-cone blocks of $\block\in\graph$ that appear no later than $\anticonecountepoch$ epochs after\footnote{If $\block$ is not on the pivot chain in $\graph$, then $\anticone(\block;\graph)$ also contains blocks appearing in earlier epochs but not referenced by $\block$.} the epoch where $\block$ resides in.
When the \tg $\graph$ is clear from context, we write $\anticone(\block)$ instead of $\anticone(\block;\graph)$ for short. 
Formally,
\begin{align}
	\anticone(\block)\eqdef \anticone(\block;\graph) \eqdef  \set{\block' \in \graph \;\big|\; \block\notin\past(\block') \land \block'\notin\past(\block)  \land \head\left(\pivot{\block'}\right)_h\le \head\left(\pivot{\block}\right)_h+10 }   
\end{align}
In other words, let $\block^{10}$ be the pivot block at height $\head(\block^{10})_h=\head\left(\pivot{\block}\right)_h+10$,
then 
\begin{align}
	\anticone(\block)\eqdef \anticone(\block;\graph) \eqdef \past(\block^{10})\backslash\left( \past(\block)\union \future(\block;\graph) \union \set{\block} \right)
\end{align}

The \emph{anti-cone penalty factor} of $\block$ is defined as
\begin{align}
	\af(\block) \eqdef \max\set{0, 1-\left(\frac{ \weight(\anticone(\block))/\mathbf{d}_{\epoch(\block)}
	% \max\set{\block_d,\mathbf{d}_{\epoch(\block)}} 
	}{\gamma}\right)^2}
\end{align}
where $\gamma\eqdef {\anticoneconstant}$ is a fixed constant and 
\newversion{$\weight(\anticone(\block)) \eqdef \sum_{\block'\in\anticone(\block)} \weight(\block')$} refers to the total {adapted weight} of blocks in the anti-cone set $\anticone(\block)$.
% \newversion{\begin{align}
%  	 \weight(\anticone(\block))\eqdef \sum_{\block'\in\anticone(\block)} \weight(\block')
% \end{align} 
% }
We remark that $\weight(\anticone(\block))/\mathbf{d}_{\epoch(\block)}$ is the equivalent number of blocks in the anti-cone of $\block$, which corresponds to the portion of computing power in $\block$'s anti-cone.
% \newversion{
% 	% As long as the anti-cone of $\block$ is equivalently less than $\gamma$ standard blocks in $\epoch(\block)$, $\block$ is entitled a positive reward.
% 	In particular, if $\af(\block)=0$, then $\block$ is also treated as partially valid, and hence $\basef(\block)=0$.
% }

This anti-cone penalty factor is introduced to incentivize inclusion of {referee} blocks as well as fast propagation.
It also punishes withholding attacks when the blocks are not broadcast immediately. 
There is no additional award for referencing {referee} blocks, nor discount in block award for  non-pivot blocks.


\subsubsection{Base Factor}
\label{sec:discount}

For convenience, we introduce the \emph{base factor} $\basef(\block)$ to indicate whether the author of $\block$ is eligible to receive any award.

If a block $\block$ in $\epoch_k$ has a lower target difficulty, i.e. $\block_d<\mathbf{d}_{\epoch_k}$,
then we decide the base award of $\block$ by its 
block quality $\quality(\block)$:
it gets normal base award if 
the block quality $\quality(\block)$ 
reaches the epoch's target difficulty $\mathbf{d}_{\epoch_k}$, and zero base award $\baseaward(\block)\eqdef 0$ in case the 
quality $\quality(\block)$ does not meet $\mathbf{d}_{\epoch_k}$.
Note that the expected base award is effectively the same as setting $\baseaward(\block)\eqdef\frac{\block_d}{\mathbf{d}_{\epoch_k}} \cdot \baseaward(\epoch_k)$.


	If a block $\block$ is partially valid, blamed, or has a large anti-cone, then the author of $\block$ must have made some mistake and hence he is not eligible for any award.


Thus, the base factor $\basef(\block)$ of block $\block$ is defined as
\begin{align}\label{eq:basef}
	\basef(\block)\eqdef \begin{cases}
		% {\color{red} w} & {\color{red} \mbox{$\block$ is a valid heavy block}}\\
		1 & \mbox{$\block$ is valid and $\af(\block)>0$, \newversion{not blamed, and satisfies the target difficulty of $\epoch_k$ (which requires  }}\\
		~ & \mbox{ $\quality(\block)\ge \mathbf{d}_{\epoch_k}$)}\\
		0 & \mbox{otherwise ($\block$ is partially valid, \newversion{$\af(\block)=0$, blamed}, or $\quality(\block)$ does not satisfy the requirement)} 
	\end{cases}
\end{align}

\subsubsection{Actual Block Award to Miners}
	\label{subsubsec:actualblockaward}
	Taking all the discounts and adjustments into account,
	the block award assigned to the author of $\block$ is defined as follows:
	\begin{align}
		\award(\block) \eqdef \left\lfloor\af(\block)\cdot \basef(\block)
		% \cdot \wf(\block)
		\cdot \baseaward(\block)\right\rfloor
		% =\af(\block)\cdot \basef(\block)
		% % \cdot \wf(\block)
		% \cdot \baseaward
	\end{align}


	\noindent{\bf Remark:} 
	A non-pivot block $\block_1$ may receive a higher block award than the pivot block $\block_2$ 
	% with $\wf(\block_1)=\wf(\block_2)$
	in the same epoch,
	in case $\weight(\anticone(\block_1))<\weight(\anticone(\block_2))$ 
	and hence $\af(\block_1)>\af(\block_2)$.




	\subsection{Storage Maintenance Reward}
	\label{subsec:storagefee}

	Miners receive interest generated by collateral for storage, as payment to the cost of occupying storage space in the world-state.
	More specifically, the CFS interest generated by all blocks in each epoch is redistributed to authors of blocks in the epoch with respect to their actual mining block award.
	In particular, the CFS interest assigned to the author of block $\block$ is calculated as follows:
	\begin{align}
		\storageaward(\block) \eqdef 
		\left\lfloor \sum_{\block'\in\epf(\block)} \left\lfloor \mathsf{ACFS}(\st(\block'))
		\times \frac{\interest}{\blockinyear}\right\rfloor
		\times \frac{\award(\block)}{\sum_{\block'\in\epf(\block)} \award(\block') } \right\rfloor
	\end{align}
	where $\st(\block')$ denotes the world-state at the beginning of the execution of transactions in block $\block'$,
	$\mathsf{ACFS}(\st(\block'))$ is the total CFS in $\block'$, 
	$\interest$ is the annual interest rate and $\blockinyear$ is the (expected) number of blocks in one year,
	and hence the value in parenthesis is the CFS interest generated by $\block'$;
	and the distribution of CFS interest in $\epf(\block)$ is proportional to actual block awards $\award(\block)$ as defined in Section~\ref{subsubsec:actualblockaward}. 

	Specially, if the total block reward for the whole epoch is zero, (i.e., $\sum_{\block'\in\epf(\block)} \award(\block')=0$), the storage maintenance reward will not be distributed. 




\subsection{Transaction Fee Reward}

If a transaction $\tx$ is first executed successfully in the $i$-th epoch $\epoch_i$, then 
the transaction fee (for purchasing the consumed gas) of $\tx$ is divided between all blocks that 
% reside in $\epoch_i$ and include $\tx$.
\emph{properly include }$\tx$.
Here ``a block $\block$ properly includes a transaction $\tx$'' means that:
% a) $\forall \block'\in\past(\block), \tx\notin\block'_{\txs}$;
%  and  
% b)
% and ``$\block$ includes $\tx$ not too late'' means that 
$\tx\in \block$ and $\block$ belongs to $\epoch_i$ (the epoch that $\tx$ is executed for the first time).


The transaction fee is distributed proportionally to the binary base factors of blocks as defined in (\ref{eq:basef}). 
% in Section~\ref{sec:discount}.
% That is, a block with a fully valid header and actual quality above the epoch target difficulty gets weight $1$ %({\color{red}or $w$ depending on whether it is a heavy block})
% , and other blocks with partially valid headers or lower-than-target actual quality gets weight $0$.
% 
In particular, if the transaction $\tx$ is exclusively packed in blocks with zero base factor, the transaction fee is burnt although the transaction will still be processed. 

\oldversion{The total transaction fee reward of a block $\block$ is the sum of its portion of fees from every transaction in $\block_\txs$.}
\newversion{After execution of block $\block$, we can get its receipt list $\mathbf{R}'(\block)$ with the same length as transaction list $\block_\txs$. So each transaction $\tx\in \block_\txs$ can be paired with its corresponding receipt $R$. The actually charged transaction fee is recorded in $R_f$.}
Recalling that $\basef(\block)$ is a binary function respecting to the validity of $\block$,
 the transaction fee assigned to $\block$ is defined as follows:
\begin{align}\label{eq:txfee}
	\feeaward(\block)\eqdef
		 \sum_{\text{$(\tx,R):$ $\block$ properly includes $\tx$}}
		 \left\lfloor
	\frac{R_f \cdot\basef(\block)}{\sum_{\block': \text{$\block'$ properly includes $\tx$}} \basef(\block') }
		 \right\rfloor
\end{align}

When there are multiple blocks properly include the transaction $\tx$, there may be a remainder for transaction fee of $\tx$. Formally, the remainder $r$ equals to

\begin{align}
	r\eqdef R_f - \sum_{\block': \text{$\block'$ properly includes $\tx$}} \basef(\block') \cdot \left\lfloor\frac{R_f \cdot\basef(\block)}{\sum_{\block': \text{$\block'$ properly includes $\tx$}} \basef(\block') }\right\rfloor.
\end{align}

In case the remainder is non-zero, the blocks with $r$ minimum block hash $\kec(\rlp(\block_\head))$ will receive one more $\unit$. So the transaction fee can be exactly distributed to miners. 


\subsubsection{Why not distributing transaction fees among all blocks in that epoch?}
	One may suggest that the transaction fee of $\tx$ should be shared by all valid blocks in that epoch, rather than among blocks that properly include $\tx$ as described in (\ref{eq:txfee}).
	In what follows we show that our current implementation has several advantages:
	\begin{itemize}
		\item {\bf For security and incentive compatibility:} 
		the first priority of the incentive mechanism design is to guarantee that every rational participant will behave honestly, i.e. they should respect the consensus protocol and reference all the blocks they have observed.
		However, the mechanism of sharing transaction fee may bring incentives that prevent the author of a block referencing other blocks.

		For example, if there is a transaction $\tx$ with a considerable fee, miners may be willing to monopolize that fee even if suffering some punishment caused by a larger anti-cone.
		In particular, the author who packs $\tx$ into a new block $\block$ would prefer not to share the transaction fee of $\tx$ with others, especially when this fee is much higher than the anti-cone punishment caused by ignoring other blocks that are eligible to share the fee of $\tx$.

		If the fee of $\tx$ is to be shared with all other blocks in the same epoch, the author of $\block$ would \emph{be hostile to all other unreferenced blocks}, and hence they get incentives to ignore those blocks by shirking all responsibility to network latency. 
		Therefore, by manipulating transaction fees and causing such intentional ignorance, an attacker is able to effectively launch a partition attack.

		If the fee of $\tx$ is to be shared only among blocks including $\tx$, the author of $\block$ may still want to ignore other blocks with $\tx$. 
		However, the difference is that the author of $\block$ would not be hostile to blocks without $\tx$, since he does not have to share the transaction fee of $\tx$ with them.
		As a result, the author of $\block$ will be open to reference other observed blocks without $\tx$ in $\block$.
		Furthermore, referencing blocks with $\tx$ in any later epoch is also safe, even if there are new transactions (other than $\tx$) with significant fees.
		Thus by manipulating transaction fees the attacker could merely cause a slightly longer latency instead of a significant partition.

		In conclusion, sharing transaction fee of $\tx$ only among blocks including $\tx$ provides a stronger incentive of referencing other blocks honestly.

		\item {\bf For efficiency:} 
		the current design better discourages ``free riders'' who only mines on empty blocks and hence saves the effort of  maintaining storage states and executing transactions.
		Since by producing empty blocks, those miners would lose the transaction fees completely, rather than receiving an average revenue of total fees in that epoch.

		\item {\bf For fairness:}
		the current design indeed implements a tailored version of Shapley value, which is a fair distribution of total gains well-known in cooperative game theory.
	\end{itemize}




\subsection{Final Reward to Miners}

The final mining reward of a block $\block$ will be added to the block author's account as specified in the {\bf author} field $\head(\block)_a$.
Since the anti-cone of $\block$ may contain blocks up to $\anticonecountepoch$ epochs after the $\epoch$ of $\block$,
the reward assigned to $\block$ effects at the end of $\minerfreeze$ epochs after the $\epoch$ of $\block$, where the balance of $\block_a$ is updated.
 % Due to \deferblk-blocks deferred execution, this reward becomes available at $\minerfreeze+\deferblk$ epochs after the epoch of $\block$. 
For example, if $\block$ appears in $\epoch_i$, then the account $\block_a$ receives the mining reward at the end of $\epoch_{i+\minerfreeze}$. 
% Such a receive will be executed at the end of $\epoch_{i+\minerfreezeexec}$.
The total mining reward of $\block$ is calculated as follows:

\oldversion{
	\begin{align}
		\reward(\block)\eqdef \af(\block)\cdot \award(\block)  + \feeaward(\block) = \af(\block)\cdot\basef(\block)\cdot\wf(\block)\cdot\award(\epoch_\block) + \feeaward(\block)
	\end{align}
}
\newversion{
	\begin{align}
	\reward(\block)\eqdef  \award(\block) + \storageaward(\block) + \feeaward(\block) 
	% = \af(\block)\cdot\basef(\block)\cdot\wf(\block)\cdot\award(\epoch_\block) + \storageaward(\block) + \feeaward(\block)
\end{align}
}

In particular, note that $\af(\block)=0$ implies $\basef(\block)=0$ and $\award(\block)=\storageaward(\block)=\feeaward(\block)=0$.
Thus  
$\reward(\block)=0$ as long as $\block$ has a large anti-cone such that $\af(\block)=0$.








%------------------------------------------------

% \section{Governance}

% \lipsum[47-49] % Dummy text

%------------------------------------------------
% \pagebreak[4]
\section{Concrete Protocol Implementation}\label{sec:parameters}

% \subsection{Parameters}

% \lipsum[50-53] % Dummy text

Concretely, we set the following parameters for \name.\\ 
{\color{red} Note that the numbers here are only for the most recent testnet and may change in the future.}

\par
\begin{center}
\begin{tabular}{ll}
\toprule
Parameter & Value \\
\midrule
Block time & $\blocktime$ s \\
Maximum block size bound & \maxblocksize \\
Starting coinbase award & $\initialblockreward$ \coinsign \\
Starting difficulty ($\mathbf{d}_0$) & $\startdifficulty=\startdifficultyline$ \\
Starting block gas limit & $\startblockgastlimit=\startblockgastlimitline$ \\
Anti-cone penalty factor ($\gamma$) & $\anticoneconstant$ \\
Deferred execution & \deferblk epochs\\
Mining reward freezing time & \minerfreeze\xspace epochs \\
Snapshot period & \snapshotperiod\xspace epochs \\ 
% Internal Contract Address & \\
AdminControl Contract Address& $\admincontract$ \\
SponsorControl Contract Address& $\sponsorcontract$ \\
Staking Contract Address& $\stakingcontract$ \\
\bottomrule
\end{tabular}
\end{center}
\par


In \name we use $\kec$ as the collision-resistant hash function unless otherwise explicitly specified.

For authentication in the current version of \name, we use the same recoverable ECDSA signature scheme as in Ethereum \cite{ETH_yellow}. 
This method utilizes the \textsf{SECP-256k1} curve.
 % as described by Courtois et al. [2014], and is implemented similarly to as described by Gura et al. [2004] on p. 9 of 15, para. 3.


%------------------------------------------------
% \newpage
\phantomsection
% \section*{Acknowledgments} % The \section*{} command stops section numbering

% \addcontentsline{toc}{section}{Acknowledgments} % Adds this section to the table of contents


%----------------------------------------------------------------------------------------
%	REFERENCE LIST
%----------------------------------------------------------------------------------------
\phantomsection
\bibliographystyle{unsrt}
\bibliography{reference}


\newpage
\appendix

\section{Checklist for porting EVM contract to Conflux}

The Ethereum contract is also a valid Conflux contract. So Ethereum contracts can be ported to Conflux easily and have almost same execution results. But notice that Conflux may have different behavior in the following points:
\begin{itemize}
	\item {\bf Gas used and refund:} Conflux requires less gas in SSTORE operation but no longer refunds resetting storage and contract destruction. 
	\item {\bf Gas fee refund:} Conflux will refund at most 1/4 of gas limit. So try to provide an accurate estimation for gas limit before signing transactions. 
	\item {\bf Contract address:} Conflux uses a different way to compute address for normal account from public key and compute contract address in contract creation. (See equation~(\ref{eq:account-address}) and (\ref{eq:new-address}) for details.) The contract developers usually don't need to handle this difference. 
	\item {\bf Contract address conflict:} If the contract address has existed before contract creation, Conflux will abort the contract creation. This is different with the behavior in Ethereum. 
	\item {\bf Collateral for storage:} Conflux requires collateral for storage. Please make sure there is enough balance for storage collateral. 

\end{itemize}

% \newpage
\section{Difference between Ethereum and {\name}}
\begin{center}
	\begin{tabu}{l X X l}%[htbp]
		% \caption{Difference between Ethereum and {\name}}
		% \smallskip
		% \begin{tabularx}{\textwidth}{lXXl}
			\toprule
			 &  
			\textbf{Ethereum} & \textbf{{\name}} &  \\
			\midrule
			\multicolumn{4}{l}{(Virtual Machine and transaction execution)}\\
			\hline
			Address	type & indistinguishable for all accounts  
			& distinct prefixes for normal (non-contract) account, \emph{Solidity} contracts, and reserved contracts (a.k.a. ``precomipled contracts'')  & \cref{subsec:accounts} \\
			\hline
			Transaction field	& --	& added {\bf chainID}, {\bf storageLimit} and {\bf epochHeight}	& \cref{sec:tx} 
			\\
			% \hline
			% Block field & {\bf nonce} has $64$ bits & {\bf nonce} has $256$ bits & \cref{sec:block}\\
			% & {\bf number} records number of all ancestor blocks  & {\bf height} records number of ancestor blocks on the pivot chain &\\
			% & {\bf stateRoot} etc. commit to the state after executing the latest block & {\bf deferredStateRoot} etc. commit to the state of deferred execution &\\
			% & --  & added {\bf blame}, {\bf adaptiveWeight} &
			% \\
			\hline
			Gas consumption and refunding & all unused gas is refundable & at most a quarter of {\bf gasLimit} & \cref{subsec:gas_and_pay} \\
			\cline{2-4}			
			& full gas fee charged if execution fails on any exception 
			& no gas fee when exception is not caused by the sender  &  \cref{subsubsec:preprocessing} \\
			\hline
			Cost of storage & one-off gas fee &  gas fee + collateral & \cref{subsec:storage consumption}  \\
			\cline{2-4}			
			% Fee schedule ($\hyperlink{SSTORE}{\op{SSTORE}}$)
			& $\hyperlink{SSTORE}{\op{SSTORE}}$ costs $5000$ or $20000$ gas depending on the effect of executing this instruction, may cause a refund of $15000$ gas for clearing a storage value 
			& $\hyperlink{SSTORE}{\op{SSTORE}}$ costs \hyperlink{G__sset}{$G_{sset}$}$=5000$ gas and every \sunitsize storage costs \sunitprice for collateral (locked until the storage is overwritten or released)   
			& \cref{sec:collateral}\\
			\hline
			Transaction validation & 
			{any invalid transaction leads to the whole block being invalid} & 
			{invalid transactions are skipped, while other transactions in the same block can still be valid} & \cref{sec:txorder}\smallskip\\
			\cline{2-4}
			& {a transaction is invalid if sender's balance cannot afford the up-front payment for transferred value and gas fee (indeed the whole block will be invalid)} & {the transaction is valid if it satisfies all other assertions, but the execution fails immediately because of insufficient balance for the up-front payment (sender's nonce is increased and gas fee is charged)}	& \cref{sec:tx validate} \\\cline{2-4}
			% \hline
			& sender must pay transaction fee from his own balance (sender's balance cannot be zero)	& a sponsor may pay for the cost of calling a smart contract (sender's balance can be zero)	& \cref{sec:sponsor} \\
			\cline{2-4}
			& validity of a transaction cannot depend on current time or height & a transaction is only valid in a specified window of epochs & \cref{sec:tx} \\
			\cline{2-4}
			& no check on recipient's address & recipient address must have a valid type (i.e. normal account, Solidity contract, or reserved contract)  & \cref{subsec:accounts} \\
			\hline
			Contract creation & the address of contract created by $\hyperlink{CREATE}{\op{CREATE}}$ does not depend on the initialization code  &  the address of contract created by both $\hyperlink{CREATE}{\op{CREATE}}$ and $\hyperlink{CREATE2}{\op{CREATE2}}$ depends on the initialization code &  \cref{eq:new-address} \\
			\cline{2-4} & 
			$\hyperlink{CREATE}{\op{CREATE}}$ costs $G_{create}=32000$, regardless initilization code length	
			& $\hyperlink{CREATE}{\op{CREATE}}$ costs the same as $\hyperlink{CREATE2}{\op{CREATE2}}$  & \cref{eq:gas_cost} \\
			\cline{2-4}
			& the maximum size of the byte-code is 24756 bytes & The maximum size of the byte-code is 49152 bytes & section~\ref{sec:creation}\\
			\cline{2-4}
			& on address conflict, reset contract but inherit the balance &  
			on address conflict, abort the contract creation   
			& \cref{sec:creation}  \smallskip\\
			\hline
			Contract destruction & only by $\hyperlink{SUICIDE}{\op{SUICIDE}}$ & 
			destruction may effect on request of the contract's administrator
			(via the AdminControl contract) & \cref{sec:admin}\\
			% \hline
			% 	&	&	& \\
			% \hline
			% 	&	&	& \\
			\midrule
			Internal Contract & -- & cannot be invoked by system operation, i.e. via $\op{CALLCODE}$ or $\op{DELEGATECALL}$  
			& \cref{sec:internal} \smallskip\\
			% \hline
			% Fee schedule (internal contracts) & -- & AdminContral: $5000$   & \cref{sec:internal} \\
			% &  & StakingContral: $10000$   &   \\
			% &  & SponsorContral: $10000$ for {\tt set\char`_sponsor\char`_for\char`_gas/collateral()}, $5000$ for each member update in {\tt add/remove\char`_privilege()} &  \\
			% \midrule
			% \multicolumn{4}{l}{(Instruction Set)}\\
			\hline
			$\hyperlink{BLOCKHASH}{\op{BLOCKHASH}}$ 	&  get the hash of one of the 256 most recent complete blocks	& get the hash of the last block, return zero if querying other blocks	& \cref{app:instruction-set}\\
			\hline
			$\hyperlink{CHAINID}{\op{CHAINID}}$ 	& EIP-1344	&  get the {\name} chain ID ($503$)	&  \cref{app:instruction-set}\\
			% \hline
			% $\hyperlink{BEGINSUB}{\op{BEGINSUB}}$ 	& OpCode = 0x5c (EIP-2315)	& OpCode = 0xb2 	& \cref{app:instruction-set}\\
			% \hline
			% $\hyperlink{JUMPSUB}{\op{JUMPSUB}}$ 		& OpCode = 0x5d (EIP-2315)	& OpCode = 0xb3	& \cref{app:instruction-set}\\
			% \hline
			% $\hyperlink{RETURNSUB}{\op{RETURNSUB}}$ 	& OpCode = 0x5e (EIP-2315)	& OpCode = 0xb7	& \cref{app:instruction-set}\\
			% \midrule
			% \multicolumn{4}{l}{(Implemented in Ethereum code but not included in Ethereum yellowpaper)}\\
			% \hline
			% $\hyperlink{SELFBALANCE}{\op{SELFBALANCE}}$ 	&  EIP-1884	& push the balance of the currently executing account to the stack.	& \cref{app:instruction-set}\\	
			\bottomrule
		% \end{tabularx}
		% \begin{tablenotes}
		% 	\item[$\ast$] Slightly better security than confirmation with $6$ successive blocks in Bitcoin.

		% 	% \item[$\ast\ast$] It is possible that the fast confirmation rule is never satisfied for a predetermined risk tolerance,{}
		% 	% in which case the block can only be confirmed with the Slow Confirmation Rule.
		% \end{tablenotes}
	\end{tabu}
\end{center}

% \newpage
\section{Fee Schedule}\label{app:fees}

The fee schedule $G$ is a tuple of $35$ scalar values corresponding to the relative costs, in gas, of a number of abstract operations that a transaction may effect.

\begin{center}
	\begin{tabu}{l r l}
		\toprule
		Name & Value & Description* \\
		\midrule
		$G_{zero}$ & 0 & Nothing paid for operations of the set {\small $W_{zero}$}. \\
		$G_{base}$ & 2 & Amount of gas to pay for operations of the set {\small $W_{base}$}. \\
		$G_{verylow}$ & 3 & Amount of gas to pay for operations of the set {\small $W_{verylow}$}. \\
		$G_{\mathrm{low}}$ & 5 & Amount of gas to pay for operations of the set {\small $W_{\mathrm{low}}$}. \\
		$G_{mid}$ & 8 & Amount of gas to pay for operations of the set {\small $W_{mid}$}. \\
		$G_{\mathrm{high}}$ & 10 & Amount of gas to pay for operations of the set {\small $W_{\mathrm{high}}$}. \\
		$G_{extcode}$ & 700 & Amount of gas to pay for an $\op{EXTCODESIZE}$ operation. \\
		$G_{extcodehash}$ & 400 & Amount of gas to pay for an $\op{EXTCODEHASH}$ operation. \\
		$G_{balance}$ & 400 & Amount of gas to pay for a {\small BALANCE} operation. \\
		$G_{sload}$ & 200 & Paid for a $\op{SLOAD}$ operation. \\
		$G_{jumpdest}$ & 1 & Paid for a $\op{JUMPDEST}$ operation. \\
		\linkdest{G__sset}{}$G_{sset}$ & 5000 & Paid for an $\op{SSTORE}$ operation. \\
		\linkdest{R__selfdestruct}{}$R_{suicide}$ & 24000 & Refund given (added into refund counter) for self-destructing an account. \\
		\linkdest{G__selfdestruct}{}$G_{suicide}$ & 5000 & Amount of gas to pay for a $\op{SUICIDE}$ operation. \\
		$G_{create}$ & 32000 & Paid for a $\op{CREATE}$ operation. \\
		\linkdest{G__codedeposit}{}$G_{codedeposit}$ & 200 & Paid per byte for a $\op{CREATE}$ operation to succeed in placing code into state. \\
		$G_{call}$ & 700 & Paid for a $\op{CALL}$ operation. \\
		$G_{callvalue}$ & 9000 & Paid for a non-zero value transfer as part of the $\op{CALL}$ operation. \\
		$G_{callstipend}$ & 2300 & A stipend for the called contract subtracted from $G_{callvalue}$ for a non-zero value transfer. \\
		\linkdest{G__newaccount}{}$G_{newaccount}$ & 25000 & Paid for a $\op{CALL}$ or $\op{SUICIDE}$ operation which creates an account. \\
		$G_{exp}$ & 10 & Partial payment for an $\op{EXP}$ operation. \\
		$G_{expbyte}$ & 50 & Partial payment when multiplied by $\lceil\log_{256}(exponent)\rceil$ for the $\op{EXP}$ operation. \\
		$G_{memory}$ & 3 & Paid for every additional word when expanding memory. \\
		\linkdest{G__txcreate}{}$G_\text{txcreate}$ & 32000 & Paid by all contract-creating transactions after the {\textit{Homestead} transition}.\\
		\linkdest{G__txdatazero}{}$G_{txdatazero}$ & 4 & Paid for every zero byte of data or code for a transaction. \\
		\linkdest{G__txdatanonzero}{}$G_{txdatanonzero}$ & 68 & Paid for every non-zero byte of data or code for a transaction. \\
		\linkdest{G__transaction}{}$G_{transaction}$ & 21000 & Paid for every transaction. \\
		$G_{\mathrm{log}}$ & 375 & Partial payment for a $\op{LOG}$ operation. \\
		$G_{\mathrm{logdata}}$ & 8 & Paid for each byte in a $\op{LOG}$ operation's data. \\
		$G_{\mathrm{logtopic}}$ & 375 & Paid for each topic of a $\op{LOG}$ operation. \\
		$G_{sha3}$ & 30 & Paid for each $\op{SHA3}$ operation. \\
		$G_{sha3word}$ & 6 & Paid for each word (rounded up) for input data to a $\op{SHA3}$ operation. \\
		$G_{copy}$ & 3 & Partial payment for $\op{*COPY}$ operations, multiplied by words copied, rounded up. \\
		$G_{blockhash}$ & 20 & Payment for $\op{BLOCKHASH}$ operation. \\
		$G_{quaddivisor}$ & 20 & The quadratic coefficient of the input sizes of the exponentiation-over-modulo precompiled\\
		&&contract. \\
		\bottomrule
	\end{tabu}
\end{center}

\newpage

\section{Contract destruction}\label{sec:contract_destruct}

The contract destruction function $\Psi(\st,A)$ updates the world state $\st$ and substate $A$ and outputs $(\st',A')$. Formally, function $\Psi$ is defined as follows.

\begin{align}
	\st^*  &\eqdef \st \qquad \mbox{ except:}\\
		\st^*[I_a]_o &\eqdef \st[I_a]_o + \left|\st[I_a]_{\bf c}\right|\times \collateralperbyte \\
		\st^*[a]_b&\eqdef\st[a]_b+\st[I_a]_p[{\sf gas}]_b
		\qquad\text{where $a  \eqdef \st[I_a]_p[{\sf gas}]_a$}\\ 
	\notag \\ 
	\st'  &\eqdef \st^* \qquad \mbox{ except:}\\
	\st'[r] &\equiv \begin{cases}
		\varnothing &\text{if}\ \st^*[r] = \varnothing\ \wedge\ \st^*[I_{a}]_{b} = 0\\
		\begin{array}{l}
			(\st^*[r]_{n}, \st^*[r]_{b} + \st^*[I_{a}]_{b}, \\
			\quad\st^*[r]_{\mathbf{s}}, \st^*[r]_{c})
		\end{array} & \text{if}\ r \neq I_{a}\\
		(\st^*[r]_{n}, 0, \st^*[r]_{\mathbf{s}}, \st^*[r]_{c}) & \text{otherwise}
	\end{cases}\\
	\st'[I_{a}]_{b} &\eqdef 0 \\
		\mbox{where: } &\\ 
		 r &\eqdef \mst_{\mathbf{s}}[0] \bmod 2^{160} \\
	\notag \\
	A' & \eqdef A \\
	\mbox{except:} & \\ 
	A'_{\mathbf{s}} &\eqdef A_{\mathbf{s}} \cup \{I_a\}
\end{align}

\section{Internal contracts}\label{sec:internal_contract}

Conflux introduces internal contracts for specific usage. A high-level description for the internal contracts is given in Section~\ref{sec:internal}. 
%
Currently, Conflux has three internal contracts with addresses as follows. 
\begin{align}
	a_{\sf admin} \eqdef \admincontract \\ 
	a_{\sf sponsor} \eqdef \sponsorcontract \\
	a_{\sf stake} \eqdef \stakingcontract 
\end{align}
%
When the recipient's address $r$ is one of the internal contracts, Conflux processes $\execute_{\sf internal}(\st^*,g,I)$ and returns $\left(\st^{**}, g^{**},  A, \vec{o} \right)$. 

\subsection{Interfaces and gas required}\label{sec:internal_gas}

In execution of internal contracts, the call data $I_\mathbf{d}$ is interpreted as a function call to Solidity interface. Function $V(\st,c,I_\mathbf{d})$ gives the gas used for internal contract execution by called function and parameters.
$\\$

\begin{tabu}{llll}
	\hline
	Address $I_a$ & Solidity interface & Formal parameters & $V(\st,c,I_\mathbf{d})$ \\
	\hline
	\multirow{2}{*}{$a_{\sf admin}$} & {\tt \small set\char`_admin(address,address)} & $a_0,a_1\in \B_{160}$ & $G_{sset}$\\\cline{2-4}
	& {\tt \small destroy(address)} & $a_0\in \B_{160}$ & $G_{sset}$ \\\hline
	\multirow{4}{*}{$a_{\sf sponsor}$} & {\tt \small set\char`_sponsor\char`_for\char`_gas(address,uint256)}  & $a_0\in \B_{160},n_1\in \N_{256}$ & $G_{sset}$\\\cline{2-4}
	& {\tt \small set\char`_sponsor\char`_for\char`_collateral(address)}  & $a_0\in \B_{160}$ &  $2\times G_{sset}$\\\cline{2-4}
	& {\tt \small add\char`_privilege(address[])} & $\mathbf{a}\in \B_{160}^*$ & $|\mathbf{a}|\times G_{sset}$ \\\cline{2-4}
	& {\tt \small remove\char`_privilege(address[])} & $\mathbf{a}\in \B_{160}^*$ & $|\mathbf{a}|\times G_{sset}$ \\ \hline
	\multirow{3}{*}{$a_{\sf stake}$} & {\tt \small deposit(uint256)}  & $n_0\in \N_{256}$ & $2\times(|\st[s]_{deposit}|+1)\times G_{sset}$\\\cline{2-4}
	& {\tt \small withdraw(uint256)}  & $n_0\in \N_{256}$ &  $2\times|\st[s]_{deposit}|\times G_{sset}$\\\cline{2-4}
	& {\tt \small vote\char`_lock(uint256,uint256)} & $n_0,n_1\in \N_{256}$ & $2\times|1+\tolist(\st[s]_{vote})|\times G_{sset}$ \\\hline
	\multicolumn{4}{l}{where $\st[s]'_{vote}\text{ is defined by }\st[s]_{vote} \text{ removing all the key }x\text{ with }\st[s]_{vote}[x]=\st[s]_{vote}[x-1]$}
\end{tabu}

\subsection{Internal contracts exceptions}
The execution of internal contracts may fail in the following cases 
\begin{itemize}
	\item Conflux parses the call data $I_\mathbf{d}$ as solidity function call. In case the call data doesn't match any solidity function interfaces list as follows, or the format is incorrect, the execution fails. 
	\item The recipient is inconsistent with the code address or internal contract is called by \op{STATICCALL}, i.e., $I_r\neq c$ or $I_w=\bot$. 
	\item The gas $g$ passed in is not enough for internal contract execution, i.e., $g<V(\st^*,c,I_\mathbf{d})$. 
	\item The staking vote contract $a_{\sf stake}$ forbids value transfer to prevent misusing, i.e., $I_a= a_{\sf stake}\;\wedge \; I_v\neq 0$. \footnote{User don't need to transfer balance to staking vote contract in staking. The transferred value to staking vote contract will lost. }
	\item If other exceptions met during the execution, the execution also fails. See the following for details. 
\end{itemize}

Whenever the execution fails, the return values are set as
\begin{align}
	\left(\st^{**}, g^{**},  A, \vec{o} \right)\eqdef (\varnothing,0,A^0,\varnothing)
\end{align} 

If the execution is successful, the remained gas and return data are set as 
\begin{align}
	\left(g^{**}, \vec{o} \right)\eqdef (g-V(\st^*,c,I_\mathbf{d}),\emptystring)
\end{align}

The resultant world-state $\st^{**}$ and substate $A$ is computed depending on interfaces. 


\subsection{Admin Contract}

The admin contract with address $a_{\sf admin}$ gives two interfaces.

\subsubsection{Set administration}

For interface {\tt set\char`_admin(address,address)}, let $a_0,a_1\in \B_{160}$ be the address parameters. 
	\begin{align}
		A & \eqdef A^0 \\
		\st^{**} &\eqdef \st^* \qquad \mbox{except:}\\
		\st^{**}[a_0]_a &\eqdef
			a_1 \qquad \text{if } \mathsf{Type}_{a_0} = [1000]_2 \;\wedge\; I_o = \st^*[a_0]_a 
    \end{align}
    
\subsubsection{Destory contract} 

For interface {\tt destroy(address)}, let $a_0\in \B_{160}$ be the address parameters. The execution fails if 
	\begin{align}
			\st^*[a_0]_o>0\;\wedge\; 	I_o = \st^*[a_0]_a
	\end{align}
	If the execution not fails, 
	\begin{align}
		(\st^{**},A) \eqdef \left\{\begin{array}{ll}
			(\st^*,A^0) & I_o \neq \st^*[a_0]_a \\
			\Psi(\st^*,A^0) & I_o = \st^*[a_0]_a
		\end{array}\right.
	\end{align}
	where $\Psi$ is defined in section~\ref{sec:contract_destruct}.

\subsection{Sponsorship Contract}
The sponsorship contract with address $a_{\sf sponsor}$ gives four interfaces.

\subsubsection{Set sponsor for gas}

For interface {\tt set\char`_sponsor\char`_for\char`_gas(address,uint256)}, let $a_0\in \B_{160},n_1\in \N_{256}$ be the address parameters. The execution fails if 
	\begin{align}
		& \st^*[a_0]=\emptyset \;\vee\; \mathsf{Type}_{a_0}\neq[1000]_2 \;\vee\; \st^*[a_{\sf sponsor}]_b < 1000\times n_1 \;\vee\; n_1\le \st^*[a_0]_p[{\sf limit}]\le \st^*[a_0]_p[{\sf gas}]_b \notag \\ 
		& \;\vee\; (\st^*[a_0]_p[{\sf gas}]_a \neq I_s \;\wedge\; \st^*[a_{\sf sponsor}]_b\le \st^*[a_0]_p[{\sf gas}]_b)  
	\end{align}
	If the execution not fails, 
		\begin{align}
			A & \eqdef A^0 \\
			\st^{**} &\eqdef \st^* \qquad \mbox{except:}\\
			\st^{**}[a_{\sf sponsor}]_b &\eqdef 0 \\ 
			\st^{**}[a_0]_p[{\sf limit}] &\eqdef n_1\\ 
			\st^{**}[a_0]_p[{\sf gas}]_a &\eqdef a_0\\ 
			\st^{**}[a_0]_p[{\sf gas}]_b &\eqdef \begin{cases}
				\st^*[a_{\sf sponsor}]_b + \st^*[a_0]_p[{\sf gas}]_b& p = I_s\\ 
				\st^*[a_{\sf sponsor}]_b& p \neq I_s
			\end{cases}\\ 
			\st^{**}[p]_b &\eqdef \begin{cases}
				\st^*[p]_b & p = I_s\\ 
				\st^*[p]_b + \st^*[a_0]_p[{\sf gas}]_b& p \neq I_s
			\end{cases}\\
			\mbox{where:}& \\
			p & \eqdef \st^*[a_0]_p[{\sf gas}]_a
		\end{align}
    
\subsubsection{Set sponsor for collateral}

For interface {\tt set\char`_sponsor\char`_for\char`_collateral(address)}, let $a_0\in \B_{160}$ be the address parameter. The execution fails if 
	\begin{align}
		& \st^*[a_0]=\emptyset \;\vee\; \mathsf{Type}_{a_0}\neq[1000]_2 \;\vee\; \st^*[a_{\sf sponsor}]_b = 0  \notag \\ 
		& \;\vee\; (\st^*[a_0]_p[{\sf col}]_a \neq I_s \;\wedge\; \st^*[a_{\sf sponsor}]_b\le \st^*[a_0]_p[{\sf col}]_b)  
	\end{align}
	If the execution not fails, 
		\begin{align}
			A & \eqdef A^0 \\
			\st^{**} &\eqdef \st^* \qquad \mbox{except:}\\
			\st^{**}[a_{\sf sponsor}]_b &\eqdef 0 \\ 
			\st^{**}[a_0]_p[{\sf col}]_a &\eqdef a_0\\ 
			\st^{**}[a_0]_p[{\sf col}]_b &\eqdef \begin{cases}
				\st^*[a_{\sf sponsor}]_b + \st^*[a_0]_p[{\sf col}]_b& p = I_s\\ 
				\st^*[a_{\sf sponsor}]_b& p \neq I_s
			\end{cases}\\ 
			\st^{**}[p]_b &\eqdef \begin{cases}
				\st^*[p]_b & p = I_s\\ 
				\st^*[p]_b + \st^*[a_0]_p[{\sf col}]_b& p \neq I_s
			\end{cases}\\
			\mbox{where:}& \\
			p & \eqdef \st^*[a_0]_p[{\sf col}]_a
        \end{align}
        
\subsubsection{Add addresses to whitelist}

For interface {\tt add\char`_privilege(address[])}, let $\mathbf{a}\in \B_{160}^*$ be the address list parameter. The execution fails if \begin{align}
		\mathsf{Type}_{I_s}\neq[1000]_2. 
	\end{align}
	If the execution not fails,
	\begin{align}
		\left(\st^{**}, A\right) &\eqdef \left(\st^{(n)}, A^0\right)\\ 
		\mbox{where:} &\\
		n&\eqdef |\mathbf{a}| \\
		\st^{(0)} &\eqdef \st^* \\ 
		\forall j \in [n],\; \st^{(j)} &\eqdef \Phi(\st^{(j-1)},a_{\sf sponsor},I_s\cdot {\bf a}[j-1],1,I_i)
	\end{align}
	Function $\Phi$ is defined in Section~\ref{sec:storage_maintain}.
    
\subsubsection{Remove addresses to whitelist}

For interface {\tt remove\char`_privilege(address[])}, let $\mathbf{a}\in \B_{160}^*$ be the address list parameter. The execution fails if
 \begin{align}
		\mathsf{Type}_{I_s}\neq[1000]_2. 
	\end{align}
	If the execution not fails, 
	\begin{align}
		\left(\st^{**}, A\right) &\eqdef \left(\st^{(n)}, A^0\right)\\ 
		\mbox{where:} &\\
		n&\eqdef |\mathbf{a}| \\
		\st^{(0)} &\eqdef \st^* \\ 
		\forall j \in [n],\; \st^{(j)} &\eqdef \Phi(\st^{(j-1)},a_{\sf sponsor},I_s\cdot {\bf a}[j-1],0,I_i)
    \end{align}
    
\subsection{Staking vote contract}
The staking vote contract with address $a_{\sf stake}$ gives three interfaces:

\subsubsection{Staking}

For interface {\tt deposit(uint256)}, let $n_0\in \N_{256}$ be the integer parameter. The execution fails if 
	\begin{align}
		n_0 < 10^{18} \;\vee\; \st^*[I_s]_b<n_0
	\end{align}
	If the execution not fails, 
	\begin{align}
		A & \eqdef A^0 \\
		\st^{**} &\eqdef \st^* \qquad \mbox{except:}\\
		\st^{**}[I_s]_b &\eqdef \st^*[I_s]_b - n_0\\ 
		\st^{**}[I_s]_t &\eqdef \st^*[I_s]_t + n_0\\ 
		\st^{**}[I_s]_{\bf d} &\eqdef \st^*[I_s]_{\bf d} \cdot (n_0,|I_\mathbf{L}|,\st^*[a_{\sf sponsor}]_{\bf s}[k_1]_v) \\
		\st^{**}[a_{\sf sponsor}]_{\bf s}[k_2]_v & \eqdef\st^*[a_{\sf sponsor}]_{\bf s}[k_2]_v + n_0 \\
		\mbox{where:}& \\
		k_1 & \eqdef [{\sf accumulate\char`_interest\char`_rate}]_{\sf ch} \\
		k_2 & \eqdef [{\sf total\char`_staking\char`_tokens}]_{\sf ch}
    \end{align}
    
\subsubsection{Lock staking to obtain vote power}

For interface {\tt vote\char`_lock(uint256,uint256)}, let $n_0,n_1\in \N_{256}$ be the integer parameters. The execution fails if 
	\begin{align}
		n_1 \le  |I_\mathbf{L}| \;\vee\; \st^*[I_s]_t<n_0
	\end{align}
	If the execution not fails, 
	\begin{align}
		A & \eqdef A^0 \\
		\st^{**} &\eqdef \st^* \qquad \mbox{except:}\\
		\forall x\le |I_\mathbf{L}|, \st^{**}[I_s]_{\bf v}[x] &\eqdef \st^*[I_s]_{\bf v}[|I_\mathbf{L}|+1]\\ 
		\forall |I_\mathbf{L}|<x\le n_1, \st^{**}[I_s]_{\bf v}[x] &\eqdef \max\{\st^*[I_s]_{\bf v}[x],n_0\}
    \end{align}
    
\subsubsection{Withdraw}

For interface {\tt withdraw(uint256)}, let $n_0 \in \N_{256}$ be the integer parameter. The execution fails if 
	\begin{align}
		n_0>\st[I_s]_t - \st[I_s]_{\bf v}[|I_\mathbf{L}|+1]
	\end{align}
	If the execution not fails, 
	\begin{align}
		A & \eqdef A^0 \\
		\st^1 &\eqdef \st^* \qquad \mbox{except:}\\
		\forall x\le |I_\mathbf{L}|, \st^1[I_s]_{\bf v}[x] &\eqdef \st^*[I_s]_{\bf v}[|I_\mathbf{L}|+1]\\ 
		\st^1[I_s]_t &\eqdef \st^*[I_s]_t - n_0\\ 
		\forall i < |\st^*[I_s]_{\bf d}|, \st^1[I_s]_{\bf d}[i][{\sf amt}] &\eqdef \st^*[I_s]_{\bf d}[i][{\sf amt}] - z_i \\
		\st^1[I_s]_b &\eqdef \st^*[I_s]_b + n_0 + q\\
		\st^1[I_s]_r &\eqdef \st^*[I_s]_r + q\\
		\st^1[a_{\sf sponsor}]_{\bf s}[k_2]_v & \eqdef\st^*[a_{\sf sponsor}]_{\bf s}[k_2]_v - n_0 \\
		\st^1[a_{\sf sponsor}]_{\bf s}[k_3]_v & \eqdef\st^*[a_{\sf sponsor}]_{\bf s}[k_3]_v + q \\
		\st^{**} &\eqdef \st^1 \qquad \mbox{except:}\\
		\st^{**}[I_s]_{\bf d} &\eqdef \mbox{Remove elements $e$ from $\st^1[I_s]_{\bf d}$ with $e[{\sf amt}]=0$} \\
		\mbox{where:}& \\
		y_i & \eqdef \sum_{j=0}^{i-1} \st^*[I_s]_{\bf d}[j][{\sf amt}] \\
		z_i & \eqdef \max\{0,\min\{n_0-y_i,\st^*[I_s]_{\bf d}[i][{\sf amt}]\}\} \\
		q & \eqdef \sum_{i=0}^{|\st^*[I_s]_{\bf d}|-1} \left\lfloor \frac{z_i \times \st^*[a_{\sf sponsor}]_{\bf s}[k_1]}{\st^*[I_s]_{\bf d}[i][{\sf accIR}]}\right\rfloor \\
		k_1 & \eqdef [{\sf accumulate\char`_interest\char`_rate}]_{\sf ch} \\
		k_2 & \eqdef [{\sf total\char`_staking\char`_tokens}]_{\sf ch} \\ 
		k_3 & \eqdef [{\sf total\char`_issued\char`_tokens}]_{\sf ch}
	\end{align}




\newpage
% !TEX root=./tech-specification.tex

\section{Virtual Machine Specification}
\label{app:vm}

\subsection{Gas Cost}

Recalling that $w$ denotes the current operation to be executed as in (\ref{eq:currentoperation}):
% Given the execution environment tuple $I$ as in Section~\ref{subsubsec:exe_env},
% let $w$ denote the current instruction specified in \hyperlink{I__b}{$I_{\vec{b}}$} as: 
\begin{equation*}
w \eqdef \begin{cases} I_{\vec{b}}[\mst_{\mathrm{pc}}] & \text{if} \quad \mst_{\mathrm{pc}} < \lVert I_{\mathbf{b}} \rVert\\
\op{STOP} & \text{otherwise}
\end{cases}
\end{equation*}

The general gas cost function, $\cost$, is defined as:

\begin{equation}\label{eq:gas_cost}
\cost(\st, \mst, I) \equiv \cost_{mem}(\mst'_{\mathrm{i}})-\cost_{mem}(\mst_{\mathrm{i}}) + \begin{cases}
G_{sset} & \text{if} \quad w = \hyperlink{SSTORE}{\op{SSTORE}} \\
G_{exp} & \text{if} \quad w = \op{EXP} \wedge \mst_{\mathbf{s}}[1] = 0 \\
G_{exp} + G_{expbyte}\times(1+\lfloor\log_{256}(\mst_{\mathbf{s}}[1])\rfloor) & \text{if} \quad w = \op{EXP} \wedge \mst_{\mathbf{s}}[1] > 0 \\
G_{verylow} + G_{copy}\times\lceil\mst_{\mathbf{s}}[2] \div 32\rceil & \text{if} \quad w = \op{CALLDATACOPY} \> \lor \\
&\quad \op{CODECOPY} \lor \op{RETURNDATACOPY} \\
G_{extcode} + G_{copy}\times\lceil\mst_{\mathbf{s}}[3] \div 32\rceil & \text{if} \quad w = \op{EXTCODECOPY} \\
G_{log}+G_{logdata}\times\mst_{\mathbf{s}}[1] & \text{if} \quad w = \op{LOG0} \\
G_{log}+G_{logdata}\times\mst_{\mathbf{s}}[1]+G_{logtopic} & \text{if} \quad w = \op{LOG1} \\
G_{log}+G_{logdata}\times\mst_{\mathbf{s}}[1]+2G_{logtopic} & \text{if} \quad w = \op{LOG2} \\
G_{log}+G_{logdata}\times\mst_{\mathbf{s}}[1]+3G_{logtopic} & \text{if} \quad w = \op{LOG3} \\
G_{log}+G_{logdata}\times\mst_{\mathbf{s}}[1]+4G_{logtopic} & \text{if} \quad w = \op{LOG4} \\
C_\text{\tiny CALL}(\st, \mst) & \text{if} \quad w = \op{CALL} \lor \op{CALLCODE} \> \lor \\
&\quad\op{DELEGATECALL} \\
C_\text{\tiny SUICIDE}(\st, \mst) & \text{if} \quad w = \op{SUICIDE} \\
G_{create}+G_{sha3word} \times \lceil \mst_{\mathbf{s}}[2] \div 32 \rceil & \text{if} \quad w = \op{\hyperlink{create}{CREATE}}\\
G_{create}+G_{sha3word} \times \lceil \mst_{\mathbf{s}}[2] \div 32 \rceil & \text{if} \quad w = \op{\hyperlink{CREATE2}{CREATE2}}\\
G_{sha3}+G_{sha3word} \times \lceil \mst_{\mathbf{s}}[1] \div 32 \rceil & \text{if} \quad w = \op{SHA3}\\
G_{jumpdest} & \text{if} \quad w = \op{JUMPDEST}\\
G_{sload} & \text{if} \quad w = \op{SLOAD}\\
G_{zero} & \text{if} \quad w \in W_{zero}\\
G_{base} & \text{if} \quad w \in W_{base}\\
G_{verylow} & \text{if} \quad w \in W_{verylow}\\
G_{\mathrm{low}} & \text{if} \quad w \in W_{\mathrm{low}}\\
G_{mid} & \text{if} \quad w \in W_{mid}\\
G_{\mathrm{high}} & \text{if} \quad w \in W_{\mathrm{high}}\\
G_{extcode} & \text{if} \quad w = \op{EXTCODESIZE}\\
G_{extcodehash} & \text{if} \quad w = \op{\hyperlink{extcodehash}{EXTCODEHASH}}\\
G_{balance} & \text{if} \quad w = \op{BALANCE}\\
G_{blockhash} & \text{if} \quad w = \op{\hyperlink{blockhash}{BLOCKHASH}}\\
\end{cases}
\end{equation}


where:
\begin{equation}
C_{mem}(a) \equiv G_{memory} \cdot a + \left\lfloor \dfrac{a^2}{512} \right\rfloor
\end{equation}

with $C_\text{\tiny CALL}$ and $C_\text{\tiny SUICIDE}$ as specified in the appropriate section below. We define the following subsets of instructions:

$W_{zero}$ = \{$\op{STOP}$, $\op{RETURN}$, $\op{REVERT}$\}

$W_{base}$ = \{$\op{ADDRESS}$, $\op{ORIGIN}$, $\op{CALLER}$, $\op{CALLVALUE}$, $\op{CALLDATASIZE}$, $\op{CODESIZE}$, $\op{GASPRICE}$, $\op{COINBASE}$,
\newline \noindent\hspace*{1.75cm} $\op{TIMESTAMP}$, $\op{NUMBER}$, $\op{DIFFICULTY}$, $\op{GASLIMIT}$, $\op{RETURNDATASIZE}$, $\op{POP}$, $\op{PC}$, $\op{MSIZE}$, $\op{GAS}$, 
\newline \noindent\hspace*{1.75cm} $\op{CHAINID}$, $\op{BEGINSUB}$\}

$W_{verylow}$ = \{$\op{ADD}$, $\op{SUB}$, $\op{NOT}$, $\op{LT}$, $\op{GT}$, $\op{SLT}$, $\op{SGT}$, $\op{EQ}$, $\op{ISZERO}$, $\op{AND}$, $\op{OR}$, $\op{XOR}$, $\op{BYTE}$, $\op{SHL}$, $\op{SHR}$, $\op{SAR}$, 
\newline \noindent\hspace*{1.75cm} $\op{CALLDATALOAD}$, $\op{MLOAD}$, $\op{MSTORE}$, $\op{MSTORE8}$, $\op{PUSH*}$, $\op{DUP*}$, $\op{SWAP*}$\}

$W_{\mathrm{low}}$ = \{$\op{MUL}$, $\op{DIV}$, $\op{SDIV}$, $\op{MOD}$, $\op{SMOD}$, $\op{SIGNEXTEND}$, $\op{SELFBALANCE}$, $\op{RETURNSUB}$\}

$W_{mid}$ = \{$\op{ADDMOD}$, $\op{MULMOD}$, $\op{JUMP}$\}

$W_{\mathrm{high}}$ = \{$\op{JUMPI}$, $\op{JUMPSUB}$\}

Note the memory cost component, given as the product of $G_{memory}$ and the maximum of 0 \& the ceiling of the number of words in size that the memory must be over the current number of words, $\mst_{\mathrm{i}}$ in order that all accesses reference valid memory whether for read or write. Such accesses must be for non-zero number of bytes.

Referencing a zero length range (e.g. by attempting to pass it as the input range to a CALL) does not require memory to be extended to the beginning of the range. $\mst'_{\mathrm{i}}$ is defined as this new maximum number of words of active memory; special-cases are given where these two are not equal.

Note also that $C_{mem}$ is the memory cost function (the expansion function being the difference between the cost before and after). It is a polynomial, with the higher-order coefficient divided and floored, and thus linear up to 724B of memory used, after which it costs substantially more.

While defining the instruction set, we defined the memory-expansion for range function, $M$, thus:

\begin{equation}
M(s, f, l) \equiv \begin{cases}
s & \text{if} \quad l = 0 \\
\max(s, \ceil{ (f + l) \div 32 }) & \text{otherwise}
\end{cases}
\end{equation}

Another useful function is ``all but one 64th'' function~$L$ defined as:

\begin{equation}
L(n) \equiv n - \lfloor n / 64 \rfloor
\end{equation}

\subsection{Instruction Set}
\label{app:instruction-set}

As previously specified in Section \ref{sec:exe model}, these definitions take place in the final context there. In particular we assume $O$ is the EVM state-progression function and define the terms pertaining to the next cycle's state $(\st', \mst')$ such that:
\begin{equation}
O(\st, \mst, A, I) \equiv (\st', \mst', A', I) \quad \text{with exceptions, as noted}
\end{equation}

Here given are the various exceptions to the state transition rules given in Section \ref{sec:exe model} specified for each instruction, together with the additional instruction-specific definitions of $J$ and $\cost$. 
For each instruction, also specified is $\pushstack$, the additional items placed on the data stack, and $\popstack$, the items removed from data stack, as defined in Section \ref{sec:exe model}.
For subrountine instracutions, further specified is $\pushrstack$, the additional items pushed into the return stack, 
and $\poprstack$, the items removed from return stack, 
as defined in Section \ref{sec:exe model}.

\begin{tabu}{r l r r l} \savetabu{opcodes}
\toprule
\multicolumn{5}{c}{\textbf{0s: Stop and Arithmetic Operations}} \\
\multicolumn{5}{l}{All arithmetic is modulo $2^{256}$ unless otherwise noted. The zero-th power of zero $0^0$ is defined to be one.} \vspace{5pt} \\
\textbf{Value} & \textbf{Mnemonic} & $\popstack$ & $\pushstack$ & \textbf{Description} \vspace{5pt} \\
\linkdest{stop}{} 0x00 & $\op{STOP}$ & 0 & 0 & Halts execution. \\
\midrule
0x01 & $\op{ADD}$ & 2 & 1 & Addition operation. \\
&&&& $\mst'_{\mathbf{s}}[0] \equiv \mst_{\mathbf{s}}[0] + \mst_{\mathbf{s}}[1]$ \\
\midrule
0x02 & $\op{MUL}$ & 2 & 1 & Multiplication operation. \\
&&&& $\mst'_{\mathbf{s}}[0] \equiv \mst_{\mathbf{s}}[0] \times \mst_{\mathbf{s}}[1]$ \\
\midrule
0x03 & $\op{SUB}$ & 2 & 1 & Subtraction operation. \\
&&&& $\mst'_{\mathbf{s}}[0] \equiv \mst_{\mathbf{s}}[0] - \mst_{\mathbf{s}}[1]$ \\
\midrule
0x04 & $\op{DIV}$ & 2 & 1 & Integer division operation. \\
&&&& $\mst'_{\mathbf{s}}[0] \equiv \begin{cases}0 & \text{if} \quad \mst_{\mathbf{s}}[1] = 0\\ \lfloor\mst_{\mathbf{s}}[0] \div \mst_{\mathbf{s}}[1]\rfloor & \text{otherwise}\end{cases}$  \\
\midrule
0x05 & $\op{SDIV}$ & 2 & 1 & Signed integer division operation (truncated). \\
&&&& $\mst'_{\mathbf{s}}[0] \equiv \begin{cases}0 & \text{if} \> \mst_{\mathbf{s}}[1] = 0\\ -2^{255} & \text{if} \> \mst_{\mathbf{s}}[0] = -2^{255} \wedge \, \mst_{\mathbf{s}}[1] = -1\\ \mathbf{sgn} (\mst_{\mathbf{s}}[0] \div \mst_{\mathbf{s}}[1]) \lfloor |\mst_{\mathbf{s}}[0] \div \mst_{\mathbf{s}}[1]| \rfloor & \text{otherwise}\end{cases}$  \\
&&&& Where all values are treated as two's complement signed 256-bit integers. \\
&&&& Note the overflow semantic when $-2^{255}$ is negated.\\
\midrule
0x06 & $\op{MOD}$ & 2 & 1 & Modulo remainder operation. \\
&&&& $\mst'_{\mathbf{s}}[0] \equiv \begin{cases}0 & \text{if} \quad \mst_{\mathbf{s}}[1] = 0\\ \mst_{\mathbf{s}}[0] \bmod \mst_{\mathbf{s}}[1] & \text{otherwise}\end{cases}$  \\
\midrule
0x07 & $\op{SMOD}$ & 2 & 1 & Signed modulo remainder operation. \\
&&&& $\mst'_{\mathbf{s}}[0] \equiv \begin{cases}0 & \text{if} \quad \mst_{\mathbf{s}}[1] = 0\\ \mathbf{sgn} (\mst_{\mathbf{s}}[0]) (|\mst_{\mathbf{s}}[0]| \bmod |\mst_{\mathbf{s}}[1]|) & \text{otherwise}\end{cases}$  \\
&&&& Where all values are treated as two's complement signed 256-bit integers. \\
\midrule
0x08 & $\op{ADDMOD}$ & 3 & 1 & Modulo addition operation. \\
&&&& $\mst'_{\mathbf{s}}[0] \equiv \begin{cases}0 & \text{if} \quad \mst_{\mathbf{s}}[2] = 0\\ (\mst_{\mathbf{s}}[0] + \mst_{\mathbf{s}}[1]) \mod \mst_{\mathbf{s}}[2] & \text{otherwise}\end{cases}$  \\
&&&& All intermediate calculations of this operation are not subject to the $2^{256}$ \\
&&&& modulo. \\
\midrule
0x09 & $\op{MULMOD}$ & 3 & 1 & Modulo multiplication operation. \\
&&&& $\mst'_{\mathbf{s}}[0] \equiv \begin{cases}0 & \text{if} \quad \mst_{\mathbf{s}}[2] = 0\\ (\mst_{\mathbf{s}}[0] \times \mst_{\mathbf{s}}[1]) \mod \mst_{\mathbf{s}}[2] & \text{otherwise}\end{cases}$  \\
&&&& All intermediate calculations of this operation are not subject to the $2^{256}$ \\
&&&& modulo. \\
\midrule
0x0a & $\op{EXP}$ & 2 & 1 & Exponential operation. \\
&&&& $\mst'_{\mathbf{s}}[0] \equiv \mst_{\mathbf{s}}[0] ^ {\mst_{\mathbf{s}}[1] }$ \\
\midrule
0x0b & $\op{SIGNEXTEND}$ & 2 & 1 & Extend length of two's complement signed integer. \\
&&&& $ \forall i \in [0..255]: \mst'_{\mathbf{s}}[0]_{\mathrm{i}} \equiv \begin{cases} \mst_{\mathbf{s}}[1]_{\mathrm{t}} &\text{if} \quad i \leqslant t \quad \text{where} \; t = 256 - 8(\mst_{\mathbf{s}}[0] + 1) \\ \mst_{\mathbf{s}}[1]_{\mathrm{i}} &\text{otherwise} \end{cases}$ \\
\multicolumn{5}{l}{$\mst_{\mathbf{s}}[x]_{\mathrm{i}}$ gives the $i$th bit (counting from zero) of $\mst_{\mathbf{s}}[x]$} \vspace{5pt} \\
\midrule
\end{tabu}

\begin{tabu}{\usetabu{opcodes}}
\toprule
\multicolumn{5}{c}{\textbf{10s: Comparison \& Bitwise Logic Operations}} \\
\textbf{Value} & \textbf{Mnemonic} & $\popstack$ & $\pushstack$ & \textbf{Description} \vspace{5pt} \\
0x10 & $\op{LT}$ & 2 & 1 & Less-than comparison. \\
&&&& $\mst'_{\mathbf{s}}[0] \equiv \begin{cases} 1 & \text{if} \quad \mst_{\mathbf{s}}[0] < \mst_{\mathbf{s}}[1] \\ 0 & \text{otherwise} \end{cases}$ \\
\midrule
0x11 & $\op{GT}$ & 2 & 1 & Greater-than comparison. \\
&&&& $\mst'_{\mathbf{s}}[0] \equiv \begin{cases} 1 & \text{if} \quad \mst_{\mathbf{s}}[0] > \mst_{\mathbf{s}}[1] \\ 0 & \text{otherwise} \end{cases}$ \\
\midrule
0x12 & $\op{SLT}$ & 2 & 1 & Signed less-than comparison. \\
&&&& $\mst'_{\mathbf{s}}[0] \equiv \begin{cases} 1 & \text{if} \quad \mst_{\mathbf{s}}[0] < \mst_{\mathbf{s}}[1] \\ 0 & \text{otherwise} \end{cases}$ \\
&&&& Where all values are treated as two's complement signed 256-bit integers. \\
\midrule
0x13 & $\op{SGT}$ & 2 & 1 & Signed greater-than comparison. \\
&&&& $\mst'_{\mathbf{s}}[0] \equiv \begin{cases} 1 & \text{if} \quad \mst_{\mathbf{s}}[0] > \mst_{\mathbf{s}}[1] \\ 0 & \text{otherwise} \end{cases}$ \\
&&&& Where all values are treated as two's complement signed 256-bit integers. \\
\midrule
0x14 & $\op{EQ}$ & 2 & 1 & Equality comparison. \\
&&&& $\mst'_{\mathbf{s}}[0] \equiv \begin{cases} 1 & \text{if} \quad \mst_{\mathbf{s}}[0] = \mst_{\mathbf{s}}[1] \\ 0 & \text{otherwise} \end{cases}$ \\
\midrule
0x15 & $\op{ISZERO}$ & 1 & 1 & Simple not operator. \\
&&&& $\mst'_{\mathbf{s}}[0] \equiv \begin{cases} 1 & \text{if} \quad \mst_{\mathbf{s}}[0] = 0 \\ 0 & \text{otherwise} \end{cases}$ \\
\midrule
0x16 & $\op{AND}$ & 2 & 1 & Bitwise AND operation. \\
&&&& $\forall i \in [0..255]: \mst'_{\mathbf{s}}[0]_{\mathrm{i}} \equiv \mst_{\mathbf{s}}[0]_{\mathrm{i}} \wedge \mst_{\mathbf{s}}[1]_{\mathrm{i}}$ \\
\midrule
0x17 & $\op{OR}$ & 2 & 1 & Bitwise OR operation. \\
&&&& $\forall i \in [0..255]: \mst'_{\mathbf{s}}[0]_{\mathrm{i}} \equiv \mst_{\mathbf{s}}[0]_{\mathrm{i}} \vee \mst_{\mathbf{s}}[1]_{\mathrm{i}}$ \\
\midrule
0x18 & $\op{XOR}$ & 2 & 1 & Bitwise XOR operation. \\
&&&& $\forall i \in [0..255]: \mst'_{\mathbf{s}}[0]_{\mathrm{i}} \equiv \mst_{\mathbf{s}}[0]_{\mathrm{i}} \oplus \mst_{\mathbf{s}}[1]_{\mathrm{i}}$ \\
\midrule
0x19 & $\op{NOT}$ & 1 & 1 & Bitwise NOT operation. \\
&&&& $\forall i \in [0..255]: \mst'_{\mathbf{s}}[0]_{\mathrm{i}} \equiv \begin{cases} 1 & \text{if} \quad \mst_{\mathbf{s}}[0]_{\mathrm{i}} = 0 \\ 0 & \text{otherwise} \end{cases}$ \\
\midrule
0x1a & $\op{BYTE}$ & 2 & 1 & Retrieve single byte from word. \\
&&&& $\forall i \in [0..255]: \mst'_{\mathbf{s}}[0]_{\mathrm{i}} \equiv \begin{cases} \mst_{\mathbf{s}}[1]_{(i + 8\mst_{\mathbf{s}}[0])} & \text{if} \quad i < 8 \wedge \mst_{\mathbf{s}}[0] < 32 \\ 0 & \text{otherwise} \end{cases} $\\
&&&& For the Nth byte, we count from the left (i.e. N=0 would be the most significant\\
&&&& in big endian). \\
\midrule
0x1b & $\op{SHL}$ & 2 & 1 & Left shift operation. \\
&&&& $\mst'_{\mathbf{s}}[0] \equiv (\mst_{\mathbf{s}}[1] \times 2^{\mst_{\mathbf{s}}[0]}) \mod 2^{256}$ \\
\midrule
0x1c & $\op{SHR}$ & 2 & 1 & Logical right shift operation. \\
&&&& $\mst'_{\mathbf{s}}[0] \equiv \lfloor \mst_{\mathbf{s}}[1] \div 2^{\mst_{\mathbf{s}}[0]} \rfloor$ \\
\midrule
0x1d & $\op{SAR}$ & 2 & 1 & Arithmetic (signed) right shift operation. \\
&&&& $\mst'_{\mathbf{s}}[0] \equiv \lfloor \mst_{\mathbf{s}}[1] \div 2^{\mst_{\mathbf{s}}[0]} \rfloor$ \\
&&&& Where $\mst'_{\mathbf{s}}[0]$ and $\mst_{\mathbf{s}}[1]$ are treated as two's complement signed 256-bit integers, \\
&&&& while $\mst_{\mathbf{s}}[0]$ is treated as unsigned. \\
\bottomrule
\end{tabu}

\begin{tabu}{\usetabu{opcodes}}
\toprule
\multicolumn{5}{c}{\textbf{20s: SHA3}} \vspace{5pt} \\
\textbf{Value} & \textbf{Mnemonic} & $\popstack$ & $\pushstack$ & \textbf{Description} \vspace{5pt} \\
0x20 & $\op{SHA3}$ & 2 & 1 & Compute Keccak-256 hash. \\
&&&& $\mst'_{\mathbf{s}}[0] \equiv \mathtt{KEC}(\mst_{\mathbf{m}}[ \mst_{\mathbf{s}}[0] \dots (\mst_{\mathbf{s}}[0] + \mst_{\mathbf{s}}[1] - 1) ])$ \\
&&&& $\mst'_{\mathrm{i}} \equiv M(\mst_{\mathrm{i}}, \mst_{\mathbf{s}}[0], \mst_{\mathbf{s}}[1])$ \\
\bottomrule
\end{tabu}

\begin{tabu}{\usetabu{opcodes}}
\toprule
\multicolumn{5}{c}{\textbf{30s: Environmental Information}} \vspace{5pt} \\
\textbf{Value} & \textbf{Mnemonic} & $\popstack$ & $\pushstack$ & \textbf{Description} \vspace{5pt} \\
0x30 & $\op{ADDRESS}$ & 0 & 1 & Get address of currently executing account. \\
&&&& $\mst'_{\mathbf{s}}[0] \equiv I_{\mathrm{a}}$ \\
\midrule
0x31 & $\op{BALANCE}$ & 1 & 1 & Get balance of the given account. \\
&&&& $\mst'_{\mathbf{s}}[0] \equiv \begin{cases}\st[\mst_{\mathbf{s}}[0]]_{\mathrm{b}}& \text{if} \quad \st[\mst_{\mathbf{s}}[0] \mod 2^{160}] \neq \varnothing\\0&\text{otherwise}\end{cases}$ \\
\midrule
0x32 & $\op{ORIGIN}$ & 0 & 1 & Get execution origination address. \\
&&&& $\mst'_{\mathbf{s}}[0] \equiv I_{\mathrm{o}}$ \\
&&&& This is the sender of original transaction; it is never an account with\\
&&&& non-empty associated code. \\
\midrule
0x33 & $\op{CALLER}$ & 0 & 1 & Get caller address. \\
&&&& $\mst'_{\mathbf{s}}[0] \equiv I_{\mathrm{s}}$ \\
&&&& This is the address of the account that is directly responsible for\\
&&&& this execution. \\
\midrule
0x34 & $\op{CALLVALUE}$ & 0 & 1 & Get deposited value by the instruction/transaction responsible for\\
&&&& this execution. \\
&&&& $\mst'_{\mathbf{s}}[0] \equiv I_{\mathrm{v}}$ \\
\midrule
0x35 & $\op{CALLDATALOAD}$ & 1 & 1 & Get input data of current environment. \\
&&&& $\mst'_{\mathbf{s}}[0] \equiv I_{\mathbf{d}}[ \mst_{\mathbf{s}}[0] \dots (\mst_{\mathbf{s}}[0] + 31) ] \quad \text{with} \quad I_{\mathbf{d}}[x] = 0 \quad \text{if} \quad x \geqslant \lVert I_{\mathbf{d}} \rVert$ \\
&&&& This pertains to the input data passed with the message call\\
&&&& instruction or transaction. \\
\midrule
0x36 & $\op{CALLDATASIZE}$ & 0 & 1 & Get size of input data in current\\
&&&& environment. \\
&&&& $\mst'_{\mathbf{s}}[0] \equiv \lVert I_{\mathbf{d}} \rVert$ \\
&&&& This pertains to the input data passed with the message call\\
&&&& instruction or transaction. \\
\midrule
0x37 & $\op{CALLDATACOPY}$ & 3 & 0 & Copy input data in current environment to memory. \\
&&&& $\forall i \in \{ 0 \dots \mst_{\mathbf{s}}[2] - 1\}: \mst'_{\mathbf{m}}[\mst_{\mathbf{s}}[0] + i ] \equiv
\begin{cases} I_{\mathbf{d}}[\mst_{\mathbf{s}}[1] + i] & \text{if} \quad \mst_{\mathbf{s}}[1] + i < \lVert I_{\mathbf{d}} \rVert \\ 0 & \text{otherwise} \end{cases}$\\
&&&& The additions in $\mst_{\mathbf{s}}[1] + i$ are not subject to the $2^{256}$ modulo. \\
&&&& $\mst'_{\mathrm{i}} \equiv M(\mst_{\mathrm{i}}, \mst_{\mathbf{s}}[0], \mst_{\mathbf{s}}[2])$ \\
&&&& This pertains to the input data passed with the message call instruction\\
&&&& or transaction. \\
\midrule
0x38 & $\op{CODESIZE}$ & 0 & 1 & Get size of code running in current environment. \\
&&&& $\mst'_{\mathbf{s}}[0] \equiv \lVert I_{\mathbf{b}} \rVert$ \\
\midrule
0x39 & $\op{CODECOPY}$ & 3 & 0 & Copy code running in current environment to memory. \\
&&&& $\forall i \in \{ 0 \dots \mst_{\mathbf{s}}[2] - 1\}: \mst'_{\mathbf{m}}[\mst_{\mathbf{s}}[0] + i ] \equiv
\begin{cases} I_{\mathbf{b}}[\mst_{\mathbf{s}}[1] + i] & \text{if} \quad \mst_{\mathbf{s}}[1] + i < \lVert I_{\mathbf{b}} \rVert \\ \op{STOP} & \text{otherwise} \end{cases}$\\
&&&& $\mst'_{\mathrm{i}} \equiv M(\mst_{\mathrm{i}}, \mst_{\mathbf{s}}[0], \mst_{\mathbf{s}}[2])$ \\
&&&& The additions in $\mst_{\mathbf{s}}[1] + i$ are not subject to the $2^{256}$ modulo. \\
\midrule
0x3a & $\op{GASPRICE}$ & 0 & 1 & Get price of gas in current environment. \\
&&&& $\mst'_{\mathbf{s}}[0] \equiv I_{\mathrm{p}}$ \\
&&&& This is gas price specified by the originating transaction.\\
\midrule
0x3b & $\op{EXTCODESIZE}$ & 1 & 1 & Get size of an account's code. \\
&&&& $\mst'_{\mathbf{s}}[0] \equiv \lVert \mathbf{b} \rVert$ \\
&&&& where $\mathtt{KEC}(\mathbf{b}) \equiv \st[\mst_{\mathbf{s}}[0] \mod 2^{160}]_{\mathrm{c}}$ \\
\end{tabu}

\begin{tabu}{\usetabu{opcodes}}
\midrule
0x3c & $\op{EXTCODECOPY}$ & 4 & 0 & Copy an account's code to memory. \\
&&&& $\forall i \in \{ 0 \dots \mst_{\mathbf{s}}[3] - 1\}: \mst'_{\mathbf{m}}[\mst_{\mathbf{s}}[1] + i ] \equiv
\begin{cases} \mathbf{b}[\mst_{\mathbf{s}}[2] + i] & \text{if} \quad \mst_{\mathbf{s}}[2] + i < \lVert \mathbf{b} \rVert \\ \op{STOP} & \text{otherwise} \end{cases}$\\
&&&& where $\mathtt{KEC}(\mathbf{b}) \equiv \st[\mst_{\mathbf{s}}[0] \mod 2^{160}]_{\mathrm{c}}$ \\
&&&& $\mst'_{\mathrm{i}} \equiv M(\mst_{\mathrm{i}}, \mst_{\mathbf{s}}[1], \mst_{\mathbf{s}}[3])$ \\
&&&& The additions in $\mst_{\mathbf{s}}[2] + i$ are not subject to the $2^{256}$ modulo. \\
\midrule
0x3d & $\op{RETURNDATASIZE}$ & 0 & 1 & Get size of output data from the previous call from the current\\
&&&& environment. \\
&&&& $\mst'_{\mathbf{s}}[0] \equiv \lVert \mst_{\mathbf{o}} \rVert$ \\
\midrule
0x3e & $\op{RETURNDATACOPY}$ & 3 & 0 & Copy output data from the previous call to memory. \\
&&&& $\forall i \in \{ 0 \dots \mst_{\mathbf{s}}[2] - 1\}: \mst'_{\mathbf{m}}[\mst_{\mathbf{s}}[0] + i ] \equiv
\begin{cases} \mst_{\mathbf{o}}[\mst_{\mathbf{s}}[1] + i] & \text{if} \quad \mst_{\mathbf{s}}[1] + i < \lVert \mst_{\mathbf{o}} \rVert \\ 0 & \text{otherwise} \end{cases}$\\
&&&& The additions in $\mst_{\mathbf{s}}[1] + i$ are not subject to the $2^{256}$ modulo. \\
&&&& $\mst'_{\mathrm{i}} \equiv M(\mst_{\mathrm{i}}, \mst_{\mathbf{s}}[0], \mst_{\mathbf{s}}[2])$ \\
\midrule
\linkdest{extcodehash}{}0x3f & $\op{EXTCODEHASH}$ & 1 & 1 & Get hash of an account's code. \\
&&&& $\mst'_{\mathbf{s}}[0] \equiv
\begin{cases} 0 & \text{if} \quad \mathtt{DEAD}(\st, \mst_{\mathbf{s}}[0] \mod 2^{160}) \\ \st[\mst_{\mathbf{s}}[0] \mod 2^{160}]_{\mathrm{c}} & \text{otherwise} \end{cases}$ \\
\bottomrule
\end{tabu}

\begin{tabu}{\usetabu{opcodes}}
\toprule
\multicolumn{5}{c}{\textbf{40s: Block Information}} \vspace{5pt} \\
\textbf{Value} & \textbf{Mnemonic} & $\popstack$ & $\pushstack$ & \textbf{Description} \vspace{5pt} \\
\linkdest{BLOCKHASH}{}0x40 & $\op{BLOCKHASH}$ & 1 & 1 & Get the hash of the last block in block order. \\
&&&&
In {\name}, we only maintain the block hash of the previous block.\\
&&&&
When querying other block numbers, the returned result is always $0$.\\
\linkdest{blockhash}{}&&&& $\mst'_{\mathbf{s}}[0] \equiv 
\begin{cases}
	\kec(I_{{\head}_{\mathbf{L}}}[-1]) & \text{if} \quad \mst_{\mathbf{s}}[0] = |I_{{\head}_{\mathbf{L}}}|-1 \\
	0 & \text{otherwise}
\end{cases}$ \\
\midrule
0x41 & $\op{COINBASE}$ & 0 & 1 & Get the block's beneficiary address. \\
&&&& $\mst'_{\mathbf{s}}[0] \equiv I_{{\head}_{\mathrm{c}}}$ \\
\midrule
0x42 & $\op{TIMESTAMP}$ & 0 & 1 & Get the block's timestamp. \\
&&&& $\mst'_{\mathbf{s}}[0] \equiv I_{{\head}_{\mathrm{s}}}$ \\
\midrule
0x43 & $\op{NUMBER}$ & 0 & 1 & Get the block's index in total order. (The index of genesis block is 0.) \\
&&&& $\mst'_{\mathbf{s}}[0] \equiv |I_{{\head}_{\mathbf{L}}}|$ \\
\midrule
0x44 & $\op{DIFFICULTY}$ & 0 & 1 & Get the block's difficulty. \\
&&&& $\mst'_{\mathbf{s}}[0] \equiv I_{{\head}_{\mathrm{d}}}$ \\
\midrule
0x45 & $\op{GASLIMIT}$ & 0 & 1 & Get the block's gas limit. \\
&&&& $\mst'_{\mathbf{s}}[0] \equiv I_{{\head}_{\ell}}$ \\
\midrule
\linkdest{CHAINID}{} 0x46 & $\op{CHAINID}$ & 0 & 1 & Get the chain ID. \\
&&&& $\mst'_{\mathbf{s}}[0] \equiv \chainid$ \\
\midrule
\linkdest{SELFBALANCE}{} 0x47 & $\op{SELFBALANCE}$ & 0 & 1 & Get balance of the currently executing account. \\
&&&& $\mst'_{\mathbf{s}}[0] \equiv  \begin{cases}\st[I_{\mathrm{a}}]_{\mathrm{b}}& \text{if} \quad \st[I_{\mathrm{a}} \mod 2^{160}] \neq \varnothing\\0&\text{otherwise}\end{cases}$ \\
\bottomrule
\end{tabu}

\begin{tabu}{\usetabu{opcodes}}
\toprule
\multicolumn{5}{c}{\textbf{50s: Stack, Memory, Storage and Flow Operations}} \vspace{5pt} \\
\textbf{Value} & \textbf{Mnemonic} & $\popstack$ & $\pushstack$ & \textbf{Description} \vspace{5pt} \\
0x50 & $\op{POP}$ & 1 & 0 & Remove item from stack. \\
\midrule
0x51 & $\op{MLOAD}$ & 1 & 1 & Load word from memory. \\
&&&& $\mst'_{\mathbf{s}}[0] \equiv \mst_{\mathbf{m}}[\mst_{\mathbf{s}}[0] \dots (\mst_{\mathbf{s}}[0] + 31) ]$ \\
&&&& $\mst'_{\mathrm{i}} \equiv \max(\mst_{\mathrm{i}}, \ceil{ (\mst_{\mathbf{s}}[0] + 32) \div 32 })$ \\
&&&& The addition in the calculation of $\mst'_{\mathrm{i}}$ is not subject to the $2^{256}$ modulo. \\
\midrule
0x52 & $\op{MSTORE}$ & 2 & 0 & Save word to memory. \\
&&&& $\mst'_{\mathbf{m}}[ \mst_{\mathbf{s}}[0] \dots (\mst_{\mathbf{s}}[0] + 31) ] \equiv \mst_{\mathbf{s}}[1]$ \\
&&&& $\mst'_{\mathrm{i}} \equiv \max(\mst_{\mathrm{i}}, \ceil{ (\mst_{\mathbf{s}}[0] + 32) \div 32 })$ \\
&&&& The addition in the calculation of $\mst'_{\mathrm{i}}$ is not subject to the $2^{256}$ modulo. \\
\midrule
0x53 & $\op{MSTORE8}$ & 2 & 0 & Save byte to memory. \\
&&&& $\mst'_{\mathbf{m}}[ \mst_{\mathbf{s}}[0] ] \equiv (\mst_{\mathbf{s}}[1] \bmod 256) $ \\
&&&& $\mst'_{\mathrm{i}} \equiv \max(\mst_{\mathrm{i}}, \ceil{ (\mst_{\mathbf{s}}[0] + 1) \div 32 })$ \\
&&&& The addition in the calculation of $\mst'_{\mathrm{i}}$ is not subject to the $2^{256}$ modulo. \\
\midrule
0x54 & $\op{SLOAD}$ & 1 & 1 & Load word from storage. \\
&&&& $\mst'_{\mathbf{s}}[0] \equiv \st[I_{\mathrm{a}}]_{\mathbf{s}}[\mst_{\mathbf{s}}[0]]_v$ \\
\midrule
\linkdest{SSTORE}{}0x55 & $\op{SSTORE}$ & 2 & 0 & Save word and its owner to storage \\
% &&&& $\st'[I_{\mathrm{a}}]_{\mathbf{s}}[ \mst_{\mathbf{s}}[0] ] \equiv (\mst_{\mathbf{s}}[1],I_i) $ \\
&&&&$(\st',A^*)\eqdef \Phi(\st,I_a,\mst_{\sf s}[0],\mst_{\sf s}[1],I_i)$\\
&&&&$A'\eqdef A\Cup A^*$\\
&&&&where $\Phi$ is defined in section~\ref{sec:storage_maintain}\\
\midrule
\linkdest{JUMP}{}0x56 & $\op{JUMP}$ & 1 & 0 & Alter the program counter. \\
&&&& $J_{\op{JUMP}}(\mst) \equiv \mst_{\mathbf{s}}[0] $ \\
&&&& This has the effect of writing said value to $\mst_{\mathrm{pc}}$. See (\ref{eq:mu_pc}) in Section \ref{sec:exe model}.\\
\midrule
\linkdest{JUMPI}{}0x57 & $\op{JUMPI}$ & 2 & 0 & Conditionally alter the program counter. \\
&&&& $J_{\op{JUMPI}}(\mst) \equiv \begin{cases} \mst_{\mathbf{s}}[0] & \text{if} \quad \mst_{\mathbf{s}}[1] \neq 0 \\ \mst_{\mathrm{pc}} + 1 & \text{otherwise} \end{cases} $ \\
&&&& This has the effect of writing said value to $\mst_{\mathrm{pc}}$. See (\ref{eq:mu_pc}) in Section \ref{sec:exe model}. \\
\midrule
0x58 & $\op{PC}$ & 0 & 1 & Get the value of the program counter \textit{prior} to the increment \\
&&&&  corresponding to this instruction. \\
&&&& $\mst'_{\mathbf{s}}[0] \equiv \mst_{\mathrm{pc}}$ \\
\midrule
0x59 & $\op{MSIZE}$ & 0 & 1 & Get the size of active memory in bytes. \\
&&&& $\mst'_{\mathbf{s}}[0] \equiv 32\mst_{i}$ \\
\midrule
0x5a & $\op{GAS}$ & 0 & 1 & Get the amount of available gas, including the corresponding reduction \\
&&&& for the cost of this instruction. \\
&&&& $\mst'_{\mathbf{s}}[0] \equiv \mst_{g}$ \\
\midrule
0x5b & $\op{JUMPDEST}$ & 0 & 0 & Mark a valid destination for jumps. \\
&&&& This operation has no effect on machine state during execution. \\
\midrule
\end{tabu}

\begin{tabu}{r l c c c c l}%{\usetabu{opcodes}}
\midrule
\multicolumn{7}{c}{\textbf{50s: Stack, Memory, Storage and Flow Operations -- Subroutine Operations}} \vspace{5pt} \\
{\bf Note 1:} & \multicolumn{6}{l}{Here we list columns of $\poprstack$ and $\pushrstack$ because $\op{JUMPSUB}$ and $\op{RETURNSUB}$ may change the return stack $\mst_{\vec{r}}$. }\\
& \multicolumn{6}{l}{However, $\mst_{\vec{r}}$ is only alterable by these two instructions, and hence there is no need to validate popped values.
}\\
{\bf Note 2:} & \multicolumn{6}{l}{ The actual state of the return stack is neither observable by \cvm code nor consensus-critical to the protocol.}\\
&\multicolumn{6}{l}{ Thus, a node implementor may code $\op{JUMPSUB}$ to unobservably push $\mathrm{pc}$ on the return stack rather than $\mathrm{pc} + 1$,}\\
&\multicolumn{6}{l}{which is allowed so long as the next $\op{RETURNSUB}$ would observably return control to the $\mathrm{pc} + 1$ location.
}
\medskip\\
\textbf{Value} & \textbf{Mnemonic} & $\popstack$ & $\pushstack$ & $\poprstack$ & $\pushrstack$ & \textbf{Description} \vspace{5pt} \\
\linkdest{BEGINSUB}{}0x5c & $\op{BEGINSUB}$ & 0 & 0 & 0 & 0 & Marks the entry point to a subroutine. \\
&&&&&& Attempted execution of a $\op{BEGINSUB}$ causes an abort: \\
&&&&&& terminate execution with an OOG (\emph{Out Of Gas}) exception. \\
\midrule
\linkdest{RETURNSUB}{}0x5d & $\op{RETURNSUB}$ & 0 & 0 & 1 & 0 & Returns from a subroutine. \\
&&&&&&   If  $\lVert \mst_\mathbf{r} \rVert = 0$, then abort:\\
&&&&&&   terminate execution with a \emph{Return Stack Underflow} exception. \\
&&&&&&   Otherwise $J_{\op{RETURNSUB}}(\mst) \eqdef \mst_{\mathbf{r}}[0] $\\
&&&&&&   This has the effect of writing said value to $\mst_{\mathrm{pc}}$. See (\ref{eq:mu_pc}) in Section \ref{sec:exe model}.\\
\midrule
\linkdest{JUMPSUB}{}0x5e & $\op{JUMPSUB}$ & 1 & 0 & 0 & 1 & Jumps to a defined $\op{BEGINSUB}$ subroutine and transfers control to it. \\
&&&&&&  If $\lVert \mst_\mathbf{r} \rVert = 1023$, then abort: \\
&&&&&&  terminate execution with an \emph{Out Of Return Stack} exception.\\
&&&&&&  Else if $I_{\vec{b}}\left[\mst_{\mathbf{s}}[0]\right] \ne \op{BEGINSUB}$ then abort: \\
&&&&&&  terminate execution with a \emph{Bad Jump Destination} exception.\\
&&&&&&  Otherwise:\\
&&&&&&  $\mst'_{\vec{r}}[0] \eqdef \mst_{\mathrm{pc}}+1$ \\
% &&&&&&  $\mst'_{\mathrm{pc}} \eqdef \mst_{\mathbf{s}}[0]+1$ \\
&&&&&&  $J_{\op{JUMPSUB}}(\mst) \eqdef \mst_{\mathbf{s}}[0]+1 $  \\
&&&&&&   This has the effect of writing said value to $\mst_{\mathrm{pc}}$. See (\ref{eq:mu_pc}) in Section \ref{sec:exe model}.\\
&&&&&&  In case $\mst_{\mathbf{s}}[0]+1 \ge \lVert I_{\vec{b}} \rVert$, i.e. the resulting $\mathrm{pc}$ is beyond the last instruction,  \\
&&&&&&  then the opcode is implicitly a $\op{STOP}$, which is not an error.\\
\bottomrule
\end{tabu}

\begin{tabu}{\usetabu{opcodes}}
\toprule
\multicolumn{5}{c}{\textbf{60s \& 70s: Push Operations}} \vspace{5pt} \\
\textbf{Value} & \textbf{Mnemonic} & $\popstack$ & $\pushstack$ & \textbf{Description} \vspace{5pt} \\
0x60 & $\op{PUSH1}$ & 0 & 1 & Place 1 byte item on stack. \\
&&&& $\mst'_{\mathbf{s}}[0] \equiv c(\mst_{\mathrm{pc}} + 1)$ \\
&&&& $\text{where} \quad c(x) \equiv \begin{cases} I_{\mathbf{b}}[x] & \text{if} \quad x < \lVert I_{\mathbf{b}} \rVert \\ 0 & \text{otherwise} \end{cases}$ \\
&&&& The bytes are read in line from the program code's bytes array. \\
&&&& The function $c$ ensures the bytes default to zero if they extend past the limits.\\
&&&& The byte is right-aligned (takes the lowest significant place in big endian). \\
\midrule
0x61 & $\op{PUSH2}$ & 0 & 1 & Place 2-byte item on stack. \\
&&&& $\mst'_{\mathbf{s}}[0] \equiv \boldsymbol{c}\big( (\mst_{\mathrm{pc}} + 1) \dots (\mst_{\mathrm{pc}} + 2) \big)$ \\
&&&& with $\boldsymbol{c}(\boldsymbol{x}) \equiv (c(\boldsymbol{x}_0), ..., c(\boldsymbol{x}_{\lVert x \rVert -1})) $ with $c$ as defined as above. \\
&&&& The bytes are right-aligned (takes the lowest significant place in big endian). \\
\midrule
\multicolumn{1}{c}{\vdots} & \multicolumn{1}{c}{\vdots} & \vdots & \vdots & \multicolumn{1}{c}{\vdots} \\
\midrule
0x7f & $\op{PUSH32}$ & 0 & 1 & Place 32-byte (full word) item on stack. \\
&&&& $\mst'_{\mathbf{s}}[0] \equiv \boldsymbol{c}\big((\mst_{\mathrm{pc}} + 1) \dots (\mst_{\mathrm{pc}} + 32) \big)$ \\
&&&& where $\boldsymbol{c}$ is defined as above. \\
&&&& The bytes are right-aligned (takes the lowest significant place in big endian). \\
\bottomrule
\end{tabu}

\begin{tabu}{\usetabu{opcodes}}
\toprule
\multicolumn{5}{c}{\textbf{80s: Duplication Operations}} \vspace{5pt} \\
\textbf{Value} & \textbf{Mnemonic} & $\popstack$ & $\pushstack$ & \textbf{Description} \vspace{5pt} \\
0x80 & $\op{DUP1}$ & 1 & 2 & Duplicate 1st stack item. \\
&&&& $\mst'_{\mathbf{s}}[0] \equiv \mst_{\mathbf{s}}[0]$ \\
\midrule
0x81 & $\op{DUP2}$ & 2 & 3 & Duplicate 2nd stack item. \\
&&&& $\mst'_{\mathbf{s}}[0] \equiv \mst_{\mathbf{s}}[1]$ \\
\midrule
\multicolumn{1}{c}{\vdots} & \multicolumn{1}{c}{\vdots} & \vdots & \vdots & \multicolumn{1}{c}{\vdots} \\
\midrule
0x8f & $\op{DUP16}$ & 16 & 17 & Duplicate 16th stack item. \\
&&&& $\mst'_{\mathbf{s}}[0] \equiv \mst_{\mathbf{s}}[15]$ \\
\bottomrule
\end{tabu}

\begin{tabu}{\usetabu{opcodes}}
\toprule
\multicolumn{5}{c}{\textbf{90s: Exchange Operations}} \vspace{5pt} \\
\textbf{Value} & \textbf{Mnemonic} & $\popstack$ & $\pushstack$ & \textbf{Description} \vspace{5pt} \\
0x90 & $\op{SWAP1}$ & 2 & 2 & Exchange 1st and 2nd stack items. \\
&&&& $\mst'_{\mathbf{s}}[0] \equiv \mst_{\mathbf{s}}[1]$ \\
&&&& $\mst'_{\mathbf{s}}[1] \equiv \mst_{\mathbf{s}}[0]$ \\
\midrule
0x91 & $\op{SWAP2}$ & 3 & 3 & Exchange 1st and 3rd stack items. \\
&&&& $\mst'_{\mathbf{s}}[0] \equiv \mst_{\mathbf{s}}[2]$ \\
&&&& $\mst'_{\mathbf{s}}[2] \equiv \mst_{\mathbf{s}}[0]$ \\
\midrule
\multicolumn{1}{c}{\vdots} & \multicolumn{1}{c}{\vdots} & \vdots & \vdots & \multicolumn{1}{c}{\vdots} \\
\midrule
0x9f & $\op{SWAP16}$ & 17 & 17 & Exchange 1st and 17th stack items. \\
&&&& $\mst'_{\mathbf{s}}[0] \equiv \mst_{\mathbf{s}}[16]$ \\
&&&& $\mst'_{\mathbf{s}}[16] \equiv \mst_{\mathbf{s}}[0]$ \\
\bottomrule
\end{tabu}

\begin{tabu}{\usetabu{opcodes}}
\toprule
\multicolumn{5}{c}{\textbf{a0s: Logging Operations}} \vspace{5pt} \\
\multicolumn{5}{l}{For all logging operations, the state change is to append an additional log entry on to the substate's log series:}\\
\multicolumn{5}{l}{\linkdest{A l}{}$A'_{\mathbf{l}} \equiv A_{\mathbf{l}} \cdot (I_{\mathrm{a}}, \mathbf{t}, \mst_{\mathbf{m}}[ \mst_{\mathbf{s}}[0] \dots (\mst_{\mathbf{s}}[0] + \mst_{\mathbf{s}}[1] - 1) ])$}\\
\multicolumn{5}{l}{and to update the memory consumption counter:}\\
\multicolumn{5}{l}{$\mst'_{\mathrm{i}} \equiv M(\mst_{\mathrm{i}}, \mst_{\mathbf{s}}[0], \mst_{\mathbf{s}}[1])$}\\
\multicolumn{5}{l}{The entry's topic series, $\mathbf{t}$, differs accordingly:}\vspace{5pt} \smallskip\\
\textbf{Value} & \textbf{Mnemonic} & $\popstack$ & $\pushstack$ & \textbf{Description} \vspace{5pt} \\
0xa0 & $\op{LOG0}$ & 2 & 0 & Append log record with no topics. \\
&&&& $\mathbf{t} \equiv \emptystring$ \\
\midrule
0xa1 & $\op{LOG1}$ & 3 & 0 & Append log record with one topic. \\
&&&& $\mathbf{t} \equiv (\mst_{\mathbf{s}}[2])$ \\
\midrule
\multicolumn{1}{c}{\vdots} & \multicolumn{1}{c}{\vdots} & \vdots & \vdots & \multicolumn{1}{c}{\vdots} \\
\midrule
0xa4 & $\op{LOG4}$ & 6 & 0 & Append log record with four topics. \\
&&&& $\mathbf{t} \equiv (\mst_{\mathbf{s}}[2], \mst_{\mathbf{s}}[3], \mst_{\mathbf{s}}[4], \mst_{\mathbf{s}}[5])$ \\
\bottomrule
\end{tabu}


\begin{tabu}{\usetabu{opcodes}}
\toprule
\multicolumn{5}{c}{\textbf{f0s: System operations}} \vspace{5pt} \\
\textbf{Value} & \textbf{Mnemonic} & $\popstack$ & $\pushstack$ & \textbf{Description} \vspace{5pt} \\
\linkdest{create}{} 0xf0 & $\op{CREATE}$ & 3 & 1 & Create a new account with associated code. \\
&&&& $\mathbf{i} \equiv \mst_{\mathbf{m}}[ \mst_{\mathbf{s}}[1] \dots (\mst_{\mathbf{s}}[1] + \mst_{\mathbf{s}}[2] - 1) ]$ \\
&&&& $\zeta \equiv \varnothing$ \\
&&&& $(\st', \mst'_{\mathrm{g}}, A^+, \mathbf{o}) \equiv \begin{cases}\Lambda(\st^*, I_{\mathrm{a}}, I_{\mathrm{o}}, I_\vec{t}\cdot I_a, I_i, L(\mst_{\mathrm{g}}), I_{\mathrm{p}}, \mst_{\mathbf{s}}[0], \mathbf{i}, I_{\mathrm{e}} + 1, \zeta, I_{\mathrm{w}}) & \text{if} \quad \mst_{\mathbf{s}}[0] \leqslant \st[I_{\mathrm{a}}]_{\mathrm{b}} \; \\ \quad &\wedge\; I_{\mathrm{e}} < 1024\\ \big(\st, \mst_{\mathrm{g}}, \varnothing\big) & \text{otherwise} \end{cases}$ \\
&&&& $\st^* \equiv \st \quad \text{except} \quad \st^*[I_{\mathrm{a}}]_{\mathrm{n}} = \st[I_{\mathrm{a}}]_{\mathrm{n}} + 1$ \\
% &&&& $A' \equiv A \Cup A^+$ which abbreviates: $A'_{\mathbf{s}} \equiv A_{\mathbf{s}} \cup A^+_{\mathbf{s}} \quad \wedge \quad A'_{\mathbf{l}} \equiv A_{\mathbf{l}} \cdot A^+_{\mathbf{l}} \quad \wedge$ \\
% &&&& $A'_{\mathbf{t}} \equiv A_{\mathbf{t}} \cup A^+_{\mathbf{t}} \wedge \quad A'_{\mathbf{r}} \equiv A_{\mathbf{r}} + A^+_{\mathbf{r}}$ \\
&&&& $\mst'_{\mathbf{s}}[0] \equiv x$ \\
&&&& where $x=0$ if the code execution for this operation failed due to an\\
&&&& \hyperlink{Exceptional_Halting_function_Z}{exceptional halting} (or for a \op{REVERT}) $\st' = \varnothing$, or $I_{\mathrm{e}} = 1024$ \\
&&&& (the maximum call depth limit is reached) or $\mst_{\mathbf{s}}[0] > \st[I_{\mathrm{a}}]_{\mathrm{b}}$ (balance of the caller\\
&&&& is too low to fulfil the value transfer); and otherwise $x=A(I_{\mathrm{a}}, \st[I_{\mathrm{a}}]_{\mathrm{n}}, \zeta, \mathbf{i} )$, the\\
&&&& address of the newly created account (\ref{eq:new-address}). \\
&&&& $\mst'_{\mathrm{i}} \equiv M(\mst_{\mathrm{i}}, \mst_{\mathbf{s}}[1], \mst_{\mathbf{s}}[2])$ \\
&&&& $\mst'_{\mathbf{o}} \equiv \emptystring$ \\
&&&& Thus the operand order is: value, input offset, input size. \\
\midrule
0xf1 & $\op{CALL}$ & 7 & 1 & Message-call into an account. \\
&&&& $\mathbf{i} \equiv \mst_{\mathbf{m}}[ \mst_{\mathbf{s}}[3] \dots (\mst_{\mathbf{s}}[3] + \mst_{\mathbf{s}}[4] - 1) ]$ \\
&&&& $(\st', g', A^+, \mathbf{o}) \equiv \begin{cases}
	\begin{array}{l}
		\Theta(\st, I_{\mathrm{a}}, I_{\mathrm{o}}, t, I_\vec{t}\cdot I_a,I_i, t, C_{\text{\tiny CALLGAS}}(\mst),\\ \quad I_{\mathrm{p}}, \mst_{\mathbf{s}}[2], \mst_{\mathbf{s}}[2], \mathbf{i}, I_{\mathrm{e}} + 1, I_{\mathrm{w}})\end{array} & 
		\text{if} \quad p\\ 
	(\st, g, \varnothing, \emptystring) & 
	\text{otherwise} 
\end{cases}$ \\
&&&&$ \begin{array}{l}
	p\eqdef \mst_{\mathbf{s}}[2] \leqslant \st[I_{\mathrm{a}}]_{\mathrm{b}} \;\wedge \; I_{\mathrm{e}} < 1024  
	\;\wedge \; \mathsf{Type}_{t} \in \set{\typereserved,\typenormal,\typecontract} \\ 
	\quad \quad \wedge\; (t\notin I_\vec{t}\;\vee\; t= I_a\;\vee\; (C_{\text{\tiny CALLGAS}}(\st, \mst)\le G_{callstipend} \;\wedge\; \mst_{\mathbf{s}}[4]=0))
\end{array}$\\
&&&& $n \equiv \min(\{ \mst_{\mathbf{s}}[6], \lVert \mathbf{o} \rVert\})$ \\
&&&& $\mst'_{\mathbf{m}}[ \mst_{\mathbf{s}}[5] \dots (\mst_{\mathbf{s}}[5] + n - 1) ] = \mathbf{o}[0 \dots (n - 1)]$ \\
&&&& $\mst'_{\mathbf{o}} = \mathbf{o}$ \\
&&&& $\mst'_{\mathrm{g}} \equiv \mst_{\mathrm{g}} + g'$ \\
&&&& $\mst'_{\mathbf{s}}[0] \equiv x$ \\
&&&& $A' \equiv A \Cup A^+$ \\
&&&& $t \equiv \mst_{\mathbf{s}}[1] \mod 2^{160}$ \\
&&&& where $x=0$ if the code execution for this operation failed due to an\\
&&&& \hyperlink{Exceptional_Halting_function_Z}{exceptional halting} (or for a \op{REVERT}) $\st' = \varnothing$ or if $p=\false$ \\
&&&& which means \cvm prevents this call; $x=1$  otherwise. \\
&&&& $\mst'_{\mathrm{i}} \equiv M(M(\mst_{\mathrm{i}}, \mst_{\mathbf{s}}[3], \mst_{\mathbf{s}}[4]), \mst_{\mathbf{s}}[5], \mst_{\mathbf{s}}[6])$ \\
&&&& Thus the operand order is: gas, to, value, in offset, in size, out offset, out size. \\
&&&&\linkdest{tiny CALL}{} $C_{\text{\tiny CALL}}(\st, \mst) \equiv C_{\text{\tiny GASCAP}}(\st, \mst) + C_{\text{\tiny EXTRA}}(\st, \mst)$ \\
&&&& $C_{\text{\tiny CALLGAS}}(\st, \mst) \equiv  \begin{cases}
C_{\text{\tiny GASCAP}}(\st, \mst) + G_{callstipend} & \text{if} \quad \mst_{\mathbf{s}}[2] \neq 0 \\
C_{\text{\tiny GASCAP}}(\st, \mst) & \text{otherwise}
\end{cases}$ \\
&&&& $C_{\text{\tiny GASCAP}}(\st, \mst) \equiv \begin{cases}
\min\{ L(\mst_{\mathrm{g}} - C_{\text{\tiny EXTRA}}(\st, \mst)), \mst_{\mathbf{s}}[0] \} & \text{if} \quad \mst_{\mathrm{g}} \ge C_{\text{\tiny EXTRA}}(\st, \mst)\\
\mst_{\mathbf{s}}[0] & \text{otherwise}
\end{cases}$\\
&&&& $C_{\text{\tiny EXTRA}}(\st, \mst) \equiv G_{call} + C_{\text{\tiny XFER}}(\mst) + C_{\text{\tiny NEW}}(\st, \mst)$\\
&&&& $C_{\text{\tiny XFER}}(\mst) \equiv \begin{cases}
G_{callvalue} & \text{if} \quad \mst_{\mathbf{s}}[2] \neq 0 \\
0 & \text{otherwise}
\end{cases}$ \\
&&&& $C_{\text{\tiny NEW}}(\st, \mst) \equiv \begin{cases}
G_{newaccount} & \text{if} \quad \mathtt{DEAD}(\st, \mst_{\mathbf{s}}[1] \mod 2^{160}) \wedge \mst_{\mathbf{s}}[2] \neq 0 \\
0 & \text{otherwise}
\end{cases}$ \\
\end{tabu}

\begin{tabu}{\usetabu{opcodes}}
\midrule
0xf2 & $\op{CALLCODE}$ & 7 & 1 & Message-call into this account with an alternative account's code. \\
&&&& Exactly equivalent to $\op{CALL}$ except: \\
&&&& $(\st', g', A^+, \mathbf{o}) \equiv 
\begin{cases}
	\begin{array}{l}
		\Theta(\st^*, I_{\mathrm{a}}, I_{\mathrm{o}}, I_{\mathrm{a}}, I_\vec{t}\cdot I_a, I_i, \\
		\quad t, C_{\text{\tiny CALLGAS}}(\mst),I_{\mathrm{p}}, \mst_{\mathbf{s}}[2], \\ 
		\quad  \mst_{\mathbf{s}}[2], \mathbf{i}, I_{\mathrm{e}} + 1, I_{\mathrm{w}})
	\end{array} 
		& \text{if} \quad p \\ 
	(\st, g, \varnothing, \emptystring) & \text{otherwise} 
\end{cases}$ \\
&&&& where $p\eqdef \mst_{\mathbf{s}}[2] \leqslant \st[I_{\mathrm{a}}]_{\mathrm{b}} \;\wedge\; I_{\mathrm{e}} < 1024 \;\wedge\; \mathsf{Type}_{t} \in \set{\typereserved,\typenormal,\typecontract}.$ \\
&&&& Note the change in the fourth parameter to the call $\Theta$ from the 2nd stack value \\
&&&& $\mst_{\mathbf{s}}[1]$ (as in $\op{CALL}$) to the present address $I_{\mathrm{a}}$. This means that the recipient is in\\
&&&& fact the same account as at present, simply that the code is overwritten.\\
\midrule
\linkdest{RETURN}{}0xf3 & $\op{RETURN}$ & 2 & 0 & Halt execution returning output data. \\
&&&& $H_{\text{\tiny RETURN}}(\mst) \equiv \mst_{\mathbf{m}}[ \mst_{\mathbf{s}}[0] \dots ( \mst_{\mathbf{s}}[0] + \mst_{\mathbf{s}}[1] - 1 ) ]$ \\
&&&& This has the effect of halting the execution at this point with output defined.\\
&&&& See section \ref{sec:exe model}. \\
&&&& $\mst'_{\mathrm{i}} \equiv M(\mst_{\mathrm{i}}, \mst_{\mathbf{s}}[0], \mst_{\mathbf{s}}[1])$ \\
\midrule
0xf4 & $\op{DELEGATECALL}$ & 6 & 1 & Message-call into this account with an alternative account's code, but\\
&&&& persisting the current values for {\it sender} and {\it value}. \\
&&&& Compared with $\op{CALL}$, $\op{DELEGATECALL}$ takes one fewer arguments. The\\
&&&& omitted argument is $\mst_{\mathbf{s}}[2]$. As a result, $\mst_{\mathbf{s}}[3]$, $\mst_{\mathbf{s}}[4]$, $\mst_{\mathbf{s}}[5]$ and $\mst_{\mathbf{s}}[6]$ in the\\
&&&& definition of $\op{CALL}$ should respectively be replaced with $\mst_{\mathbf{s}}[2]$, $\mst_{\mathbf{s}}[3]$, $\mst_{\mathbf{s}}[4]$ and\\
&&&& $\mst_{\mathbf{s}}[5]$. Otherwise it is equivalent to $\op{CALL}$ except:\\
&&&& $(\st', g', A^+, \mathbf{o}) \equiv 
\begin{cases}
	\begin{array}{l}
		\Theta(\st^*, I_{\mathrm{s}}, I_{\mathrm{o}}, I_{\mathrm{a}}, I_\vec{t}\cdot I_a, I_i ,t, C_{\text{\tiny CALLGAS}}(\mst), \\\quad I_{\mathrm{p}}, 0, I_{\mathrm{v}}, \mathbf{i}, I_{\mathrm{e}} + 1, I_{\mathrm{w}})
	\end{array} & \text{if} \quad p
	 \\
	(\st, g, \varnothing, \emptystring) & \text{otherwise} 
\end{cases}$ \\
&&&& where $p\eqdef I_{\mathrm{v}} \leqslant \st[I_{\mathrm{a}}]_{\mathrm{b}} \;\wedge\; I_{\mathrm{e}} < 1024 \;\wedge\; \mathsf{Type}_{t} \in \set{\typereserved,\typenormal,\typecontract}.$ \\
&&&& Note the changes (in addition to that of the fourth parameter) to the second \\
&&&& and ninth parameters to the call $\Theta$.\\
&&&& This means that the recipient is in fact the same account as at present, simply\\
&&&& that the code is overwritten {\it and} the context is almost entirely identical.\\
\midrule
\linkdest{CREATE2}{} 0xf5 & $\op{CREATE2}$ & 4 & 1 & Create a new account with associated code. \\
&&&& Exactly equivalent to \hyperlink{create}{$\op{CREATE}$} except: \\
&&&& The salt $\zeta \equiv \mst_{\mathbf{s}}[3]$.\\
\midrule
0xfa & $\op{STATICCALL}$ & 6 & 1 & Static message-call into an account. \\
&&&& Exactly equivalent to $\op{CALL}$ except: \\
&&&& The argument $\mst_{\mathbf{s}}[2]$ is replaced with $0$. \\
&&&& The deeper argument $\mst_{\mathbf{s}}[3]$, $\mst_{\mathbf{s}}[4]$, $\mst_{\mathbf{s}}[5]$ and $\mst_{\mathbf{s}}[6]$ are respectively replaced \\
&&&& with $\mst_{\mathbf{s}}[2]$, $\mst_{\mathbf{s}}[3]$, $\mst_{\mathbf{s}}[4]$ and $\mst_{\mathbf{s}}[5]$. \\
&&&& The last argument of $\Theta$ is $\bot$. \\
\midrule
0xfd & $\op{REVERT}$ & 2 & 0 & Halt execution reverting state changes but returning data and remaining gas. \\
&&&& The effect of this operation is described in (\ref{eq:X-def}). \\
&&&& For the gas calculation, we use the memory expansion function, \\
&&&& $\mst'_{\mathrm{i}} \equiv M(\mst_{\mathrm{i}}, \mst_{\mathbf{s}}[0], \mst_{\mathbf{s}}[1])$ \\
\midrule
0xfe & $\op{INVALID}$ & $\varnothing$ & $\varnothing$ & Designated invalid instruction. \\
\midrule
0xff & $\op{SUICIDE}$ & 1 & 0 & Halt execution and register account for later deletion. \\
&&&& $(\st',A')\eqdef \Psi(\st,A)$\\ 
&&&& where $\Psi$ is defined in section~\ref{sec:contract_destruct}.\\
&&&&\linkdest{C tiny SUICIDE}{} $C_{\text{\tiny SUICIDE}}(\st, \mst) \equiv G_{suicide} + \begin{cases}
G_{newaccount} & \text{if} \quad n \\
0 & \text{otherwise}
\end{cases}$ \\
&&&& $n \equiv \mathtt{DEAD}(\st^*, \mst_{\mathbf{s}}[0] \mod 2^{160}) \wedge \st[I_{\mathrm{a}}]_{\mathrm{b}} \neq 0$ \\
\bottomrule
\end{tabu}

% \begin{tabu}{\usetabu{opcodes}}
% 	\toprule

% \end{tabu}



\newpage
% !TEX root=./tech-specification.tex

\section{Multi-point Evaluation Hashing}
\label{app:mp_eval_hash}

\subsection{Definitions}
We employ the following definitions:

\begin{tabu*}{lcl}
\toprule
Name & Value & Description \\
\midrule
\linkdest{J__wordbytes}{}$J_{wordbytes}$ & $4$  & Bytes in word. \\
\linkdest{J__datasetinit}{}$J_{datasetinit}$ & $3\times 2^{31}$ & Bytes in dataset at genesis. \\
\linkdest{J__datasetgrowth}{}$J_{datasetgrowth}$ & $2^{24}$ & Dataset growth per stage. \\
\linkdest{J__cacheinit}{}$J_{cacheinit}$ & $3\times 2^{23}$ & Bytes in cache at genesis. \\
\linkdest{J__cachegrowth}{}$J_{cachegrowth}$ & $2^{16}$ & Cache growth per stage. \\
\linkdest{J__stage}{}$J_{stage}$ & $2^{19}$ & Epoches per stage. \\
\linkdest{J__mixbytes}{}$J_{mixbytes}$ & $128$ & mix length in bytes. \\
\linkdest{J__hashbytes}{}$J_{hashbytes}$ & $64$ & Hash length in bytes. \\
\linkdest{J__parents}{}$J_{parents}$ & $256$ & Number of parents of each dataset element. \\
\linkdest{J__cacherounds}{}$J_{cacherounds}$ & $3$ & Number of rounds in cache production. \\
\linkdest{J__accesses}{}$J_{accesses}$ & $64$ & Number of accesses in hashimoto loop. \\
\bottomrule
\end{tabu*}

\subsection{Size of dataset and cache}
The size for the hash function's cache $\mathbf{c} \in \B^*$  and dataset $\mathbf{d} \in \B^*$ depend on the stage, which in turn depends on the block height $\head_{h}$.
\begin{equation}
 E_{stage}(\head_{h}) \eqdef \left\lfloor\frac{\head_{h}}{J_{stage}}\right\rfloor
\end{equation}
The size of the dataset growth by $J_{datasetgrowth}$ bytes, and the size of the cache by $J_{cachegrowth}$ bytes, every stage. In order to avoid regularity leading to cyclic behavior, the size must be a prime number. Therefore the size is reduced by a multiple of $J_{mixbytes}$, for the dataset, and $J_{hashbytes}$ for the cache.
\linkdest{d__size}{}Let $d_{size} = \lVert\dataset \rVert$ be the size of the dataset, 
which is calculated using
\begin{equation}
 d_{size} \eqdef E_{\mathrm{prime}}(J_{datasetinit} + J_{datasetgrowth} \cdot E_{stage} - J_{mixbytes}, J_{mixbytes})
\end{equation}
The size of the cache, $c_{size}$, is calculated using
\begin{equation}
 c_{size} \eqdef E_{\mathrm{prime}}(J_{cacheinit} + J_{cachegrowth} \cdot E_{stage} - J_{hashbytes}, J_{hashbytes})
\end{equation}
where $E_{\mathrm{prime}}(x, y)$ computes the greatest integer $x'$ below $x$ such that $x'/y$ is a prime (given that $x\gg y$).
\begin{equation}
	E_{\mathrm{prime}}(x, y) \eqdef 
	\begin{cases}
		x & \text{if $x / y$ is prime} \\
		E_{\mathrm{prime}}(x - 2 \cdot y, y) & \text{if $x / y$ is not prime but $x / y \in \N$} \\
		% 0 & \text{if $x / y<0$ (this case should not happen on valid input $x,y$)} \\
		E_{\mathrm{prime}}(\lfloor x/y \rfloor\cdot y, y) & \text{otherwise ($x$ is rounded to a multiple of $y$)}
	\end{cases}
\end{equation}
\subsection{Dataset generation}
In order to generate the dataset we need the cache $\mathbf{c}$, which is an array of bytes. It depends on the cache size  $c_{size}$ and the seed hash $\mathbf{s} \in \B_{256}$.
\subsubsection{Seed hash}
The seed hash is different for every stage. For the first stage it is the Keccak-256 hash of a series of 256 bits (32 bytes) of zeros. For every other stage it is always the Keccak-256 hash of the previous seed hash:
\begin{equation}
 \mathbf{s} \eqdef C_{seedhash}(\head_{h})
\end{equation}
\begin{equation}
 C_{seedhash}(\head_{h}) \eqdef \begin{cases}
\mathbf{0}_{256} & \text{if} \quad E_{stage}(\head_{h}) = 0 \quad  \\
\kec(C_{seedhash}(\head_{h} - J_{stage})) & \text{otherwise}
\end{cases}
\end{equation}
where $\mathbf{0}_{256}$ denotes $256$ bits of zeros.

\subsubsection{Cache}
The cache production process involves using the seed hash to first sequentially filling up $c_{size}$ bytes of memory, then performing $J_{cacherounds}$ passes of the RandMemoHash algorithm created by \cite{lerner2014randmemohash}. The initial cache $\mathbf{c'}$ will be constructed as follows.

Recalling that $\kec512$ denotes the Keccak-512 hash function whose output length is $512$ bits ($64$ single bytes), 
we define the array $\mathbf{c}_{i}$ as the $i$-th element of the initial cache:
\begin{equation}
	\mathbf{c}_{i} \eqdef 
	\begin{cases}
		\kec512(\mathbf{s}) & \text{if} \quad i = 0 \quad  \\
		\kec512(\mathbf{c}_{i-1}) & \text{otherwise}
	\end{cases}
\end{equation}

Let $n$ denote the number of elements in cache:
\begin{equation}
	n \eqdef \left\lfloor c_{size}/J_{hashbytes}\right\rfloor
\end{equation}
The initial cache $\mathbf{c'}$ can be defined as:
\begin{equation}
 \mathbf{c'}[i] \eqdef \mathbf{c}_{i}, \quad \forall \quad i \in\set{0,1,2,\dots, n-1}
\end{equation}


The cache $\mathbf{c}$,
consisting of $n$ items of $\kec512$ hash values, 
is calculated by performing $J_{cacherounds}$ rounds of the RandMemoHash algorithm to the initial cache $\mathbf{c'}$:
\begin{equation}
 \mathbf{c} \eqdef E_{cacherounds}(\mathbf{c'}, J_{cacherounds})
\end{equation}
\begin{equation}
	E_{cacherounds}(\mathbf{x}, y) \eqdef 
	\begin{cases}
		\mathbf{x} & \text{if} \quad y = 0 \quad  \\
		E_\text{\tiny RMH}(\mathbf{x}) & \text{if} \quad y = 1 \quad  \\
		E_{cacherounds}(E_\text{\tiny RMH}(\mathbf{x}), y -1 ) & \text{otherwise}
	\end{cases}
\end{equation}

Every single round of the RandMemoHash algorithm modifies each subset of the cache as follows:
\begin{equation}
	E_\text{\tiny RMH}(\mathbf{x}) \eqdef \big( E_{rmh}(\mathbf{x}, 0), E_{rmh}(\mathbf{x}, 1), ... , E_{rmh}(\mathbf{x}, n - 1) \big)\linkdest{E__cacherounds}{}
\end{equation}
\begin{multline}
	E_{rmh}(\mathbf{x}, i) \eqdef \kec512(\mathbf{x'}[(i - 1 + n) \mod n] \oplus \mathbf{x'}[\mathbf{x'}[i][0] \mod n]) \\
	\text{with} \quad \mathbf{x'} = \mathbf{x} \quad \text{except} \quad \mathbf{x'}[j] = E_{rmh}(\mathbf{x}, j), \quad \forall \quad j < i
\end{multline}

\subsubsection{Full dataset calculation} \label{app:dataset}
Essentially, we combine data from $J_{parents}$ pseudorandomly selected cache nodes, and hash that to compute the dataset. The entire dataset is then generated by a number of items, each of $J_{hashbytes}$ bytes in size:
\begin{equation}
\dataset[i] \eqdef E_{datasetitem}(\mathbf{c}, i), \quad \forall \quad i < \left\lfloor\frac{d_{size}}{J_{hashbytes}}\right\rfloor
\end{equation}
In order to calculate the single item we use an algorithm $E_\text{\tiny FNV}:\B_{32}\times\B_{32} \to \B_{32}$ inspired by the FNV hash \cite{FowlerNollVo1991FNVHash} in some cases as a non-associative substitute for XOR.
\begin{equation}
	E_\text{\tiny FNV}(\mathbf{x}, \mathbf{y}) \eqdef \left( (\mathbf{x} \times \mathrm{0x01000193}) \oplus \mathbf{y}\right) \mod 2^{32}
\end{equation}
The single item of the dataset can now be calculated by iteratively mixing items from the cache $\mathbf{c}$ as follows:
\begin{equation}
 E_{datasetitem}(\mathbf{c}, i) \eqdef 
 \kec512\Big(E_{parents}\big(\mathbf{c}, i, 0, \kec512(\mathbf{c}[i \mod n ] \oplus i)\big)\Big)
\end{equation}
\begin{equation}
  E_{parents}(\mathbf{c}, i, p, \mathbf{m}) \eqdef \begin{cases}
E_{parents}\big(\mathbf{c}, i, p +1, E_{mix}(\mathbf{m}, \mathbf{c}, i, p)\big) & \text{if} \quad p < J_{parents} -1 \\
E_{mix}(\mathbf{m}, \mathbf{c}, i, p) & \text{otherwise}
\end{cases}
\end{equation}
\begin{equation}
 E_{mix}(\mathbf{m}, \mathbf{c}, i, p) \eqdef E_\text{\tiny FNV}^*\Big(\mathbf{m}, \mathbf{c}[E_\text{\tiny FNV}(i \oplus p, ~ \mathbf{m}\big[p \mod \lfloor J_{hashbytes} / J_{wordbytes} \rfloor ]) \mod n \big] \Big)
\end{equation}
where $\mathbf{m}\in\B_{512}$ should be interpreted as an array of $\lfloor J_{hashbytes} / J_{wordbytes} \rfloor$ words (each of $J_{wordbytes}$ bytes, i.e. $32$ bits),
and $E_\text{\tiny FNV}^*$ denotes the element-wise invocation of $E_\text{\tiny FNV}$.

\subsection{Proof-of-work function}
\label{appsec:pow}

Essentially, we maintain a ``mix'' of $J_{mixbytes}$ bytes wide, and repeatedly sequentially fetch $J_{mixbytes}$ bytes from the full dataset and use the $E_\text{\tiny FNV}$ function to combine it with the mix. 
$J_{mixbytes}$ bytes of sequential access are used so that each round of the algorithm always fetches a full page from RAM, minimizing translation lookaside buffer misses which ASICs would theoretically be able to avoid.

If the output of this algorithm is below the desired target, then the nonce is valid. Note that the extra application of $\kec$ at the end ensures that there exists an intermediate nonce which can be provided to prove that at least a small amount of work was done; this quick outer PoW verification can be used for anti-DDoS purposes. It also serves to provide statistical assurance that the result is an unbiased, 256 bit number.

The $\mpethash$ function takes $\headernon$, which is the hash of the header excluding the {\bf mixHash} and  {\bf nonce} fields,
i.e. $\headernon\eqdef \kec\left(\rlp( \head_{-n} )\right)$,
together with the nonce $\head_{n}$
and the dataset $\dataset$ from \cref{app:dataset} as input.
The output of $\mpethash$ consists of an array with the compressed mix $\compressedmix$ as its first item and the Keccak-256 hash of the concatenation of the seed hash $\seedhash\in \B_{512}$, the compressed mix $\compressedmix\in \B_{256}$,
and the compressed multi-point mix $\mpmix\in\B_{32}$ as the second item:
\begin{equation}
	\mpethash(\headernon, \head_{n},\dataset) \eqdef 
	\set{\compressedmix, \kec\left(\seedhash \circ \compressedmix \circ \mpmix \right)}
\end{equation}

\subsubsection{Seed hash}
The seed hash $\seedhash\in \B_{512}=\Byte_{J_{hashbytes}}$ is defined on $\headernon$ and $\head_{n}$ as follows:
\begin{equation}
 \seedhash \eqdef 
 \seedhash(\headernon, \head_{n}) \eqdef \kec512\left(\headernon \circ E_{revert}(\head_{n})\right)
\end{equation}
where $E_{revert}(\head_{n})$ returns the reverted bytes sequence of the nonce $\head_{n}$ as $E_{revert}(\head_{n})[i] \eqdef \head_{n}\left[~\lVert \head_{n} \rVert -i~\right]$,
so that the $64$-bit nonce in big-endian order is changed to little-endian representation.

\subsubsection{Compressed mix}
The compressed mix $\compressedmix \in \B_{256}$ is obtained from the seed hash $\seedhash$ and the dataset $\dataset$:
\begin{equation}\label{eq:compress}
 \compressedmix \eqdef 
 \compressedmix\big(\seedhash,\dataset) \eqdef E_{compress}(E_{accesses}(\dataset, \mathbf{m}_0, \seedhash, 0), 0\big)
\end{equation}
where $\mathbf{m}_0$ and $E_{accesses}$, $E_{compress}$ are defined as follows.

The initial mix $\mathbf{m}_0$ is an array of $\lfloor J_{mixbytes} / J_{wordbytes} \rfloor$ many $32$-bit (i.e. $4$-byte) words 
obtained by replicating the seed hash $\seedhash$ for $n_{mix}$ times, 
with $n_{mix}$ defined as:
\begin{equation}
	n_{mix}\eqdef  \left\lfloor \frac{J_{mixbytes}}{J_{hashbytes}}\right\rfloor
\end{equation}
Formally, the initial mix 
$\mathbf{m}_0 \in \B_{512 \cdot n_{mix}}
=\left(\Byte_{J_{hashbytes}}\right)^{n_{mix}}$ is defined as:
\begin{equation}
	\mathbf{m}_0 \eqdef \underbrace{\seedhash \circ \cdots \circ \seedhash}_{\text{$n_{mix}$ many copies of $\seedhash$}}
\end{equation}

In order to add random dataset nodes to the mix, the $E_{accesses}$ function is used:
\begin{equation}
 	E_{accesses}(\mathbf{d}, \mathbf{m}, \mathbf{s}, i) 
 	\eqdef 
 	\begin{cases}
		E_{mixdataset}(\mathbf{d}, \mathbf{m},  \mathbf{s}, i) & \text{if} \quad i = J_{accesses} -1 \\
		E_{accesses}\big(\dataset, E_{mixdataset}(\mathbf{d}, \mathbf{m}, \mathbf{s}, i), \mathbf{s}, i + 1\big) & \text{otherwise}
	\end{cases}
\end{equation}
\begin{equation}
 E_{mixdataset}(\mathbf{d}, \mathbf{m}, \mathbf{s}, i) \eqdef E_\text{\tiny FNV}^*\big(\mathbf{m}, E_{newdata}(\mathbf{d}, \mathbf{m}, \mathbf{s}, i)\big)
\end{equation}
where $E_\text{\tiny FNV}^*$ denotes the element-wise invocation of $E_\text{\tiny FNV}$ on $32$-bit elements,
and $E_{newdata}$ returns an array consisting of $n_{mix}$ elements from the dataset $\dataset$ (so that the array is of same length as the intermediate mix $\mathbf{m}$, i.e. each of $J_{mixbytes} = n_{mix}\cdot J_{hashbytes}$ bytes):
\begin{equation}
 	E_{newdata}(\mathbf{d}, \mathbf{m}, \mathbf{s}, i)[j] 
 	\eqdef \dataset\left[ {p} \cdot n_{mix} + j\right], \quad \forall \quad 0\le j \le n_{mix}-1
\end{equation}
with the mixing index ${p}$ obtained as follows:
\begin{equation}
	{p} \eqdef 
	E_\text{\tiny FNV}\left(i \oplus \mathbf{s}[0], \mathbf{m}\left[i \mod \left\lfloor\frac{J_{mixbytes}}{J_{wordbytes}}\right\rfloor\right]\right) \mod \left\lfloor\frac{d_{size} / J_{hashbytes}}{n_{mix}}\right\rfloor
\end{equation}

The $E_{compress}$ function converts $J_{mixbytes}$-byte mix eventually returned by $E_{accesses}$, 
which is an array of $\lfloor J_{mixbytes} / J_{wordbytes} \rfloor$ words,
into the compressed mix $\compressedmix \in \Byte_{32}=\Byte_{\lfloor J_{mixbytes}/4 \rfloor}$ as follows:
\begin{equation}
	E_{compress}(\mathbf{m}, i) \eqdef 
	\begin{cases}
		\emptystring & \text{if} \quad i \geqslant \lVert \mathbf{m} \rVert - 4 \\
		E_\text{\tiny FNV}(E_\text{\tiny FNV}(E_\text{\tiny FNV}(\mathbf{m}[i], \mathbf{m}[i + 1]), \mathbf{m}[i + 2]), \mathbf{m}[i + 3]) 
		\circ E_{compress}(\mathbf{m}, i + 4) & \text{otherwise}
	\end{cases}
\end{equation}

\subsubsection{Multi-point mix}
The multi-point mix $\mpmix\in \B_{32}$ is calculated from the header $\headernon$
and $\seedhash$, $\dataset$ as follows:
\begin{equation}
	\mpmix \eqdef \mpmix\left( \headernon, \seedhash, \dataset \right)
	\eqdef  E_{\text{\tiny FNV}-compress}\left(E_{mp-eval}(a,b,c,w,\vec{z}),J_{accesses} \right)
\end{equation}
with arguments and functions described in the rest of this section.

Interpreting $\headernon$ as a byte array, 
input arguments $a,b,c,w\in \B_{32}$ are defined as follows:
\begin{align}
	a &\eqdef \headernon[0]\\
	b &\eqdef \headernon[1]\\
	c &\eqdef \headernon[2]\\
	w &\eqdef \headernon[3]
\end{align}
whereas the last argument $\vec{z}$ is an array of $J_{accesses}$ items drawn from $\B_{32}$.
More specifically, $\vec{z}[i]$ is defined as:
\begin{equation}
	\vec{z}[i] \eqdef 
	E_{parity}\left(E_{mp-accesses}(\mathbf{d}, \mathbf{m}_0, \seedhash, i, 0)\right),
	\quad \forall \quad 0\le i \le J_{accesses}-1
\end{equation}
where the $E_{mp-accesses}$ function is defined as:
\begin{equation}
	E_{mp-accesses}(\mathbf{d}, \mathbf{m}, \mathbf{s}, i, j)
	\eqdef 
 	\begin{cases}
		E_{mixdataset}(\mathbf{d}, \mathbf{m},  \mathbf{s}, i) & \text{if} \quad j \ge i \\
		E_{mp-accesses}\big(\dataset, E_{mixdataset}(\mathbf{d}, \mathbf{m}, \mathbf{s}, j), \mathbf{s}, i, j + 1\big) & \text{otherwise}
	\end{cases}
\end{equation}
and the $E_{parity}$ function is defined over $\Byte_{J_{mixbytes}} \to \B_{32}$ and 
it simply returns the parity of all items in the input array at word level (i.e. realizing $\Byte_{J_{mixbytes}}$ as ${\lfloor J_{mixbytes}/J_{wordbytes}\rfloor}$ elements each of $J_{wordbytes}$ bytes),  
formally defined as:
\begin{equation}
	E_{parity}(\vec{m}) \eqdef \oplus_{i=0}^{\lfloor J_{mixbytes}/J_{wordbytes}\rfloor -1} \vec{m}[i]
\end{equation}

The $E_{mp-eval}$ function evaluates the polynomial $E_{polynomial}$ on multiple points $\vec{x}[i]$ for $i\in \set{0,1,\dots, J_{accesses}-1}$ and returns the concatenation of $J_{accesses}$ elements drawn from $\B_{32}$ as follows:
\begin{equation}
	E_{mp-eval}(a,b,c,w,\vec{z}) \eqdef 
	E_{polynomial}\left(\vec{x}[0]\right) \circ E_{polynomial}\left(\vec{x}[1]\right) \circ \cdots \circ
	E_{polynomial}\left(\vec{x}[J_{accesses-1}]\right)
\end{equation}
where $\vec{x}$ is an array of $32$-bit integers defined as:
\begin{equation}
	\vec{x}[i] \eqdef a+b\cdot w^i + c\cdot w^{2i}, \quad \forall \quad 0\le i \le J_{accesses}-1
\end{equation}
and the $E_{polynomial}$ function is a polynomial with coefficients specified by $\vec{z}$:
\begin{equation}
	E_{polynomial}(x) \eqdef \left(\sum_{j=0}^{J_{accesses}-1} \vec{z}[j]\cdot x^j\right) \mod (10^9+7)
\end{equation}

The FNV-compress function $E_{\text{\tiny FNV}-compress}$ is defined over $\left(\B_{32}\right)^* \times \N \to \B_{32}$ 
and used to compress an array of $\B_{32}$ elements into a single $\B_{32}$ element:
\begin{equation}
	E_{\text{\tiny FNV}-compress}(\vec{m},i) \eqdef 
	\begin{cases}
		\vec{0}_{32} & \text{if} \quad i < 1 \\
		E_\text{\tiny FNV}\left( 
		E_{\text{\tiny FNV}-compress}(\vec{m},i-1) ,
		\vec{m}[i-1]\right)
		& \text{otherwise}
	\end{cases}
\end{equation}


%----------------------------------------------------------------------------------------

\end{document}